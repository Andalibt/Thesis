\chapter{Future nEDM Measurement at TRIUMF}

Finding a non-zero neutron EDM is directly linked to the extra sources
of CP violation beyond the standard model. The TUCAN collaboration
proposes a world-leading experiment to measure the nEDM, improving the
precision by a factor of thirty compared to the present world’s best
experimental result. The current nEDM experiments suffer from low UCN
statistics. As a result, TUCAN has intended to build the strongest UCN
source in the world. To achieve this goal extensive studies of the
current vertical UCN source have been conducted~(See
Chapters~\ref{chap:UCNattriumf} and ~\ref{chap:UCNresult}).

To measure the neutron EDM, an ensemble of polarized UCN are put in
the presence of aligned electric and magnetic fields. The hamiltonian
of the interaction of the UCN with electric and magnetic fields are
described in Eqn.~\ref{eqn:hamiltonian}.  The larmor precession
frequency of UCN is then measured in two oriantations of parallel and
anti-parallel electric and magnetic fields. For the parallel $\bf{E}$
and $\bf{B}$ fields the Larmor precession frequency of UCN is written
as
\begin{equation}
\label{eqn:parallelEandB}
  h \nu_{\uparrow \uparrow} = 2 \mu_n \vert {\bf{B^{\uparrow \uparrow}}} \vert+ 2 d_n\vert \bf{ E^{\uparrow \uparrow}} \vert
\end{equation}
and for anti-parallel  $\bf{E}$ and $\bf{B}$ fields it is
\begin{equation}
\label{eqn:antiparallelEandB}
  h \nu_{\uparrow \downarrow} = 2 \mu_n \vert {\bf{B^{\uparrow \downarrow}}} \vert+ 2 d_n\vert \bf{ E^{\uparrow \downarrow}} \vert.
\end{equation}
Here $\uparrow \uparrow$ indicates the parallel Electric and Magnetic
fields and $\uparrow \downarrow$ represent the anti-parallel
orientation of those fields.
A nonzero nEDM is then extracted from any frequecy shift between these
two measurements:
\begin{equation}
  \label{eqn:dn}
  d_n = \frac{h \left( \nu_{\uparrow \uparrow} - \nu_{\uparrow \downarrow} \right) - 2 \mu_n \left( \vert {\bf{B^{\uparrow \uparrow}}} \vert -\vert {\bf{B^{\uparrow \downarrow}}} \vert \right)}{2 \left(\vert \bf{ E^{\uparrow \uparrow}} \vert - \vert \bf{ E^{\uparrow \downarrow}} \vert \right)}
\end{equation}
The main reason to employ this method is because it is impossible to
completely eliminate the $\bf{B}$ field to extract the neutron
EDM. These measurements are either performed in two adjacent volumes
with
$\vert E^{\uparrow \uparrow} \vert = - \vert E^{\uparrow \downarrow}
\vert$ and
$\vert B^{\uparrow \uparrow}\vert - \vert B^{\uparrow \downarrow} = 0
$ or measured in the same volume where the configuration of the fields
change in time. In the first case it is essential to make sure that
the magnetic field inside both volumes are the same and there is no
field gradient and in the second method it is essential to make sure
that the magnetic field is stable in time.

\section{Ramsey Method of Separated Oscillating
  Fields\label{sec:Ramsey}}

The Ramsey method of separated oscillating fields is the well-known
measurement technique to extract the neutron EDM. Ramsey obtained an
expression for the quantum mechanical transition probability of a
system between two states when the system subjected to such separated
oscillating
fields~\cite{ramsey1950}. Fig.~\ref{fig:ramsey}~\cite{Schmidt2016} left
shows a cycle of measurement. An ensemble of polarized UCN with the
inital spin $\ket{ \uparrow}$ are exposed to a DC magnetic field of
$B_0$.  A first RF pulse of $B_1 \cos (\omega_{rf}t)$ prependicular to
the $B_0$ field tips the spin of the neutrons to the transverse
plane. The neutrons precess freely with their Larmor precession
frequency $\omega_0$ for some time $T$ while accumulating a phase of
$\phi = \gamma_n BT$. Then again the second oscillating magnetic field
pulse of $B_1 \cos (\omega_{rf}t)$ is applied to the neutron
ensemble. The essential idea is to compare the phase $\phi$ with
$\omega_{rf}T$ and if they are identical then
$B= \omega_{rf} / \gamma_n$.

\begin{figure}[h]
  \centering
  \includegraphics[width=1.0\textwidth]{ramsey.png}
  \caption{Ramsey method of separated oscillating fields. Left shows
    the scheme of a measurement procedure and right shows the data
    points. The blue points are the UCN counts with the spin up and
    the red points are the UCN with spin down (data from the PSI-nEDM
    collaboration). The width at half height~$\Delta \nu$ of the
    central fringe is approximately $1/2T$, the four vertical lines
    indicate the working points.}
  \label{fig:ramsey}
\end{figure}

The probability to find the UCN with spin up is
\begin{equation}
  \label{eqn:spinup}
  P(T, \omega_{rf}) = \bra{\uparrow} U(T, \omega_{rf}) \ket{\uparrow}
  = 1 - \frac{4 \omega_1^2}{\Omega^2} \sin ^2 \frac{\Omega t_{\pi/2}}{2} \left[ \frac{\Delta}{\Omega} \sin  \frac{\Omega t_{\pi/2}}{2} \sin \frac{T\Delta}{2} - \cos  \frac{\Omega t_{\pi/2}}{2} \cos\frac{T\Delta}{2} \right]^2,
\end{equation}
where $U(T, \omega_{rf})$ is the time evolutions operator,
$\omega_1 = - \gamma_n B_1$, $\Delta = \omega_{rf} - \omega_0$, and
$\Omega = \sqrt{\Delta^2 + \omega_1^2}$. When the spin-flipping pulses
are optimized we would have $\gamma_n B_1 t_{\pi/2} = \pi / 2$. In
this case the central fringe range ($\Delta \ll \omega_1$) and
Eqn.~\ref{eqn:spinup} simplifies to
\begin{equation}
  P(T, \omega_{rf}) = \frac{1}{2} \left( 1 - \cos(T\Delta) \right).
\end{equation}
In a real measurement with $N$ UCN inside a magnetic field region this becomes
\begin{equation}
  \label{eqn:Nup}
N^{\uparrow} = \frac{N}{2} \left\lbrace 1 - \alpha(T) \cos \left[ (\omega_{rf} - \gamma_n B_0 ) \cdot \left(T+\frac{T+4t_{\pi/2}}{\pi}\right)\right]\right\rbrace,
\end{equation}
where $\alpha$ is the visibility of the central fringe with spin either up or down
\begin{equation}
  \label{eqn:visibility}
  \alpha^{\uparrow /\downarrow} = \frac{N_{max}^{\uparrow /\downarrow} - N_{min}^{\uparrow /\downarrow}}{N_{max}^{\uparrow /\downarrow}+ N_{min}^{\uparrow /\downarrow}}.
\end{equation}
The term $4t_{\pi/2}/\pi$ is necessary to account for field
inhomogeneities of $B_1$ and $B_0$ which become relevant when the
pulse lenght $t_{\pi/2}$ is finite. The graph in Fig.~\ref{fig:ramsey}
shows the Ramsey interference pattern by scanning $\omega_{rf}$ while
everything else is kept the same. In actual nEDM measurements, only 4
points with the highest sensitivity are measured. These points are
refered to as the working points. For each configuration of the
electric and magnetic fields (parallel or anti-parallel)
Eqn.~\ref{eqn:Nup} is fitted to the data to extract the Larmor
frequency. Taking the differences of those Larmor frequencies then
give access to the neutron EDM
\begin{equation}
  \label{eqn:fitteddn}
  d_n = \frac{\hbar (\omega_0 ^{\uparrow \uparrow} - \omega_0 ^{\uparrow \downarrow})}{2(E^{\uparrow \uparrow} - E^{\uparrow \downarrow})} = \frac{\hbar \Delta \omega}{4E}.
\end{equation}
with the assumption that the magnetic field is constant~(See
Eqn.~\ref{eqn:dn}).
The statistical sensitivity of nEDM measurement is
\begin{equation}
  \label{eqn:dnsensitivity}
\sigma(d_n) = \frac{\hbar}{2 \alpha T E \sqrt{\bar{N}}}
\end{equation}
where $\bar{N}$ is the average total number of detected UCN.

\section{ TRIUMF nEDM Components}
The future nEDM experiment at TRIUMF will use a room-temperature nEDM
apparatus, connected to a horizontal cryogenic UCN
source~Fig~\ref{fig:triumfEDM}. 
The cyclotron at TRIUMF produces a~$\sim$~500~MeV proton beam. Protons
are guided to the spallation target using a variety of
magnets. Spallation neutrons are moderated and converted to UCN in a
superfluid He-II volume, which diffuse through UCN guides to the nEDM
measurement cell.

In 2016 the vertical UCN source from RCNP in Japan was shipped to TRIUMF for
the resarch towards the developement of the new horizontal UCN source.
The UCN beamline and the current vertical UCN source
are described in Chapter~\ref{chap:UCNattriumf}. The result of the
first set of UCN experiments with the vertical source is available in
Chapter~\ref{chap:UCNresult}.

\begin{figure}[h]
  \centering
  \includegraphics[width=1.0\textwidth]{edmtriumf.png}
  \caption{Conceptual design of the proposed UCN source and nEDM
    experiment. Protons strike a tungsten spallation target. Neutrons
    are moderated in the LD$_2$ cryostat and become UCN in a
    superfluid $^4$He bottle, which is cooled by another cryostat
    located farther downstream. UCN pass through guides and the
    superconducting magnet (SCM) to reach the nEDM experiment located
    within a magnetically shielded room (MSR). Simultaneous spin
    analyzers (SSA’s) detect the UCN at the end of each nEDM
    experimental cycle.  }
  \label{fig:triumfEDM}
\end{figure}

A brief description of each nEDM component is presented below.








%Based on the previous measurements of nEDM, the dominant systematic
%uncertainty is due to the magnetic field insability and
%inhomogeneity. As a result, different types of magnetic shielding is
%needed to meet the magnetic requirements.

% For after I am back
% I can use the old CDR and my thesis proposal to write some components
% I can use another CDR to write about high voltage and stuff



\subsection{Magnetic Components}
To chieve the desired sensitivity of ~$10^{-27}$~e$\cdot$cm an
extremely stable and homogeneuos $B_0$ magnetic field is required. The
magnetic stability upper limit for TUCAN's nEDM measurement is 1~pT
and the homogeneity of 1~nT/m. Because of the challenges to achieve
this level of magnetic stability co-magnetometers will be used to
correct for the $B_0$ field fluctuations. To achieve these
specifications, both active and passive shielding will be utilized to
nullify the uncontrolled and time-varying external fields. The desired
internal magnetic field will be generated by using uniform and shim
coils. Fig.~\ref{fig:magneticscheme} shows the schematic drawing of
the magnetic components of the TUCAN nEDM experiment. Each magnetic
component is explained below.

\begin{figure}[h!]
  \centering
  \includegraphics[width=0.7\textwidth]{magneticscheme.png}
  \caption{Schematic drawing for the TUCAN nEDM magnetics. From
    Outside in: The active compensation system followed by several
    layers of magnetically shielded room and passive shields nullify
    the environmental magnetic field. The magnetometers inside the
    active shielding monitor the changes in the magnetic field
    internal to that region. The internal coil system~($B_0$ and $B_1$
    coils) generate the magnetic fields for the Ramsey cycle. The UCN
    and the co-magnetometers are internal to the coils.  }
  \label{fig:magneticscheme}
\end{figure}



\subsubsection{Active Shielding}

The magnetic environment at the location of the planned nEDM
experiment at TRIUMF is dominated by a 400~$\mu$T static field due to
the main cyclotron at TRIUMF with 1 to 100-nT fluctuations due to the
other external magnetic sources such as the electrical equipment or
the displacement of large magnetic objects~(e.g., vehicle traffic).
The TUCAN's plan is to reduce the static field to less than 1~$\mu$T
using dedicated compensation coils and constant-current supplies with
a readily achievable steability of $10^{-3}$ and to reduce the
remaining static field and fluctuations by up to a factor of 100
through a separate set of compensation coils and current supplies
using fluxgate magnetometers for magnetic feedback. The fluxgate
sensors will be placed in the region between the compensation coils
and the passive shields as shown in Fig.~\ref{fig:magneticscheme}.  A
prototype active compensation system has been built at the University
of Winnipeg based on Refs.~\cite{beatrice,afach2014dynamic}. The
system employs a set of coils centered around a cylindrical passive
magnetic shield system using four 3-axis fluxgates for feedback~(See
Fig.~\ref{fig:prototype_active}). Overall, the active shielding
system should be able to reduce the net background magnetic field to
the level of tens of nT over the volume of the nEDM cell.


\subsubsection{Passive Shielding}
The passive shielding system nullifies the residual background fields
to the pT level. It will be a two-stage system: (1) a 2-layer
magnetically shielded room~(MSR) with (2) a smaller 3-layer shield
that fits inside the room and surrounds the nEDM apparatus. The
innermost layer also serves as a return yoke for the magnetic flux
generated by the internal coils for the shield-coupled coil desings. A
degaussing~(idealization) system will be used to stabilize the
shields. A combined DC shielding factor of the order of $10^6$ is
expected. In principle, by utilizing both active and passive
shielding, the magnetic field from external sources will be reduced to
the level of tens of fT over the volume of the nEDM cell.  There are
two prototype four-layer passive shields at the University of
Winnipeg. The shields are now used to facilitate a variety of magnetic
field R\&D. These are made of high permeability material. In addition,
there are three small witness cylinders which are made of the same
material and annealed in the same oven as the large passive
shields. The design principles behind the small shield, shielding
factor measurements, and comparison to simulation are described in
Ref.~\cite{martin2015large}.  The witness cylinders are used to
evaluate the temperature dependence of the shield material properties,
which could be an important consideration for internal field
stability~(See Chapter~\ref{muofT}).


\subsubsection{Internal Coils}
For internal coils, self-shielded $B_0$ coils and shim coils are
considered surrounding the nEDM cells since they provide immunity from
the field perterbations induced by changes in the magnetic
permeability of the passive shields arising from temperature
fluctuations~(See Chapter~\ref{chap:muofT}).  High-precision current
supplies ($\sim$~1~ppm) will be used to drive all internal coils,
regardless of design.  AC coils will apply π/2 pulses for the UCN and
comagnetometer species, to initiate free spin precession.






\subsection{High Voltage EDM Cells}
The nEDM measurement volume, consists of two storage cells to enable
simultaneous measurements with both up and down orientations of the
electric field~(See Fig.~\ref{fig:HVcell}). Each cell consists of
electrodes separated by a cylindrical wall of dielectric insulator,
appropriately coated for UCN compatibility. An electric field of
12~kV/cm will be created between the electrodes with minimal leakage
current~($\<$~10 pA).  The storage cells will be housed inside a
non-magnetic vacuum chamber providing insulating vacuum for the high
voltage applied to the central electrode which separates the two
cells.

\begin{figure}[h!]
  \centering
  \includegraphics[width=0.7\textwidth]{HVcell.png}
  \caption{3D drawing of the double EDM cell with vacuum chamber and
    UCN guides}
  \label{fig:HVcell}
\end{figure}



There are two grounded electrodes integrated into the lids of a vacuum
chamber. These electrodes are separated by an insulating side wall,
containing the particles of interest, with inlets for UCN, Xe and
Hg. The insulator must have a large dielectric strength and low
permittivity. Therefore, it is designed to be made of deuterated
polystyrene with quartz windows to allow optical access, coated by
deuterated polyethylene. These materials combine high Fermi
potentials, UV transparency as well as good dielectric strength.
Ports will allow the introduction of the $^{199}$Hg comagnetometer
atoms and UCN into the cell. The optical readout of the
comagnetometers requires UV-transparent windows in the insulating side
wall. The use of two cells with a central electrode allows first-order
compensation of magnetic field drifts and a measurement of the
magnetic field gradient.


\subsection{Dual Comagnetometer}
To measure any changes in the precesion $B_0$ field, a dual-species
$^{129}$Xe/ $^{199}$Hg Comagnetometer will be used. Here, polarized
$^{129}$Xe and $^{199}$He are simultaneously introduced along with the
UCN in the nEDM cell. Comagnetometry offers the only way to correct
for false EDMs caused by leakage currents.  Each atomic species is
polarized using optical pumping techniques. Polarized atoms are
introduced into the nEDM cell at the same time as UCN, and the
spin-precession frequencies of both species are measured
simultaneously. The atoms are expected to have smaller EDMs than the
neutrons, and so their precession frequencies may be used to normalize
magnetic field drifts.  The design of the $^{199}$Hg comagnetometer
will be similar to that employed in the previous ILL
experiment~\cite{Baker2006,Griffith2009}. The nuclear magnetization of
Xe will be measured by sensing decay light after two-photon
excitation~\cite{momose2014development}.


\subsection{UCN Handling and Transport}
Fig.~\ref{fig:UCNdelivery} shows the UCN transport to the EDM cell.
UCN will be transported out of the source by specular reflection via
guides with special coatings compatible with UCN transport and
polarization. Special coatings such as NiMo, NiP and DLC are top
candidates because of their high Fermi potential , small absorption
and inelastic upscattering, and good specularity.  A superconducting
magnet (SCM) accelerates polarized UCN through barrier foils to a
vacuum volume at room temperature. The UCN are then transported to the
nEDM experiment by additional guides.



\begin{figure}[h!]
  \centering
  \includegraphics[width=1.0\textwidth]{UCNdelivery.png}
  \caption{UCN delivery and the nEDM experiment. UCN exit the source
    by passing through the SCM spin polarizer and UCN switch and
    detector system, where they then enter the proposed Phase 2 nEDM
    experiment. UCN are loaded into the measurement cells within a
    MSR/coil system. At the end of the measurement cycle, UCN are
    counted by simultaneous spin analyzers (SSA’s) including
    detectors. An ambient magnetic compensation system and thermally
    controlled room which will surround the nEDM apparatus~(not
    shown). For scale, the innermost layer of the MSR is a 1.8~m
    side-length cube.}
  \label{fig:UCNdelivery}
\end{figure}

\subsection{UCN Detection}
The detector system will consist of a split UCN guide, two magnetized
iron foils, adiabatic-fast- passage spin-flippers, and UCN counters
based on $^6$Li scintillating glass coupled to photomultiplier
tubes. This is based on the detector used in the PSI UCN
experiments~\cite{Ban2009}. This configuration allows simultaneous
counting of both UCN spin states and hence extraction of the UCN
polarization with maximum efficiency. A prototype detector, based on
scintillating lithium glass, and capable of handling the highest rates
of UCN expected with the TRIUMF source has been developed and tested
in the highest rate UCN beam available at
PSI~\cite{jamieson2017characterization}~(See
Chapter~\ref{chap:UCNattriumf}).
%\begin{description}
%\item{An introduction about the long term nEDM effort at TRIUMF, what
% the plan is, when it will start (roughly). I guess I can probably
% get this information from some proposals. I am not sure how much
% detail should go here.}
  
%\item{How the EDM experiment is actually done, talk about different
%  components of the system. Here is where I talk about the Ramsey
%  cycle ...}
  
%\item{nEDM measurement systematic effects: This is where I talk about
%  the GPE and ... . Basically here is to kind of motivate that we need
%  to have stable magnetic fields and we need lots of neutrons.}
  
%\item{Introduction to the magnetic stability requirements at
%  TRIUMF. What I mean is that there is 400 $\mu$T background field at
%  TRIUMF. Hopefully we have a field map of the area soon(?).}
  
%\item{From ouside in: Magnetically shielded room, what is the status
%  of that, are we going to have it? when? How good is it going to be
%  compared to the other ones worldwide? Why is it designed that way?
%  What is the design? Drawings of it. General question: Some of these
%  are about things that will happen in the future and I have not
%  worked on them. Should they even go to my thesis? I feel I have to
%  say a little about this since my thesis is nEDM related and it is
%  part of it.}
  
%\item{Passive shieldings: Again same questions as above, motivate for
%  the next chapter}

  
%\item{Say what will be discussed in the two coming chapters}
  
%\item{what else?}
%\end{description}