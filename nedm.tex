\chapter{Future nEDM Measurement at TRIUMF}

Finding a non-zero neutron EDM is directly linked to the extra sources
of CP violation beyond the standard model. The TUCAN collaboration
proposes a world-leading experiment to measure the nEDM, improving the
precision by a factor of thirty compared to the present world’s best
experimental result. The current nEDM experiments suffer from low UCN
statistics. As a result, TUCAN has intended to build the strongest UCN
source in the world. To achieve this goal extensive studies of the
current vertical UCN source have been conducted~(See
Chapters~\ref{chap:UCNattriumf} and ~\ref{chap:UCNresult}).

To measure the neutron EDM, an ensemble of polarized UCN are put in
the presence of aligned electric and magnetic fields. The hamiltonian
of the interaction of the UCN with electric and magnetic fields are
described in Eqn.~\ref{eqn:hamiltonian}.  The larmor precession
frequency of UCN is then measured in two oriantations of parallel and
anti-parallel electric and magnetic fields. For the parallel $\bf{E}$
and $\bf{B}$ fields the Larmor precession frequency of UCN is written
as
\begin{equation}
\label{eqn:parallelEandB}
  h \nu_{\uparrow \uparrow} = 2 \mu_n \vert {\bf{B^{\uparrow \uparrow}}} \vert+ 2 d_n\vert \bf{ E^{\uparrow \uparrow}} \vert
\end{equation}
and for anti-parallel  $\bf{E}$ and $\bf{B}$ fields it is
\begin{equation}
\label{eqn:antiparallelEandB}
  h \nu_{\uparrow \downarrow} = 2 \mu_n \vert {\bf{B^{\uparrow \downarrow}}} \vert+ 2 d_n\vert \bf{ E^{\uparrow \downarrow}} \vert.
\end{equation}
Here $\uparrow \uparrow$ indicates the parallel Electric and Magnetic
fields and $\uparrow \downarrow$ represent the anti-parallel
orientation of those fields.
A nonzero nEDM is then extracted from any frequecy shift between these
two measurements:
\begin{equation}
  \label{eqn:dn}
  d_n = \frac{h \left( \nu_{\uparrow \uparrow} - \nu_{\uparrow \downarrow} \right) - 2 \mu_n \left( \vert {\bf{B^{\uparrow \uparrow}}} \vert -\vert {\bf{B^{\uparrow \downarrow}}} \vert \right)}{2 \left(\vert \bf{ E^{\uparrow \uparrow}} \vert - \vert \bf{ E^{\uparrow \downarrow}} \vert \right)}
\end{equation}
The main reason to employ this method is because it is impossible to
completely eliminate the $\bf{B}$ field to extract the neutron EDM. In
such measurements it is of great importance to have a very stable and
homogenoeus magnetic fields.

\section{Magnetic Stability Requirements}
\subsection{Active Shielding}

\subsection{Magnetically Shielded room}

\subsection{Passive Shielding}


\section{High Voltage EDM Cells}

\section{Dual Comagnetometer}



\begin{description}
\item{An introduction about the long term nEDM effort at TRIUMF, what
  the plan is, when it will start (roughly). I guess I can probably
  get this information from some proposals. I am not sure how much
  detail should go here.}
  
\item{How the EDM experiment is actually done, talk about different
  components of the system. Here is where I talk about the Ramsey
  cycle ...}
  
\item{nEDM measurement systematic effects: This is where I talk about
  the GPE and ... . Basically here is to kind of motivate that we need
  to have stable magnetic fields and we need lots of neutrons.}
  
\item{Introduction to the magnetic stability requirements at
  TRIUMF. What I mean is that there is 400 $\mu$T background field at
  TRIUMF. Hopefully we have a field map of the area soon(?).}
  
\item{From ouside in: Magnetically shielded room, what is the status
  of that, are we going to have it? when? How good is it going to be
  compared to the other ones worldwide? Why is it designed that way?
  What is the design? Drawings of it. General question: Some of these
  are about things that will happen in the future and I have not
  worked on them. Should they even go to my thesis? I feel I have to
  say a little about this since my thesis is nEDM related and it is
  part of it.}
  
\item{Passive shieldings: Again same questions as above, motivate for
  the next chapter}

  
\item{Say what will be discussed in the two coming chapters}
  
\item{what else?}
\end{description}
