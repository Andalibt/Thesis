\chapter{UCN Production and Detection\label{chap:UCNresult}}

The first UCN at TRIUMF (Nov.~2017) were produced by using the
vertical UCN source described in Sec.~\ref{vertical_source}. The
spallation neutrons were converted to UCN through phonon excitation in
the isotopically pure superfluid helium.  During data taking several
measurements were performed for better understandting of the vertical
UCN source and to help design the next generation UCN source. In this chapter,
the experiments are described and the result are shown.

Fig.~\ref{fig:volume_schematic} is a simple schematic of the UCN
production and detection volume.  Here volume $V_1$ is the production
volume before the valve where $N_1$ number of UCN is produced, $V_2$
is the secondary volume where $N_2$ number of UCN enters after opening
the valve , and $V_3$ is the detector volume with $N_3$ number of UCN.



\begin{figure}[h]
  \centering
  \includegraphics[width=0.9\textwidth]{volume_schematic.png}
  \caption{Schematic drawing of a UCN source. $V_1$ is the production
    volume with $N_1$ number of UCN, $V_2$ is the secondary volume
    where $N_2$ number of UCN exist and $V_3$ is the detector with
    $N_3$ number of UCN. }
  \label{fig:volume_schematic}
\end{figure}


When the beam is on and the valve is closed, the number of UCN in
$V_1$ goes up while the total number of UCN in $V_2$ and $V_3$ is
zero. This is described as 

\begin{equation}
  \label{eqn:dndt}
\frac{dN_1}{dt} = P - \frac{N_1}{\tau_1}  
\end{equation}
where $P$ is the UCN production rate in the source as describe in
Sec.~\ref{sec:UCN_production} and $\tau_1$ is the UCN storage lifetime
in the source.

After the beam is turned off, the valve is opened and the UCN travels
to the volume $V_2$ and eventually $V_3$. The valve is usually left
open for 2 to 3 minutes. The UCN trade between $V_1$, $V_2$ and $V_3$
is described by the differential Eqn.~\ref{eqn:alldndt}.

\begin{equation}
  \label{eqn:alldndt}
  \begin{aligned}
    \frac{dN_1}{dt} =&- \frac{N_1}{\tau_{c,1}} - \frac{N_1}{\tau_1} + \frac{N_2}{\tau_{c,2}}  \\
    \frac{dN_2}{dt} =& \frac{N_1}{\tau_{c,1}} - \frac{N_2}{\tau_{c,2}} - \frac{N_2}{\tau_2} - \frac{N_2}{\tau_{c,3}} \\
    \frac{dN_3}{dt} =& \frac{N_2}{\tau_{c,3}}.
  \end{aligned}
\end{equation}


In these equations, \large $\frac{dN_1}{dt}$ \normalsize shows the
change in the UCN counts over time in $V_1$,~\large $\frac{dN_2}{dt}$
\normalsize shows the change in the UCN counts in $V_2$ and \large
$\frac{dN_3}{dt}$ \normalsize is change in the UCN count in $V_3$
which is the detector all after the valve is opened.

The total number of UCN in $V_1$ depends on three things. The UCN that
get into $V_2$~\large($\frac{N_1}{\tau_{c,1}}$) \normalsize, the UCN that is lost
with the storage lifetime of $\tau_1$, and the UCN that bounce back from
$V_2$ to $V_1$~\large ($\frac{N_2}{\tau_{c,2}}$) \normalsize.

In $V_2$, some UCN cross from $V_1$ to $V_2$~ \large
($\frac{N_1}{\tau_{c,1}}$) \normalsize, some get lost with the
lifetime of $\tau_2$~\large ($\frac{N_2}{\tau_2}$) \normalsize, some
cross the gate valve to go back to $V_1$~ \large
($\frac{N_2}{\tau_{c,2}}$)\normalsize and some get to the detector
~\large ($\frac{N_2}{\tau_{c,3}}$) \normalsize. The rate of the UCN
detection $\frac{dN_3}{dt}$ is the number of UCN crossing from $V_2$.
The end of the measurement cycle is determined by the valve closing
time.

Solving these equation could give an estimate of how many UCN exist in
each volume. The process described above is refered to the UCN
production in {\it {batch mode}}. The UCN rate for 1~$\mu$A beam current
and 60~s irradiation time is shown in Fig.~\ref{fig:UCNRate}.


\begin{figure}[h]
  \centering
  \includegraphics[width=0.7\textwidth]{UCNRate.png}
  \caption{The figure shows the UCN rate at 60~s irradiation time and
    1~$\mu$A beam current. In this case, the UCN gate valve is opened
    immediately after the end of irradiation. At this time, the UCN
    rate reaches the peak of about 2000 UCN/s. The UCN rate decays
    down to zero. The Valve is left open for 120~s. }
  \label{fig:UCNRate}
\end{figure}



A 3D drawing of the experimental setup is shown in
Fig.~\ref{fig:Source_all}. In this case, $V_1$ is the UCN source
bottle and the horizontal section of the UCN guide before the UCN gate
valve and $V_2$ and $V_3$ are the volumes after the UCN valve and the
detector volume respectively.




Another possible mode of operation is when we leave the UCN valve open
while irradiating the target. This is called the {\it{ steady-state}}
mode where we have a constant stream of UCN to the main detector.


\begin{figure}[h!]
  \centering
  \includegraphics[width=1.1\textwidth]{Source_all.png}
  \caption{The UCN source and the guide geometry at TRIUMF }
  \label{fig:Source_all}
\end{figure}


The following sections are focused on the result of the UCN yield
optimization, the UCN storage lifetime measurements, the two detector
comparisons and the UCN guide transmission measurements.


\section {Data Quality Checks}
The main detector for most of the collected data is the
$^6\mathrm{Li}$ glass based scintillator detector which described in
Sec.~\ref{sec:Li6detector}. Before using the collected data to
extract the desired information, it is critical to make sure that the
detector was working as expected.

The PSD versus $Q_L$ distribution from a UCN data run is shown in
Fig.~\ref{fig:psd_vs_ql} for all the PMTs combined. This is the most useful
way of separating the signal and background.
% The graph is for run 541
\begin{figure}[h!]
  \centering
  \includegraphics[width=0.9\textwidth]{PSD_vs_QL.pdf}
  \caption{PSD versus $Q_L$ for all of the PMTs for a standard
    1~$\mu$A proton beam current and 60~s target irradiation storage
    lifetime measurement. }
  \label{fig:psd_vs_ql}
\end{figure}
Here the UCN events are mainly around the PSD value of 0.5 and $Q_L$
between 3000 and 12000. The events at PSD~$\sim 0$ represent the
$\gamma$-rays in the lightguides. To get the actual UCN counts, a PSD
cut of 0.3 and a $Q_L$ cut of 2000 were applied. This ensures that
only the UCN events are counted represented by the central oval-shaped
region. Out of all 9 channels, the centeral channel counts the most
number of UCN events while the corner channels receive the least as
expected. Fig.~\ref{fig:channelcounts}.

\begin{figure}[h!]
  \centering
  \includegraphics[width=0.9\textwidth]{channelcounts.pdf}
  \caption{Number of UCN events for each channel. The total number of
    UCN events decrease as we move toward the corner channels.  }
  \label{fig:channelcounts}
\end{figure}

% should add a bit more about the dead time and stuff

\section{UCN Count Measurement \label{UCNCounts}}

The total number of produced UCN in the vertical source, $N$, at a
certain time t$_i$ when the UCN valve is closed is the integration of
eqn.~\ref{eqn:dndt}
\begin{equation}
  \label{eq:totalUCN}
  N = P \tau_1\left[ 1- \exp \left(\frac{t_i }{ \tau_1}\right) \right]
\end{equation}
where the UCN storage lifetime $\tau_1$ is given by

\begin{equation}
  \label{eqn:tau1}
  \frac{1}{\tau_1} = \frac{ f_\mathrm{He,1}}{\tau_\mathrm{He}} + \frac{1}{\tau_\mathrm{wall,1}}.
\end{equation}

The storage lifetime consists of two terms: the upscattering rate in
the superfluid helium and the loss rate in the UCN guide walls. The
volume in which these UCN are produced consists of the UCN bottle as
well as the horizontal guide section before the gate valve~(See
Fig.~\ref{fig:Source_all}).This volume is not fully filled with the
superfluid helium. As a result, $ f_\mathrm{He,1}$ is the probablity
of UCN being in the superfluid helium and \large
$\frac{ f_\mathrm{He,1}}{\tau_\mathrm{He}}$ \normalsize is the
upscattering rate in the superfluid helium which is a function of the
superfluid helium temperature~(See Sec.~\ref{sec:upscattering}. The
loss rate in the guide walls is $\frac{1}{\tau_\mathrm{wall,1}}$.

After the valve is opened, the total UCN lifetime is

\begin{equation}
  \label{eqn:tau2}
  \frac{1}{\tau_2} = \frac{ f_\mathrm{He,2}}{\tau_\mathrm{He}} + \frac{1}{\tau_\mathrm{wall,2}}+\frac{1}{\tau_d}
\end{equation}
where $\tau_d^{-1}$ is the loss rate in the detector,
${ f_\mathrm{He,2}}$ is the probability of UCN being in the superfluid
  helium and ${\tau_\mathrm{wall,2}}^{-1}$ is the UCN guide loss rate
  in this case where the valve is open and the target irradiation is
  stopped. Fig.~\ref{fig:UCNRate_with_lines} shows three measurement
  cycles at 1~$\mu$A beam current and 60 second irradiation time with
  zero second delay time for opening the UCN valve. The dashed lines
  indicate the start of the irradiation for a cycle, the dotted lines
  show the end of irradiation which in this case is open to the UCN
  valve open time. The solid lines shows the valve close time.


\begin{figure}[h!]
  \centering
  \includegraphics[width=1.1\textwidth]{UCNRate_with_lines.png}
  \caption{Three measurement cycles for 1~$\mu$A beam current, 60~s
    irradiation time and 0 seconds cycle delay time. The dashed line
    shows the start of the target irradiation, the dotted line shown
    the end of the irradiation and the valve open time for each cycle
    and the solid line shows the end of a cycle which is the valve
    close time.  }
  \label{fig:UCNRate_with_lines}
\end{figure}

The total UCN counts are given by the integration of all the UCN
events for the duration of the valve open time. However, this method
of counting includes the measured background UCN as well. To subtract
the background counts from the actual UCN counts, the UCN background
rate is calculated during the valve closed time for cycle ($i$-1) and
the irradiation start time for cycle $i$. This rate is then multiplied
by the valve open duration time and it gives an estimate of the total
background UCN counts which is typically less than 5~UCN/s. The
subtraction of the latter from the total UCN counts gives the actual
UCN counts produced by the isopure helium converter.

At low and moderate UCN counts, the statistical uncertainty is readily
available by taking the square root of the number of measured events,
as follows conveniently from Poisson
statistics~\cite{pomme2015uncertainty}.


Fig.~\ref{fig:counts_vs_beam} shows the total UCN counts versus the
applied proton beam current in $\mu$A at 60~s irradiation time. At
lower beam currents, the total UCN counts increase linearly with the
proton beam current. The dashed line shows the extrapolation to higher
beam current in an ideal case. However, at higher beam currents the
total UCN counts deacreses due to the increase in the heat load on the
isotopically pure superfluid helium and its temperature. The
upscattering rate in the superfluid helium is proportional to its
temperature as $T^7$~(See Sec.~\ref{sec:upscattering}).



\begin{figure}[h!]
  \centering
  \includegraphics[width=0.8\textwidth]{UCNCounts_vs_Beam.pdf}
  \caption{The total UCN counts versus the applied proton beam
    current. The labelas show the average superfluid helium
    temperature for that measurement. The dashed line is fit to the
    UCN counts at low beam current. }
  \label{fig:counts_vs_beam}
\end{figure}



The labels in the graph show the average temperature during the
measurement cycle. Four temperature sensors were used to measure the
superfluid helium temperature: TS11, TS12, TS14 and TS16. The location
of these sensors are shown in Fig.~\ref{fig:TSs}. Temperature sensor
TS11 is located at the UCN heat exchanger bottom while the temperature
sensor TS14 is located at the UCN heat exchanger top. The temperature
sensor TS12 is located at the UCN double tube bottom while the
temperature sensor TS16 is located at the UCN double tube top. At low
temperature around 0.8~K these temperature sensors show a maximum of
0.1~K discrepancy with TS16 showing the highest value and TS12 showing
the lowest value.



\begin{figure}[h!]
  \centering
  \includegraphics[width=0.8\textwidth]{TSs.png}
  \caption{Screenshot of the Epics temperature monitoring screen. TS11
    is located at the UCN head exchanger bottom, TS12 is located at
    the UCN double tube bottom, TS14 is located at the heat exchanger
    double tube top and TS16 is located at the UCN double tube
    top. For further information about the source schematic see
    Sec.~\ref{sec:vertical_source} }
  \label{fig:TSs}
\end{figure}




The total UCN count is optimized by irradiating the target with
different proton beam currents and different irradiation times. The
result is shown in Fig.~\ref{fig:counts_vs_irrad}. At higher beam
currents, the saturation time constant decreases due to the higher
heat load and faster temperature increase in the superfluid helium. At
higher beam currents and longer irradiation times the total measured
UCN counts are below the exponential extrapolation due to the higher
temperature and higher upscattering rate in the superfluid helium.

\begin{figure}[h!]
  \centering
  \includegraphics[width=0.8\textwidth]{UCNCounts_vs_irradTime.pdf}
  \caption{Number of UCNs extracted from the source after irradiating
    the target for different times with different beam currents. The
    dashed lines extrapolate the data for irradi- ation times below
    60~s using exponential saturation curves.  The labels show the
    saturation time constant for each beam current. }
  \label{fig:counts_vs_irrad}
\end{figure}


The result shown so far achieved in the batch mode of operation. In
addition to such measurements, the UCN rate at higher beam currents in
the steady-state mode were measured. In these measurements, the valve
was left open and the target was irradiated for 10~min. A typicall UCN
rate graph for 3~$\mu$A beam current and 10 minute irradiation time is
shown in Fig.~\ref{fig:UCNRate_steadystate}. The maximum UCN rate is
achievable at the start of target irradiation. As the irradiation
continues, the heat load on the cryostat increases the temperature and
the upscattering rate in the superfluid helium. As a result, the UCN
rate decreases. The change in the temperature is shown in
Fig.~\ref{fig:UCNRate_temp}.


\begin{figure}[h!]
  \centering
  \includegraphics[width=0.9\textwidth]{654_UCNRate.png}
  \caption{The UCN rate at 3~$\mu$A beam current at 10~min irradiation
    time at the steady-state mode of operation. The UCN valve is left
    open throughout the measurement cycle. Quickly after the start of
    the target irradiation the UCN rate in the detector goes up. The
    target irradiation creates heatload on the cryostat and superfluid
    helium which gives rise to a slow temperature increase in the
    source. As a result, the UCN rate goes down due to the higher
    upscattering rate.  }
  \label{fig:UCNRate_steadystate}
\end{figure}

\begin{figure}[h!]
  \centering
  \includegraphics[width=0.7\textwidth]{UCNRate_temp.png}
  \caption{ The temperature of the superfluid helium~(TS12) for the
    steady state mode of operation at 3~$\mu$A beam current and 10~min
    target irradiation. After the irradiation stops, the temperature
    starts to go down. }
  \label{fig:UCNRate_temp}
\end{figure}

The steady-state UCN rate measurements were conducted at different
proton beam currents leading to different temperature changes for all
temperature sensors. The result of all those measurements are shown in
Fig.~\ref{fig:rate_vs_temp}. In this figure, the vertical axis is the
measured UCN rate normalized to the proton beam current and the
horizontal axis shows the temperature of the isotopically pure
superfluid helium for all the temperature sensors.


\begin{figure}[h!]
  \centering
  \includegraphics[width=0.7\textwidth]{rate_vs_temp.pdf}
  \caption{Histogram of measured UCN rates and temperatures from all
    four temperature sensors while the target is continuously
    irradiated with the UCN valve open. The four solid lines are fits
    of equation~\ref{eq:steadystaterate} to the data of each individual tem- perature
    sensor.}
  \label{fig:rate_vs_temp}
\end{figure}


The detected rate of the detected UCN is given by

\begin{equation}
  \label{eqn:rate}
  R = \frac{P \tau_3}{\tau_d} = \frac{P \tau_d^{-1}}{\tau_\mathrm{wall,2}^{-1} + \tau_d^{-1} + f_\mathrm{He,3}\tau_\mathrm{He}^{-1}}
\end{equation}
where $\tau_d^{-1}$ is the loss rate in the detector,
${\tau_\mathrm{wall,2}}^{-1}$ is the UCN guide wall loss and
  $\tau_\mathrm{He}^{-1}$ is the loss rate in the superfluid helium.

\begin{table}[h!]
  \centering
  \begin{tabular}{|c|c|c|c|}
    \hline
      Temp. sensor & $a$ & $b$ & $c$ (\si{\per\second}) \\
      \hline
      TS11 & $7.55 \pm 0.03$ & $0.0697 \pm 0.0009$ & $1535 \pm 1$ \\
      \hline
      TS12 & $6.48 \pm 0.03$ & $0.1293 \pm 0.0016$ & $1606 \pm 2$ \\
      \hline
      TS14 & $7.34 \pm 0.03$ & $0.0832 \pm 0.0011$ & $1555 \pm 2$ \\
      \hline
      TS16 & $6.67 \pm 0.04$ & $0.1215 \pm 0.0019$ & $1685 \pm 3$ \\
      \hline
    \end{tabular}
  \caption{Parameters determined by fitting equation
    \ref{eq:steadystaterate} to the measured rates shown in
    fig. \ref{fig:UCNRate_with_fit} for each individual temperature
    sensor.}
  \label{tab:steadystateparams}
\end{table}


Assuming
$\tau_\mathrm{He}^{-1} = B \left( T \right)^a$
the Eqn.~\ref{fig:rate} could be written as

\begin{equation}
R(T) = \frac{c}{1 + b \left( \frac{T}{\SI{1}{\kelvin}} \right)^{a}}
\label{eq:steadystaterate}
\end{equation}
which can be used to fit the data shown in
Fig.~\ref{fig:rate_vs_temp}.  Since the four temperature sensors in
the superfluid helium deviate by up to \SI{0.1}{\kelvin}, the rate for
each temperature sensor is fitted individually (See
Fig.~\ref{fig:UCNRate_with_fit} and table~\ref{tab:steadystateparams}),
and considered their differences systematic uncertainties. The
exponent $a$ can be directly determined this way, giving
\begin{equation}
a = 7.02 \pm 0.02_\mathrm{stat.} \pm 0.53_\mathrm{syst.},
\label{eq:a}
\end{equation}
which is in good agreement with the theoretical prediction of $a = 7$
(see Sec.\ref{sec:upscattering}). Here the statistical error comes
from the fit and the systematic error comes from the temperature
difference from the sensors and their propagated error.


The other parameters are
\begin{align}
\label{eq:b}
  b =& \frac{f_\mathrm{He,3} B}{\tau_\mathrm{wall,2}^{-1} + \tau_d^{-1}} = 0.0995 \pm 0.0007_\mathrm{stat.} \pm 0.0298_\mathrm{syst.} \\
  c =& \frac{P \tau_d^{-1}}{\tau_\mathrm{wall,2}^{-1} + \tau_d^{-1}} = (1610 \pm 1_\mathrm{stat.} \pm 75_\mathrm{syst.}) \, \si{\per\second}
\end{align}



% total UCN counts at different isopure temperatures???


The total UCN counts for 1~$\mu$A beam current and 60~s irradiation
time over the course of the experimental run is shown in
Fig.~\ref{fig:UCNCounts_time}. 
\begin{figure}[h]
  \centering
  \includegraphics[width=0.8\textwidth]{UCNCounts_vs_Time.png}
  \caption{The total UCN counts extracted from the source for 1~$\mu$A
    beam current and 60~s irradiation time at different days during
    the experimental run. }
  \label{fig:UCNCounts_time}
\end{figure}
The source volume is connected to a long UCN guide sealed with an
O-ring. It is expected that the rest gas to contaminate the source
every time the UCN valve is opened. This caused a reduction in the UCN
yield over the course of the measurement as shown in
Fig.~\ref{fig:UCNCounts_time}. In addition, the changes in the UCN
guide geometry in the latter half of the run pottentially affected
this drop.

\section{Storage Lifetime: Measurements And
  Simulations\label{storagelifetime}}

The total number of detected UCN strongly depends on the storage
lifetime of the source $\tau_1$~(See Eqn.~\ref{eqn:tau1}) which
indicates the performance of the UCN source. The storage lifetime of
UCN is determined by measuring the detected UCN at different valve
open delay times right after the irradiation stops. The typical chosen
values are 0~s, 5~s, 10~s, 20~s, 30~s, 60~s, 80~s, 120~s and
170~s. The exponential decay constant in the fit function to the total
UCN counts for different valve open delay times is the total storage
lifetime in the source.

Fig.~\ref{fig:storage_exmaple} shows the total UCN counts versus the
valve open delay time for 1~$\mu$A proton beam current and 60~s irradiation
time. The longer delay times give rise to lower UCN counts due to the
loss mechanisms. The one exponential fit function
\begin{equation}
\text{UCN counts} = A e^{-t/\tau_1}
\end{equation}
determines the storage lifetime $\tau_1$. At 170~s valve open delay
time, the total UCN counts are not consistent with what the fit
function predicts. However, the result of the fit is not driven by
this inconsistency as it has a negligible effect on the extrected
storage lifetime.

\begin{figure}[h]
  \centering
  \includegraphics[width=0.9\textwidth]{17002_StorageLifetime.pdf}
  \caption{The total UCN counts at different valve open delay times
    for 1~$\mu$A beam current and 60~s irradiation time. The red line
    is the one exponential fit. }
  \label{fig:storage_example}
\end{figure}


The storage lifetime of the UCN source is measured at different proton
beam currents and different irradiation times for better
optimization. The result of those measurements is shown in
Fig.~\ref{fig:storage_beam_irrad}. The UCN production rate increases at higher
proton beam current. However, this creates a higher heat load on the
UCN source which leads to higher upscattering rate. The higher
proton beam currents and longer irradiation times give rise to lower
storage lifetimes.

\begin{figure}[h]
  \centering
  \includegraphics[width=0.9\textwidth]{StorageLifetime_17009_and_17009A.pdf}
  \caption{Storage lifetime in the source at different irradiation
    times and proton beam currents. Different markers refer to
    different target irradiation times. At longer irradiation times
    and higher beam currents the storge lifetime decreases due to the
    increase heat load in the cryostat and increase in the superfluid
    helium temperature. }
  \label{fig:storage_beam_irrad}
\end{figure}

\subsubsection{PENTrack Simulations}

For better understanding of the loss mechanisms, the experiment was
also simulated in PENTrack~\cite{schreyer2017pentrack}. PENTrack is a
particle tracking simulation software which simulates the trajectories
of UCN and their decay products (e.g. Protons and Electrons) and their
spin precession in complex geometries in Electric and Magnetic fields
by solving the relativistic equation of motion. As discussed in
Chapter~\ref{chap:intro}, UCN interacts with all four fundamental forces. To
describe the interaction of UCN with matter, a complex optical
potential is used to describe matters~\cite{ucnbook}:
\begin{equation}
  \label{eqn:fermipotential}
  U = V - iW
\end{equation}
where the real part, $V$, depends on the number densities and bound
coherent scattering lenghtes of each nucleus species. The imaginary
part, $W$, depends on the loss cross-section for a given velocity.
Upon the incidence of the UCN on a surface, it can be scatterd either
specularly or diffusely. PENTrack uses two models to calculate the
scattering distribution of the UCN impinging on the material surface:
Lambert model or the Microroughness~\cite{Steyerl1972}.
  
Experimental geometries imported in PENTrack are the StL files made
through CAD models. For these simulations, the exact model of the
vertical UCN source was used including the burst disk, the actual
shape of the UCN valve in the open and close state, pinhole foil and
the detector. 

The abosorption in the foil is set according to the measurements
in~\cite{atchison2009transmission}. The main decetor is modeled with
its two scintillator layes~\cite{jamieson2017characterization} and
their corresponding Fermi potentials and absorption cross section, as
stated in~\cite{Ban2016};

In the simulations, it is assumed that the spectrum of produced UCN is
proportional to $\sqrt{E}$. The wall loss parameters were tuned to
give a storge lifetime of $\tau_1 = 34.9 \pm .8$~s with an
upscattering lifetime in the superfluid of
$\tau_{\mathrm{He}}^{-1} = (390~\mathrm{s})^{-1} =
0.008~\mathrm{s}^{-1}\cdot 0.85^{7}$, resulting in material parameters
shown in Table~\ref{tab:materials}.


\begin{table}
  \centering
\begin{tabular}{|c|c|c|}
  \hline
Material & Fermi pot. (neV) & Diffusivity \\
\hline
He-II ($\tau_\mathrm{He} = \SI{390}{\second}$) & $18.8 - 8.44\cdot10^{-10} i$ & 0.16 \\
Prod. volume (NiP) & $213 - 0.100 i$ & 0.05 \\
Guides (stainl. steel) & $183 - 0.120 i$ & 0.03 \\
%Pinhole (copper) & $171 - 0.0726 i$ & 0.20 \\
Foil (aluminium) & $54.1 - 0.00281 i$ & 0.20 \\
GS30 scintillator & $83.1 - 0.000123 i$ & 0.16 \\
  GS20 scintillator & $103 - 1.24 i$ & 0.16 \\
  \hline
\end{tabular}
\caption{Material parameters used in the PENTrack simulation.~\cite{atchison2009transmission,Ban2016,sears1992neutron}}
\label{tab:materials}
\end{table}

The simulation and the measurement data are both fitted with the fucntion

\begin{equation}
R(t) = R_0 \left[ 1 - \exp \left( -\frac{t - \Delta t}{\tau_\mathrm{rise}} \right) \right] \exp \left( -\frac{t - \Delta t}{\tau_2} \right) + R_B.
\end{equation}

after opening the valve at $t=0$. In this equation, $\Delta t$ is the
delay time between opening the valve and detecting the first UCN and
it is 2 to 3~s. The parameter $ R_B$ is the background UCN rate in the
experimental data and is zero in simulations.  The Lambert model is
used to tune the probablity of UCN being diffusely reflected on the
guide walls to match the rise time $\tau_{mathrm{rise}}$ and fall
time~$\tau_2$ of the UCN rate in the storage lifetime
measurements~(See Fig.~\ref{fig:falltime} and~\ref{fig:risetime}).
The experimental data is best described by the diffusivity of 3~\% and 5~\%. The


\begin{figure}[h!]
  \centering \includegraphics[width=0.9\textwidth]{falltime.pdf}
  \caption{Comparison of fall time $\tau_2$ in the experimental data
    and the simulations with different diffuse-reflection
    probabilities. The boxes indicate the second and third quartile of
    the experimental data.}
\label{fig:falltime}
\end{figure}

\begin{figure}[h!]
\centering
\includegraphics[width=0.9\textwidth]{risetime.pdf}
\caption{Comparison of rise time $\tau_{\mathrm{rise}}$ in
  experimental data and simulations with different diffuse-reflection
  probabilities. The boxes indicate the second and third quartile of
  the experimental data.}
\label{fig:risetime}
\end{figure}

The simulations are also used to determine the parameter
$f_{\mathrm{He}}$ or the fraction of time that UCN spends in the
superfluid helium in the detectable range of 120~neV to 200~neV. For
the steady-state measurements, this fraction turned out to be almost
constant over a range of upscattering lifetime in the superfluid
giving

\begin{equation}
f_\mathrm{He,3} = 0.464 \pm 0.001_\mathrm{stat.} \pm 0.003_\mathrm{syst.},
\end{equation}
where the systematic uncertainty is the variation in simulations with
$\tau_{\mathrm{He}}$ from 3.05~s to 390~s. 
The Eqn.~\ref{eqn:tau2} could be rewritten as
\begin{equation}
  \label{eqn:tau2rewritten}
\tau_d^{-1} + \tau_\mathrm{wall,2}^{-1} = \tau_2^{-1}(T_0) - f_\mathrm{He,2} B \left( T_0 \right)^a
\end{equation}
where the value for $\tau_2$ is
\begin{equation}
\tau_2^{-1}(T_0) = (19.3 \pm 0.8_\mathrm{stat.})\,~\si{\second}
\end{equation}
comes from fall time in the storage lifetime measurements at the
temperature $T_0$ which is
\begin{equation}
T_0 = 0.92 \pm 0.005_\mathrm{stat.} \pm 0.05_\mathrm{syst.}.
\end{equation}
Replacing Eqn.~\ref{eqn:tau2rewritten} in Eqn.~\ref{eq:b}, $B$ can be written as
\begin{equation}
B = \frac{b \tau_2^{-1}(T_0)}{f_\mathrm{He,3} + f_\mathrm{He,2} b \left( \frac{T_0}{\SI{1}{\kelvin}} \right)^a}.
\end{equation}
Here all the parameters are known except for $ f_\mathrm{He,2}$ which
does not significantly affect the result, and hence, it is assumed to
lie between 0 and 1. As a result
\begin{equation}
B = (10.4 \pm 0.4_\mathrm{stat.} \pm 4.1_\mathrm{syst.}) \cdot 10^{-3} \, \si{\per\second}.
\end{equation}
which is consistent with the result in~\cite{Leung2016}.


%add UCN production simulation result here.


\section{Heater Tests of The Source}
The cooling process of the spallation neutrons create a heat input on
the cryostat including the superfluid helium which then must be
removed to keep the temperature constant. The temperature gradient in
the superfluid helium is described by its heat conductivity. The
temperature dependence of the heat conductivity in the superfluid
helium is describe by theorical models to from the lambda point 2.17~K
down to around 1.4~K which is above the temperature for the UCN
production~($<~1$~K).  Because of the difficulty to reach such low
temperatures, the mechanism of heat transfer in that region is not
fully understood.  To check the validity of theoretical models, the
extrapolation to lower temperatures is compared to acquired data.


In order to create excess heat load on the superfluid helium, there
are heater tapes wrapped around the superfluid helium bottle. These
heaters can create a temperature gradient between the heat exchanger
and the superfluid helium bottle. This temperature gradient can be
measured using the temperature sensors shown in Fig.~\ref{fig:TSs}.

The heater test procedure is the following. The heater around the
superfluid helium bottle is turned on when the temperature of the
superfluid is stable. This is called the base temperature. The applied
heat load could easily be calculated since the applied current and
voltage are known. After the heater is turned on, the temperature of
the superfluid helium starts to increase. This causes the flow rate in
the $^3$He pot to increase. After some time, the temperature of the
superfluid starts to settle and reach a new equilibrium. This
temperature is referred to as the saturation temperature. At this
point, the heater could be turned off which causes the superfluid
temperature and the flow rate to go back to base conditions.

In 2017 two sets of heat tests were performed on the vertical UCN
source: April heat test and November heat test. In April the base
temperature of the superfluid helium was slightly higher than in
November. In addition, there was no beam during the April cooling and
it was purely a cryostat cooling test. Table.~\ref{tab:aprilheattest}
shows the heat load and the temperatures of the April heat test.


\begin{sidewaystable}
  \centering
  \begin{tabular}{|c|c|c|c|c|c|c|c|c|c|c|}
    \hline
    Heater Power & T$_{\mathrm{base, TS10}}$ &  T$_{\mathrm{sat., TS10}}$ & T$_{\mathrm{base, TS11}}$ &  T$_{\mathrm{sat., TS11}}$ & T$_{\mathrm{base, TS12}}$ &  T$_{\mathrm{sat., TS12}}$ & T$_{\mathrm{base, TS14}}$ &  T$_{\mathrm{sat., TS14}}$ & T$_{\mathrm{base, TS16}}$ &  T$_{\mathrm{sat., TS16}}$ \\
    (mW) & (K) & (K) & (K) & (K) & (K) & (K) & (K) & (K) & (K) & (K) \\
    \hline
   2.5 & 0.717 & 0.718 & 0.93 & 0.931 & 0.926 & 0.9271 & 0.93 & 0.931 & 1.012 & 1.013 \\
    \hline
    12.5 & 0.717 & 0.7185 & 0.93 &  0.9315 & 0.924 & 0.929 & 0.93 & 0.9315 & 1.011 & 1.015 \\
    \hline
    25 & 0.719 & 0.723 & 0.928 & 0.931 & 0.919 & 0.929 & 0.928 & 0.931 & 1.008 & 1.015 \\
    \hline
  \end{tabular}
  \caption{}
  \label{tab:aprilheattest}

  \end{sidewaystable}



\section{Detector Comparison}
Using the rotary valve

\section{Background Measurements}
With Ni foil

\section{UCN guide Transmission Measurements}

\section{Result And Conclusion}

%\begin{description}
%\item{I think this belongs to this chapter: UCN production by
%  multiphonon excitation in superfluid helium (I can use my candidacy
%  report for this part as a start)}
  
%\item{Some information about the detector}
  
%\item{UCN data goes here}
  
%\item{what else?}
%\end{description}
