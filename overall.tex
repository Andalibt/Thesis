\chapter{Conclusion\label{chap:overall}}

%Finding a non-zero neutron EDM confirms beyond the standard model
%theories that provide extra sources of CP violation.

The work presented in this thesis is part of the studies towards the
future nEDM experiment at TRIUMF. The existence of a non-zero nEDM
confirms beyond the standard model theories that provide extra sources
of CP violation. These extra sources of CP violation are essential to
create the observed Baryon asymmetry in the universe.

The focus of the research in this thesis is on the two aspects of the
nEDM measurement: The magnetic field stability and UCN production and
storage.

To measure the nEDM, an ensemble of polarized neutrons are placed in
the presence of aligned electric and magnetic fields. The Larmor
precession frequency of the UCN is measured once when the electric and
magnetic fields are parallel and once when they are anti-parallel. The
frequency shift between these two geometries is proportional to the
nEDM. In this process, the existence of a very stable and homogeneous
magnetic environment is essential and the applied DC magnetic field
should be held constant. To achieve the magnetic requirements several
layers of magnetic shielding is employed including active and passive
shielding. Interal coils are placed inside the passive shielding to
create the DC magnetic field. These are called the shield-coupled
coils. In this case, a change in the properties of the passive
shields, such as magnetic permeability $\mu$, would affect the
magnetic field measured internally. One of the factors that can cause
such changes in $\mu$ is the changes in the environmental temperature.

In Chapter~\ref{chap:muofT} the result of the studies of the changes
of $\mu$ with temperature were presented and discussed. Two methods
were pursued to study the correlation of the changes in the measured
internal magnetic field with respect to the changes in temperature. In
method one, a witness cylinder with a length of 15.2~cm and a diameter
of 5.2~cm was put inside a coil system that produced a low-frequency
magnetic field. The axial shielding factor was then measured with a
magnetometer as a function of the temperature. The temperature was
measured via the sensors attached on the witness cylinder. These
measurements were repeated with two different coils to study the
systematic effects. In the second technique, which is more common, the
witness cylinder was used as a core of a transformer. A primary and a
secondary coil were wound on the witness cylinder. Here the slopes of
the minor $B-H$ loops as a function of temperature were measured.

The overall result of the $B(T)$ measurements are presented in
Tables~\ref{tab:axial} and~\ref{tab:transformer}. These measurements
were conducted in AC fields with frequencies around 1~Hz as opposed to
the DC fields in the actual nEDM experiments.  To related these
measurements to $\mu(T)$, finite element simulations were performed to
find the shielding factor of the witness cylinders as a function of
$\mu$. Combining the measurements and the simulations, in the first
method it was found that
0.6\%/K~$<\frac{1}{\mu}\frac{d\mu}{dT}<2.7\%$/K with $H_m$-amplitude
of 0.004~A/m at 1~Hz. In the second method, it was found that
0.0\%/K~$<\frac{1}{\mu}\frac{d\mu}{dT}<2.2\%$/K with a typical
$H_m$-amplitude of 0.1~A/m at 1~Hz.

Considering the overall value of
0.0\%/K~$<\frac{1}{\mu}\frac{d\mu}{dT}<2.7$\%/K and the generic EDM
experiment sensitivity of $\frac{\mu}{B_0}\frac{dB_0}{d\mu}=0.01$, the
temperature dependence of the magnetic field in a typical nEDM
experiment would be $\frac{dB_0}{dT}=0-270$~pT/K. This means, to
achieve the magnetic stability goal of 1~pT in the interal field, the
temperature of the innermost magnetic shield in the nEDM experiment
should be controlled to $<0.004$~K level which puts a challenging
constraint on the future nEDM experiment design.

The second half of this thesis was focused on the current UCN facility
at TRIUMF. In 2016 the prototype vertical UCN source, previously built
and tested in Japan, was shipped to TRIUMF. The unique feature of this
facility is producing UCN by combining spallation neutrons with a
helium converter. In Chapter~\ref{chap:UCNattriumf} the faciliy was
described.

In November 2017 the first UCN were produced with the prototype
vertical source. Those experiments and the result of data analysis
were presented in Chapter~\ref{chap:UCNresult}. Such experiments are
essential for a better understanding of the cryostat and for the
design of the next generation UCN source. Around 40000 UCN were
produced at the standard measurement of 1~$\mu$A proton beam current
while irradiating the target for 60~s. The maximum number of produced
UCN were 325000 at 10~$\mu$A proton beam current. In the three weeks
of experiments, the measured storage lifetime of UCN dropped from 37~s
to 27~s. This was due to the contamination in the source by opening
the UCN valve. The UCN yield also dropped by about 40\%. Other than
the source contamination, different experimental configuration in the
second half of the experimental run period caused this drop.

%The TUCAN team is focused on the R\&D for the next generation UCN
%source.







