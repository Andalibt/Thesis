\chapter{Introduction\label{chap:intro}}
\renewcommand{\thepage}{\arabic{page}}% Arabic numerals for page
                                      % counter
\setcounter{page}{1}% Start page number with 1


The work presented in this thesis is focused on the important factors
for successfully measuring the neutron electric dipole moment~(nEDM)
at TRIUMF. Those two are having a very stable magnetic field
environment, as well as high ultracold neutron~(UCN) statistics. The
TRIUMF Advanced Ultracold Neutron source~(TUCAN) collaboration's goal
is to measure the nEDM to the $10^{-27}$~e$\cdot$cm sensitivity level.


This chapter provides some information on the interest in the nEDM
measurement from a physics perspective. Finding a nonzero nEDM would
answer questions regarding the matter-antimatter or baryon asymmetry
of the universe. The nEDM measurement method and a brief overview of
the UCN and nEDM facilities worldwide are also presented here.
Chapter~\ref{chap:nedm} gives a description of the future nEDM
measurement at TRIUMF and its experimental setup
components. Chapter~\ref{chap:muofT} is focused on the work towards
the temperature dependence of magnetic permeability $\mu$, which helps
to define the temperature stability requirements for the nEDM
measurement. Chapter~\ref{chap:UCNattriumf} presents the current UCN
facility at TRIUMF, where the first UCN were produced with the
vertical UCN source. Chapter~\ref{chap:UCNresult} presents the result
of those measurements with UCN. The final remarks and notes are
available in Chapter~\ref{chap:overall}.



\section{History of Fundamental Symmetries }

Studies of discrete symmetries have revealed the interactions of the
elementary particles and helped develop the underlying theories.
There are three discrete symmetries in physics: Charge
conjugation~($C$), Parity~($P$) and Time-reversal~($T$). $C$-symmetry
simply decribes physical laws under a particle-antiparticle
transformation. Parity transformation, is simply the inversion of
spatial coordinates, and Time-reversal transformation is changing the
direction of time.  Tests of Charge $C$, $P$ and $T$ symmetries
established the structure of the Standard
Model~(SM)~\cite{pospelov2005electric}.

In 1956, fall of discrete symmetries started with the famous
$\theta-\tau$ paradox in the K-mesons decay. The paradox was that two
particles previously known as $\theta^+$ and $\tau^+$, which had the
same mass and lifetime, decayed into products with different parities
\begin{equation}
  \begin{split}
    \theta^+ &\rightarrow \pi^+ + \pi^0 \\
    \tau^+ &\rightarrow \pi^+ + \pi^+ + \pi^-~.
  \end{split}
\end{equation}
At first, it was assumed that the initial states should also have
different parities, but precise measurements revealed that this is not
the case. Yang and Lee suggested that the paradox is originated from a
$P$ violation in the weak
interactions~\cite{lee1957parity}. Immediately after, an experimental
search was suggested by Ramsey for Parity violation in the $\beta$
decay of Co-60. Within a few months, $P$ violation was demonstrated by
three different experiments
~\cite{PhysRev.105.1413,PhysRev.105.1415,friedman1957nuclear}. After
the observation of $P$ violation, Landau showed that electric dipole
moments~(EDMs) are forbidden by $T$
symmetry~\cite{landau1957conservation}, and then, it was suggested
that $T$ symmetry should also be checked
experimentally~\cite{PhysRev.106.517}.
%In 1964 it was discovered that the $C$ and $P$ symmetries are broken in
%the $K$-meson decay~\cite{christenson1965regeneration}. 

One of the most fundamental symmetries in physics is
$CPT$~(Charge-Parity-Time) symmetry. The sequential operation of $C$,
$P$ and $T$ leave the system unchanged in quantum field theories. To
date, there is no experimental evindence for $CPT$ symmetry
breaking. Because of the $CPT$ invariance, breakdown of $CP$ symmetry
should be accompanied by violation of Time-reversal symmetry.

A finite EDM provides a good probe for the new sources of $CP$
violation beyond the Standard Model. EDMs caused by $CP$ violation in
the Standard Model are negligible. But most extensions of the Standard
Model~(such as supersymmetry) naturally produce EDMs that are
comparable to or larger than the present experimental limits.
% ~\cite{romalis2001new}.

The search for EDMs can be traced back to 1950, when Purcell and
Ramsey tested the possibility of finding EDMs for particles and
nuclei~\cite{PhysRev.78.807}. Smith, Purcell and Ramsey started an
experiment to search for neutron EDM $d_n$, and they achieved the
upper limit of $d_n < 5 \times 10^{-20}$~e $\cdot$
cm~\cite{smith1957experimental}.  Over the years, the upper limit on
the neutron EDM has been improved by many orders of
magnitude. Measurement of particle EDMs provide some of the tightest
constraints on the extensions to the Standard Model to probe $CP$
violation. The most recent upper limit on the neutron EDM is found to
be $\vert d_n\vert < 3.0 \times 10^{-26} $~e$\cdot$
cm~\cite{pendlebury2015revised}.



\section{Baryon Asymmetry of the Universe}
The neutron EDM provides a highly sensitive diagnostic for $CP$
violation, which is an important element for the observed
baryon asymmetry in the universe. The dominance of matter over
antimatter in the universe can be characterized by~\cite{Cline}
\begin{equation}
\eta = \frac{n_b-\bar{n_b}}{n_{\gamma}} \simeq 6 \times 10^{-10}
\end{equation}
where $n_b$ is the number of baryons, $\bar{n_b}$ is the number of
anti-baryons, and $n_{\gamma}$ is the number of photons in the cosmic
microwave backgorund.

It is possible to assume that, maybe, the universe is baryon symmetric
in a very large scale, and it is split into regions that are made of
only baryons or anti-baryons. If that was the case, an excess of gamma
rays in between these separated regions was expected to be observed
due to annihilation. But, even in the least dense regions of the
space, there are hydrogen gas clouds.

\subsubsection{Sakharov Criteria}
There are three key ingredients needed in a theory to create baryon
asymmetry known as Sakharov criteria~\cite{Sakharov:1967dj}:
\begin{center}
\begin{description}
\item[$\bullet$]Baryon number violation
\item[$\bullet$] $C$ and $CP$ violation
\item[$\bullet$] Departure from the thermal equilibrium.
\end{description}
\end{center}

The first condition is obvious because it simply means, in a reaction,
if the net baryon number is zero, there would be no baryon
asymmetry. In the reactions that violate baryon number, if there is no
$C$ and $CP$ violations, the net baryon number would be zero, and this
is because, the reactions that create excessive baryons will be
counter-balanced by the reactions that create excessive
anti-baryons~\cite{theearlyuniverse}. The third condition is essential
for a net nonzero baryon asymmetry over time, since the equilibrium
average of $B$ vanishes. Sakharov suggested that baryogenesis took
place immediately after the big bang, at a temperature not far below
the Planck scale of $10^{19}$~GeV, when the universe was expanding so
rapidly that many processes were out of thermal
equilibrium~\cite{cohen1993progress}.

\section{CP violation Beyond the Standard Model}
A testable scenario to create the baryon asymmetry in the universe is
the Electroweak Baryogenesis~(EWBG)~\cite{Morrissey2012}. The initial
condition for this process is a hot radiation-dominated early universe
with zero net baryon number. This scenario requires a first-order
electroweak phase transition in order to form bubbles of broken
electroweak phase nucleate within the surrouding plasma. The baryon
asymmetry is generated near the walls of these expanding bubbles.

EWBG satisfies all Sakharov criteria using processes that are naively
permitted in the Standard Model. However, there are two problems with
this model: the first-order first transition is not strong enough, and
it does not provide enough $CP$ violation to create the baryon
asymmetry currently observed in the universe. As a result, beyond the
Standard Model physics is necessary, such as Minimal Supersymmetric
Standard Model~(MSSM).

The $CP$ symmetry may be violated in QCD. The QCD Lagrangian consists
of two term
\begin{equation}
  \label{eqn:qcd}
\mathscr{L}= \mathscr{L}_{cp} - \theta \frac{n_f g^2}{32 \pi^2} F_{\mu \nu} \tilde{F}^{\mu \nu}~,
\end{equation}
where
$F_{\mu \nu} = \partial_\mu A_\nu + \partial_\mu A_\nu +
i[A_\mu,A_\nu]$, and
$\tilde{F}^{\mu \nu}=1/2 \epsilon^{\mu \nu \alpha \beta}F_{\alpha
  \beta}$ are the gluon field strength tensor and dual field strength
tensor respectively. The first term describes the gluons, quarks, and
their interaction, and the second term describes the interaction of
the gluons with vacuum. The existing strong bound on the
nEDM~($3 \times 10^{-26}$~e$\cdot$cm) requires the angle $\theta$ to
be very small~($\theta < 10^{-9}$). Since $\theta$ is a free
parameter, all the values between 0 and $2\pi$ are equally
likely. This is referred to as the ``Strong $CP$ problem''.

A solution to the strong $CP$ problem was proposed by Peccei and
Quinn~\cite{Peccei1977}. They proposed a new symmetry to explain the
$CP$ conservation of strong interactions, which in turn predicts a
neV~(Goldstone boson) pseudoparticle named the axion. To date the
axion has never been discovered although there is still a strong link
between axion and nEDM experiments.

\section{Neutron Electric Dipole Moment and Symmetry Breaking}
A permanent nEDM is an intrinsic property of a neutron. This
fundamental property is a measure for the separation of positive and
negative charges internal to the neutron. However, no nEDM has been
measured so far.

The interaction of the EDM $d$ of a spin-1/2 particle with the
electromagnetic field strength $F_{\mu \nu}$ in the relativistic
invariant form can be written as

\begin{equation}
\label{eqn:hamiltonianRELelectric}
H_d = \frac{d_n}{2} \bar{\psi} \gamma_5 \sigma_{\mu \nu} \psi F^{\mu \nu}~.
\end{equation}
Similarly, the interaction of a magnetic moment with the
electromagnetic field strength $F_{\mu \nu}$ is
\begin{equation}
\label{eqn:hamiltonianRELmagnetic}
  H_\mu = \frac{\mu_n}{2} \bar{\psi} i \sigma_{\mu \nu} \psi F^{\mu \nu}~.
\end{equation}

%To study the transformation properties of the Hamiltonian, it is
%interesting to see how all bilinear covariants behave under discrete
%symmetries transformation.
%In four-vector notation, the parity operator as a
%$4 \times 4$ matrix is

%\begin{equation}
%\label{eqn:parityMatrix}
%p = \left(
%  \begin{array}{cccc}
%    1 & 0 & 0 & 0 \\
%    0 & -1 & 0 & 0 \\
%    0 & 0 & -1 & 0 \\
%    0 & 0 & 0 & -1 
%  \end{array}
%  \right) ~,
%\end{equation}
%the time-reversal symmetry is
%\begin{equation}
%\label{eqn:TMatrix}
%p = \left(
%  \begin{array}{cccc}
%    -1 & 0 & 0 & 0 \\
%    0 & 1 & 0 & 0 \\
%    0 & 0 & 1 & 0 \\
%    0 & 0 & 0 & 1 
%  \end{array}
%  \right) ~.
%\end{equation}
%The transformation of all bilinear covariants is listed in
%Table~\ref{tab:transformation}.
%\begin{table}[h!]
%  \begin{center}
%    \renewcommand{\arraystretch}{2}
%\begin{tabular} {| c || c | c | c | c |} 
%  \hline
%  & current & P & C & T \\ \hline
%  \hline
%  Scalar & $\bar{\psi_1} \psi_2$ & $\eta_1 \eta_2^* \bar{\psi_1}\psi_2$ &  $\xi_1 \xi_2^* \bar{\psi_1}\psi_2$ &  $\zeta_1 \zeta_2^* \bar{\psi_1}\psi_2$ \\ \hline
%  Vector & $\bar{\psi_1}\gamma_\mu \psi_2$ &  $\eta_1 \eta_2^* \bar{\psi_1}\gamma_\mu \psi_2$ & - $\xi_1 \xi_2^* \bar{\psi_1}\gamma_\mu \psi_2$ & - $\zeta_1 \zeta_2^* \bar{\psi_1}\gamma_\mu \psi_2$ \\ \hline
%  Tensor &  $\bar{\psi_1} \sigma_{\mu \nu} \psi_2$ &   $\eta_1 \eta_2^* \bar{\psi_1} \sigma_{\mu \nu} \psi_2$ & - $\xi_1 \xi_2^* \bar{\psi_1} \sigma_{\mu \nu} \psi_2$ &  - $\zeta_1 \zeta_2^* \bar{\psi_1} \sigma_{\mu \nu} \psi_2$ \\ \hline
%  Pseuo-vector & $\bar{\psi_1} \gamma_{\mu}\gamma_5 \psi_2$ & - $\eta_1 \eta_2^* \bar{\psi_1} \gamma_{\mu}\gamma_5 \psi_2$ & $\xi_1 \xi_2^* \bar{\psi_1} \gamma_{\mu}\gamma_5 \psi_2$ &  $\zeta_1 \zeta_2^* \bar{\psi_1} \gamma_{\mu}\gamma_5 \psi_2$ \\ \hline
%  Pseudo-scalar & $\bar{\psi_1} \gamma_5 \psi_2$ & - $\eta_1 \eta_2^*\bar{\psi_1} \gamma_5 \psi_2$ & $\xi_1 \xi_2^*\bar{\psi_1} \gamma_5 \psi_2$ & $\zeta_1 \zeta_2^*\bar{\psi_1} \gamma_5 \psi_2$ \\\hline
%\end{tabular}
%\caption{Symmetry properties of all bilinear covariants  \label{tab:transformation}}
%\end{center}
%\end{table}


Assuming ${\bf{d_n}} = d_n \frac{\bf{J}}{J}$, and
$\boldsymbol{\mu_n} = \mu_n \frac{\bf{J}}{J}$ ,the interaction of a
nonrelativistic neutron with the electromagneticol field can be
described by the follwoing hamiltonian

\begin{equation}
  \label{eqn:hamiltonian}
 H= -\boldsymbol{\mu_n} \cdot \bf{B} - \bf{d}_n \cdot \bf{E}~,
 \end{equation}
where $\boldsymbol{\mu}_n$ is the magnetic moment of the neutron
interacting with the magnetc field $\bf{B}$, and $\bf{d}_n$ is
the electric dipole moment of the neutron interacting with the
electric field $\bf{E}$.

The properties of the Hamiltonian under discrete symmetries is
summarized in Table~\ref{tab:Hsymmetry}. Based on this, the first term
is $CP$-even and $T$-even, and the second term is $CP$-odd and
$T$-odd. Because of $CPT$ invariance, a nonzero EDM may exist if
both Parity and Time-reversal symmetries are broken.


\begin{table}[h!]
\begin{center}
\begin{tabular}{| l | l | l | l |} 
\hline
 & C & P & T \\ \hline
\textbf{B} & - &+ &- \\ \hline
\textbf{E} & -&- &+ \\ \hline
$\boldsymbol{\mu}$ &- &+ &- \\ \hline 
\textbf{d} & -&+ &- \\ \hline
\end{tabular}
\caption{Symmetry properties of different components of the EDM
  Hamiltonian  \label{tab:Hsymmetry}}
\end{center}
\end{table}
  
The nEDM measurement technique and a brief survey of nEDM measurements
conducted worldwide are presented in this chapter. Ultracold neutrons
are the main tool in the measurement of nEDM and their properties are
discussed below.
\section{Ultracold Neutrons\label{sec:ucnproperties}}
The measurement of the nEDM requires high neutron statistics.
Table~\ref{tab:ucnenergy} shows the various energy regimes of neutrons
and their corresponding velocities, temperatures and de Broglie
wavelengths which are related via
\begin{equation}
  \label{eqn:ucnenergy}
  E = \frac{1}{2} m v^2 = \frac{3}{2} k_B T = \frac{h^2}{2m \lambda^2}~,
\end{equation}
where $m$ is the mass of the neutron, $v$ is the neutron velocity,
$k_B$ is the Boltzmann constant, $T$ is the equivalent temperature,
$h$ is the Planck's constant, and $\lambda$ is the de Broglie
wavelength.

\begin{table}
  \centering
  \begin{tabular}{|c|c|c|c|c|}
    \hline
    Name & Energy $E$ & Velocity $v$ & Temperature $T$ & Wavelength $\lambda$ \\
    \hline
    \hline
    Fast & 10~MeV & $4.4 \times 10^7$~m/s & $7.7$~GK & 9.0~pm \\
    \hline
    Thermal & 0.0254~eV & $ 2.2 \times 10^3$~m/s & 290~K & 0.2~nm \\
    \hline
    Cold & 1~meV & 500~m/s & 8~K & 3~nm \\
    \hline
    Ultracold & 300~neV & 8~m/s & 3~mK & 50~nm \\
    \hline
  \end{tabular}
  \caption[Neutron names in different energy ranges and their
  corresponding velocities, temperatures, and wavelength]{Commonly
    used names for neutrons in different energy ranges and their
    corresponding velocity, temperature and
    wavelenght \label{tab:ucnenergy}}
\end{table}


Ultracold neutrons are neutrons with kinetic energies $\lesssim 300$~neV,
corresponding to velocities ~$\lesssim 8$~m/s, or temperatures
$\lesssim 3$~mK. UCN move so slowly that they can populate traps made
of matter, magnetic, and gravitational fields, and can be stored and
manipulated for several hundreds of seconds in such traps. Because of
their properties, UCN are a valuable tool for precise measurements in
fundamental physics.

% such as studies of quantum states of the neutron in the Earth's
% gravitational field or the measurement of the neutron EDM.
High precision studies of UCN and their interactions provide important
data for particle physics and cosmology. In addition, they enable
sensitive searches for new physics. Examples of the experiments using
UCN, which aim to discover new physics, are searches for a permanent
EDM of the
neutron~\cite{Baker2006,Serebrov2009,Lam_Gol,Altarev2010,Pendlebury2015},
precision measurements of the neutron
lifetime~\cite{pattie2018measurement,Paul2009,Wietfeldt2011,Arzumanov2000,Serebrov2005,Huffman},
and $\beta$-decay correlation
parameters~\cite{plaster2012measurement,Mendenhall,Broussard}, as well
as quests for dark matter candidates~\cite{Serebrov2008,Zimmer2010},
axion-like particles~\cite{Baessler,Serebrov2010,Afach2015}, Lorentz
invariance violations~\cite{Altarev2009}, and the measurements of the
quantum states of UCN in the gravitational field of the
Earth~\cite{Nesvizhevsky2003}.

%Among recent new topics addressed with UCN feature searches for
%‘‘mirror matter’’ as a viable candidate for dark matter, a sensitive
%test of Lorentz invariance, searches for a new fundamental force
%mediated by axionlike particles, and a demonstration of the effect of
%accelerated matter on the neutron wave.  More long-standing are
%efforts to improve the accuracy of the weak axial-vector and vector
%coupling constants of the nucleon derived from precise values of the
%neutron lifetime and the beta asymmetry, i.e., the asymmetry of
%electron emission with respect to the spin of the decaying
%neutron. These values crucially enter the calculation of reaction
%rates in big-bang nucleosynthesis and stellar fusion [16]. They are
%also applied to calculate various processes in particle physics such
%as for the calibration of antineutrino detectors, which is currently
%scrutinized in view of a ‘‘reactor antineutrino anomaly’’ hinting at
%the existence of sterile neutrinos [17,18].

The neutron is an electrically neutral hadron, and it participates in
all four fundamental interactions. As described below, UCN have
interesting kinetic energies in relation to these potential energies
of those interactions.


\subsubsection{The Gravitational Interaction}
A neutron has a mass of $m_n\approx 940$~MeV/c$^2$, and therefore, it has a
potential in the Earth's gravitational field as
\begin{equation}
V_g=mgh~.
\end{equation}
Here $h$ is the vertical displacement, and $g=9.8$~m/s$^2$ is the
acceleration due to the Earth's gravitational field.  In experiments
using thermal or cold neutrons, the effects of gravity can usually be
negligible due to the short survival times of the neutrons. However,
with the UCN experiments, since they are confined for up to several
hundreds of seconds, gravity has a significant influence.

Here
\begin{equation}
mg=102\; \text{neV/m}~,
\end{equation}
which is comparable to the UCN kinetic energy. This means, a UCN of
energy 200~neV can rise by at most 2~m.

\subsubsection{The Weak Interaction}
The weak interaction governs the radioactive $\beta$-decay of the
neutrons. A UCN decays into a proton, an electron, and an electron
antineutrino via
\label{neutrondecay}
\begin{equation}
n\longrightarrow p+e^{-}+\bar{\nu_{e}}~.
\end{equation}
The value of the neutron lifetime sets the maximum time constant with
which UCN can be stored. The current value of the neutron lifetime is
$880.2 \pm 1.0$~s~\cite{PDG2018} which is comparable to trapping times
for UCN. This sets ultimate upper bound on experiment time scale.

\subsubsection{The Electromagnetic Interaction}

Neutron is an electrically neutral, spin-1/2 particle that possesses a
magnetic dipole moment due to its internal structure, through which it
interacts with a magnetic field \textbf{B} as
\begin{equation}
  \label{eqn:vmag}
V_m=-\boldsymbol{\mu}_n \cdot \textbf{B}~,
\end{equation}
as discussed earlier, where
\begin{equation}
\vert \boldsymbol{\mu}_n \vert =60 \; \text{neV/T}~.
\end{equation}

In an inhomogeneous magnetic field, UCN experience a force described
by
\begin{equation}
  \label{eqn:fmag}
  {\bf{F}}_m = - {\boldsymbol{\nabla}V}_m = \pm \vert {\boldsymbol{\mu}}_n \vert \boldsymbol{\nabla} \bf{B}.
\end{equation}
In the nEDM measurements, the interaction of UCN with the magnetic
field is used to polarize UCN, and to measure its polarization at the
end of the measurement cycle~(see Section~\ref{sec:Ramsey}). In
Eqn.~\ref{eqn:fmag}, the sign $\pm$ corresponds to the relative
orientation between the magnetic moment and the magnetic field.  UCN
of anti-parallel spin to the magnetic field~(magnetic moment parallel)
are called ``high field seekers'', have negative $V_m$, accelerate
towards strong magnetic fields, and are attracted to it. UCN with
parallel spin~(and thus magnetic moment anti-parallel) to the magnetic
field, are called ``low field seekers'', have positive $V_m$, and
are repelled by the magnetic field.

Eqn.~\ref{eqn:fmag} is true only if UCN move adiabatically through the
magnetic field. This condition will be fulfilled when the Larmor
precession frequency of UCN is smaller than the changes in the
magnetic field in the rest fram of the UCN. However, since the UCN
have low speeds, this condition is easily fulfilled. If the UCN spin
adiabatically traces the magnetic field, it will be fully polarized,
which can be achieved by passing the UCN through a strong~$\sim 6$~T
magnetic field.

%If the magnetic field $\textbf{\textit{B}}$ is inhomogeneous and the
%neutron spin traces the magnetic field adiabatically, the neutron will
%experience a force given by:

%\begin{equation}
%\label{eqn:emforce}
%\textbf{\textit{F}}_{mag}= - \mathbf{\mathit{\nabla}} V_{mag}=\pm
%\mu_n \mathbf{\mathit{\nabla}} \vert \textbf{\textit{B(r)}} \vert.
%\end{equation}
%The neutrons that experience a force towards regions of higher
%magnetic field strength are called high-field seekers. Conversely if
%the neutrons experience a force towards regions of lower magnetic
%field (a 3D minimum in the $\textbf{\textit{B}}$ field) strength they
%are called low-field seekers. It is this force that allows UCN with
%sufficiently low energy to be confined by magnetic field gradients.
%%If the UCN spin adiabatically traces the magnetic field, it will be fully polarized
%%which can be achieved by passing UCN through a strong $\sim$~6~T magnetic field.
%Furthermore, Nuclear Magnetic Resonance (NMR) experiments can be
%conducted on UCN. For example, NMR is used to measure the EDM of
%neutrons (the Ramsey technique) where UCN are placed in aligned
%electric and magnetic fields.

\subsubsection{The Strong Interaction}
Neutrons and protons are bound in the nucleus by the strong
interaction. However, this interaction has a short range, and it only
affects the neighbouring nuclei. The Woods-Saxon potential
approximately describes the nucleons interaction inside the atomic
nucleous
\begin{equation}
  \label{eqn:woodsax}
  V(r) = - \frac{V_0}{1+\exp(\frac{r-R}{b})}~,
\end{equation}
where $V_0$ represents the potential well depth, $b$ represents the
surface thickness of the nucleus, and $R = r_0 A^{1/3}$ where
$r_0 = 1.25$~fm and $A$ is the mass number. For neutrons and protons
the depth is $V_0 \approx 40$~MeV.
%The UCN energy and their binding
%energy and the depth of the potential differ by many orders of
%magnitude.
%As a result, it is not possible to use perturbation theory
%to describe neutron scattering.
Fermi realized that it is possible to introduce an equivalent
potential which can be used to calculate the small changes in the
wavefunction outside the range of the interaction by perturbation
theory.


The scattering of a neutron from a nucleus can be described as a
superposition of an incoming plane wave and a scattered spherical
wave:

\begin{equation}
  \psi(z, \theta) = e^{ikz} + f(\theta)\frac{e^{ikz}}{z}~,
\end{equation}
where $f(\theta)$ is the angle dependent scattering amplitude and is
determined by the boundary condition at $r = R$. Since the wavelength
of the UCN is much larger than the range of the strong interaction
$R$, there is no angular momentum transfer, and the process is
dominated by the s-wave~($l = 0$) scattering. In this case, the
scattering amplitude $f$ does not depend on the incident angle
\begin{equation}
f(\theta) = const. = -a~.
\end{equation}
Since the differential cross section becomes
$\frac{d\sigma}{d\Omega} = |f(\theta)|^2 = a^2$ , where $a$ is the
scattering length, a quantity that could be experimentally measured.

The interaction of an incident neutron with a nucleus can therefore
alternatively be described by a single $\delta$-function

\begin{equation}
  V(\textbf{r}) = \frac{2 \pi \hbar^2}{m_N} a \delta^{(3)}(\textbf{r})~,
\end{equation}
where $m_N$ is the mass of neutron, and $a$ is the scattering
length. Therefore, the interaction of an incident neutron with a
liquid or a solid can be described by sum of equivalent $\delta$
functions

\begin{equation}
  \label{eqn:vFermi}
  V(\textbf{r}) = \frac{2\pi \hbar^2}{m_N} \sum_i a_i \delta (\bf{r} - \bf{r_i})~,
\end{equation}
where $r_i$ is the position of the $i$th nucleus, $a_i$ is the
scattering length with the $i$th nucleus, and the sum is over all the
nuclei. This is the Fermi pseudopotential. Because of UCN's large
wavelength, this equation can be written as
\begin{equation}
  V(\textbf{r}) = \frac{2\pi \hbar^2}{m_N}\sum_i N_ia_i\Theta(\textbf{r})~,
\end{equation}
where $N_i$ is the number density in the material i, and $\Theta(r)$
is the step function inside the domain of the material. Here the sum
is over the domains with of nuclei with the same scattering length
$a_i$. In neutron physics, this potential is typically called neutron
optical potential, since if the energy of the neutron is less than the
optical potential $E < V$, the neutron will be fully reflected from
the material surface under any angle of incidence. This sets a limit
on the UCN velocity.

UCN can be lost when reflected from the material walls.  This can be
due to the upscattering in which UCN absorb energy, or absorption in
which UCN get captured by a nucleus in the reflecting material. To
include the losses in the potential, the optical potential is usually
written as
\begin{equation}
  U(r) = V(r) - iW(r)~,
\end{equation}
where $W(r) = \Sigma N_i \sigma^i_l v$ with $N_i$ being the density,
$\sigma^i_l$ being the loss cross-section~(absorption plus inelastic
scattering) of nuclei $i$, and $v$ being the velocity. If we solve the
Schr\"{o}dinger equation with this potential, we get an additional
decay of probability density as

\begin{equation}
\rho = \rho_0 e^{-2Wt/\hbar}~,
\end{equation}
where $1/\tau_{abs} = 2W/\hbar$.  The ratio
$f = \frac{W}{V} = \frac{\sigma_l k}{4 \pi a}$ is the UCN loss
coefficient, where as earlier, $\sigma_l$ is the total loss
cross-section for neutrons with wavenumber $k$, and $a$ is the
scattering length~\cite{ucnbook}.

For almost all materials $V \gg W$. The total energy of UCN are
$E < V$ for many materials and therefore, they are reflected from
material surfaces at all angles of incidence. For $V > E$, the mean
probability of neutron loss on reflection from an abrupt potential
step~($V \gg W$), averaged over all angles of incidence, is given
by~\cite{richardson1991measurement}

\begin{equation}
  \mu(E) = 2f \left[ \frac{V}{E} \mathrm{arcsin} \left( \sqrt{\frac{V}{E}} \right) - \sqrt{\frac{\left( V - E\right)}{E}} \right]~,
\end{equation}
where $f$ is the UCN loss coefficient as described above.


The strong interaction plays a crucial role in the nEDM
measurements. Choosing certain materials enables us to store and guide
UCN to the measurement cell. The highest known value for the optical
potential is $V_F=335$~neV, and is measured for $^{58}$Ni.

%\subsection{Superthermal Sources of Ultracold Neutrons}



%The search for a nonzero neutron EDM provides a promising route to
%investigate new mechanisms of CP violation beyond the standard
%model. These in theory could help explain the matter-antimatter
%asymmetry in the Universe. At the present best level of sensitivity
%$d_n = 3.0 \times 10^{-26}$~e$\cdot$cm (90\%
%C.L)~\cite{Pendlebury2015} which was limited by counting
%statistics, severe constraints on new sources of CP violation were
%placed.

%As most other experiments with UCNs, the EDM search has been
%performed using a long-serving source~\cite{Steyerl1986} at the
%high-flux reactor of the Institut Laue Langevin (ILL) in Grenoble,
%France. It employs a neutron turbine for a phase-space transformation
%of very cold neutrons from a liquid deuterium moderator down to the
%energy range of UCN, whose high-energy limit is set by the neutron
%optical potential of the material selected for a UCN trap (such as
%252 neV for beryllium) or by the magnetic potential provided by field
%gradients in a magnetic bottle (60 neV=T). With UCN densities in the
%order of 10 per cm3 made available for experiments in a typical
%configuration of the UCN extraction from the turbine [23], the ILL
%source has defined the state of the art for more than 25
%years. However, notably,
%%%%%%%%%%%%%%%%%%%%%%%%%%%%%%%%%%%%%%%%%%%%%%%%%%%%%5
%The prospect to make an important discovery in refining the neutron
%EDM search has strongly motivated many research groups to develop next
%generation UCN sources~\cite{Golub75,Zimmer2011} which aim to
%improve the available UCN densities by more than 2 orders of
%magnitude.
%%%%%%%%%%%%%%%%%%%%%%%%%%%%%%%%%%%%%%%%%%%%%%%%%%%%%%%%%%%5
%\subsection{Properties of UCN}
%expand the first two paragraphs of Leung's paper. Also look at
%Leung's thesis.  Neutron energy and its velocity are related as
%$E_n=\frac{m_n v^2}{2}= \frac{\hbar^2 k^2}{2 m_n}=\frac{h^2}{2 m_n
%\lambda_n^2}$ where $m_n$ is the neutron mass, $v$ is the neutron
%velocity and $k=\frac{2 \pi}{\lambda_n}$ is the wave number. The
%kinetic energy is related to temperate as $E_n=k_B T_n$ where $k_B$
%is the Boltzmann constant.

%UCN have velocities less than 8~m/s and energies about 260~neV which
%corresponds to temperatures below 2~mK.  The kinetic energy of UCN is
%less than the neutron optical potential of well-chosen materials and
%so they can reflect from material surfaces at all incident angles,
%allowing them to be stored in a vessel and studied for times
%approaching the neutron lifetime.



\section{Superthermal UCN sources}
\label{sec:ucn_with_heII}



In thermal UCN sources, neutrons are extracted from a distribution
almost in thermal equilibrium with a moderation system. The UCN
turbine source at the Institute Laue-Langevin~(ILL) extracted very
cold neutrons vertically from a cold source~(liquid deuterium), and
slowed them down using the mechanical action of a
turbine~\cite{Steyerl1986,Steyerl1975}. Here cold neutrons with
velocities of $\sim$~40~m/s are decelerated by reflection from a set
of curved turbine blades moving with a velocity $\sim$~20~m/s in the
same direction as the neutrons. A UCN density of $\sim$~40~UCN/cm$^3$
was achieved with this method~\cite{ucnbook,Albert_talk}. The current
UCN density of this source is 110~UCN/cm$^3$ for neutrons with
velocities $<$~7~m/s~\cite{Steyerl1986}.


%\subsection{Goals of Superthermal UCN Sources}
In 1975 it was shown that, it is possible to achieve higher steady
state UCN densities corresponding to temperatures much lower than the
temperature of the moderator~\cite{Golub75}. These are called the
``superthermal converters''. Here thermal or cold neutrons are
inelastically scattered and transfer their kinetic energy to an
excitation of the converter medium~(e.g., to a phonon).  Superthermal
sources have the ability to provide much higher UCN densitys than
conventional sources such as the ILL turbine source. The best
candidates for the superthermal converters to date are liquid $^4$He
and solid deuterium.


%UCN are a powerful tool to study new physics and the properties of
%the neutrons.  An intense UCN beam is a common need for all of these
%experiments.  Producing a high intensity UCN source is an essential
%need for such studies.  Earlier method Before using superthermal
%converters, UCN were produced by neutron turbine
%sources~\cite{Steyerl1986,Steyerl1975}. In turbine source,
%cold neutrons with velocities of $\sim$ 40~m/s are brought into UCN
%range by reflection from a set of curved turbine blades moving with a
%velocity $\sim$ 20~m/s in the same direction as the neutrons. A UCN
%density of $\sim$30-40 UCN/cm$^3$ was achieved with this
%method~\cite{ucnbook}.

%In earlier methods of UCN production, it was assumed that the maximum
%UCN density would be achieved if the incident neutron flux is
%thermalized to the moderator~\cite{Shapiro}. The maximum
%achievable UCN density with these methods is
%$10^2-10^3$/cm$^3$~\cite{ucnbook}. The process of extracting UCN
%has a considerable degree of loss, which means, the available UCN
%will always be much less than this~\cite{ucnbook}.  The low
%density of UCN produced by thermal sources was the main constrain of
%high precision measurements such as neutron $\beta$-decay with
%UCN~\cite{Huffman} and neutron electric dipole moment
%measurement~\cite{Steyerl1986,Harris99}.

 
\subsection{Basic Idea of Superthermal UCN Sources\label{sec:basic_idea}}
The mechanism of a superthermal UCN source is the following.  An
incident neutron can lose almost its entire energy in a single
scattering event by creating excitations~( e.g., phonons) in a
converter medium~\cite{ucnbook, Golub75}. Because of the loss in the
kinetic energy, this process is called downscattering. The reverse
process is called upscattering, where a UCN absorbs kinetic energy from
the medium.


Consider a simple model for the medium as a two-level system with an
energy gap $E_0^*$. A neutron can excite a quasi-particle from the
lower state to the higher state by transferring the energy $E_0^*$. A
quasi-particle from the higher state can fall down to the lower state
by transfer of the energy $E_0^*$ to a neutron.  The principle of
detailed balance links the cross-section for upscattering
$\sigma(E_{\text{UCN}} \rightarrow E_{\text{UCN}}+E_0^*)$ and
downscattering
$\sigma(E_{\text{UCN}}+E_0^* \rightarrow
E_{\text{UCN}})$~\cite{ucnbook}

\begin{equation}
\label{eqn:detailed_balance}
\sigma(E_{\text{UCN}} \rightarrow E_{\text{UCN}}+E_0^*)= \frac{(E_{\text{UCN}}+E_0^*)}{E_{\text{UCN}}}
e^{-\frac{E_0^*}{k_B T}}\sigma(E_{\text{UCN}}+E_0^* \rightarrow E_{\text{UCN}})
\end{equation}
where $T$ is the temperature of the medium, $E_{\text{UCN}}$ is the
energy of the UCN, and $k_B$ is the Boltzmann constant.

In general, $\sigma(E_{\text{UCN}}+E_0^* \rightarrow E_{\text{UCN}})$
is practically independent of $T$, so that for
$E_0^* \gg k_B T \gg E_{\text{UCN}}$, the upscattering cross-section
for UCN can be made arbitrarily small by decreasing the
temperature. If the converter is now placed in a neutron flux at a
temperature $T_n \geq E_0^*$, there will be a significant number of
downscattering events, and a negligible number of upscattering events.

If the converter is contained in a vessel whose walls are good UCN
reflectors with potential $V \gg V_m$, where $V_m$ is the UCN potential
of the converter, and the walls are transparent to the neutrons of
energy $E_0^*$, then UCN will build up in the moderator to a density
until the rate of loss is equal to the rate of UCN production.

%Coherent inelastic scattering can provide the equivalent of a
%two-level system for UCN \cite{ucnbook}.  A neutron at rest can
%only absorb a phonon with energy E$_0^*$ and a neutron with
%E$>$E$_0^*$ can come to rest by emitting a phonon with energy
%E$_0^*$.  The upscattering process is suppressed by a Boltzmann
%factor $e^{-E_0^*/{k_BT}}$. If the medium is sufficiently cold and
%the excitation energy introduced by the neutrons can be cooled away,
%upscattering becomes negligible.  If the converter is at thermal
%equilibrium at temperature $T$, then $E_0^* \gg k_B T \gg E_{\text{UCN}}$
%where $E_{\text{UCN}}$ is the UCN energy. Only neutrons with energies close
%to $E_0^*$ can scatter in the converter medium. In addition, the UCN
%bottle wall has to have a much higher effective potential than the
%converter medium to prevent UCN loss.

The steady-state UCN density in the source is given by
\begin{equation}
\label{ucndensity}
\rho_{\text{UCN}}=P_{\text{UCN}} \tau,
\end{equation}
where $P_{\text{UCN}}$~(UCN/cm$^3 \cdot$s) is the UCN production rate,
and $\tau$~(s) is the UCN mean lifetime in the system.  The mean
lifetime $\tau$ of the UCN in the vessel is restricted by a variety of
possible loss mechanisms
\begin{equation}
\frac{1}{\tau} = \frac{1}{\tau_a}+ \frac{1}{\tau_W}+\frac{1}{\tau_{up}}+\frac{1}{\tau_{\beta}},
\end{equation}
where $1/\tau_a$ is the UCN absorption rate in the medium, $1/\tau_W$
is the rate of the UCN loss on the walls, $1/\tau_{up}$ is the neutron
loss due to the upscattering in the medium, and $1/\tau_{\beta}$ is
the $\beta$-decay losses.

%An important factor in choosing a UCN source is the absorption rate of the neutron. As a result, pure
Pure deuterium and liquid $^4$He are good candidates for superthermal
conductors, possessing a balance of high production rate, and small
neutron absorption cross-section, and upscattering rate.


\subsection{UCN Production by Superfluid $^4$He}

\subsubsection{Superfluid $^4$He Definition}

$^4$He is an isotope of helium with two protons and two neutrons. It
has an integer spin of zero, which makes it a boson. As a result, it
follows the Bose-Einstein statistics. It has two liquid states known
as He-I and He-II. The He-I phase is the normal fluid phase. The He-II
phase can be described by the so-called ``two-fluid model'', which
consists of normal liquid helium and superfluid helium.
% with zero viscosity, and zero entropy.
Fig.~\ref{fig:phasetransition} shows the phase transition diagram of
$^4$He. The two phases are separated out by the $\lambda$-line. The
phase transition happens at 2.172~K. Below 1~K, the liquid is mostly
the superfluid component of the two-fluid model.

In this thesis, I use the term ``superfluid helium'' generally to
refer to He-II cooled significantly below the lambda point, which is
nearly synonymous with that component of the two-fluid model.

\begin{figure}[h!]
  \centering \includegraphics[width=0.7\textwidth]{phasetransition.png}
  \caption[Phase diagram of $^4$He]{The phase diagram of $^4$He. Here
    the normal fluid phase or He-I and the superfluid phase or H-II
    are shown.}
\label{fig:phasetransition}
\end{figure}

Because of its zero viscosity, superfluid helium has the ability to
flow through very small capillaries or narrow channels without
experiencing any friction at all. The climbing of He-II along the
surface is called ``film flow''.


\subsubsection{Superfluid Helium Converter}

Superfluid $^4$He is an attractive candidate as a UCN source, and was
studied in Ref.~\cite{Golub77}. The dominant production mechanism in
the superfluid helium is the excitation of a single phonon at the
crossing of the free neutron and phonon dispersion curves, with a
momentum $q\sim 0.7$/\AA~\cite{Brome2001}, and energy 1~meV
corresponding to a neutron wavelength 8.9~\AA. The availability of
8.9~\AA~cold neutrons is crucial and their flux must be maximized. In
superfluid helium, the upscattering losses become smaller than
$\beta$-decay losses below $T \sim 0.7$~K. It has zero neutron
absorption cross-section, resulting in $\tau_a \rightarrow \infty$,
which makes it a good candidate as a UCN source.

%For a long UCN lifetime in superfluid helium, besides the low
%temperature of the converter, the $^3$He contamination must be low
%($^3$He/$^4$He $\le 10^{-12}$), due to its large absorption
%cross-section, which requires $^4$He purification.
There are two types of UCN sources based on superfluid helium: sources
where experiment and source are combined in one apparatus, and the
measurement is performed inside the superfluid helium, and
extracted-UCN sources where the source is an apparatus on its own, and
delivers neutrons to experiments at room temperature connected to it
by UCN guides.



\subsubsection{UCN Production Rate with Single Phonon scattering in
  Superfluid
  helium\label{sec:UCN_production}}
UCN can be produced in superfluid helium when the energy of the
incident neutrons is equal to that of the one phonon excitation in the
converter medium~\cite{Korobkina2002,Schmidt2009,Golub77}. The
incident neutrons then scatter down to UCN in the converter medium
while creating a phonon.
%The crossing of the dispersion curve for excitations in superfluid
%helium and the free energy of the neutron shows that the neutron
%wavelength should be 0.89~nm~\cite{Brome2001} or 8.9~\AA ~in
%order to excite phonon transitions.
\begin{figure}[h!]
\begin{center}
   \includegraphics[width=0.7\textwidth]{FIG1_2.PNG}
   \caption[Free neutron and superfluid helium dispersion
   relation]{\cite{PSI_news} Dispersion relation of superfluid
     helium~(c) and of the free neutron~(a). Neutrons with
     $E\simeq 1$~meV and wavenumber $q \simeq 0.7$/\AA~can excite a
     single phonon with the same energy and momentum and be
     downscattered to UCN energy range. The UCN production rate
     (b)(circles) shows the dominance of this single phonon process
     with respect to multiphonon processes at higher momentum $q$.
%    The two curves cross at $q=0$ and at $q=q^*$, which corresponds
%    to a neutron wavelength of 0.89~nm, oe an energy of 12~K.
    }
%     \vspace{-2.em}
    \label{fig:FIG1}
    \end{center}
\end{figure} 
Fig.~\ref{fig:FIG1} shows the dispersion relation of the superfluid
helium and a free neutron. A neutron at rest can absorb energy $\hbar
\omega$ and momentum $\hbar q$ with
\begin{equation}
\label{neutron_energy}
\omega=\frac{\hbar q^2}{2m}~,
\end{equation}
where $m$ is the mass of the neutron. A neutron with this energy and
momentum can come to rest after transferring its energy and momentum
to the superfluid $^4$He. For single phonon interactions, which are
usually dominant, the superfluid helium can only exchange quantities
of energy and momentum that are related by the dispersion curve

\begin{equation}
\label{dispersion_helium}
\omega=\omega(q) \simeq cq~,
\end{equation}
where $\omega$ is the energy of the phonon, $q$ is the phonon's
momentum, and $c$ is the speed of sound in the moderator. Here an
approximation is used to simplify the discussion. The neutrons can
only come to rest by emission of a single phonon, if they have the
resonant energy $E_0^*$ given by the intersection of
Eqns.~(\ref{neutron_energy}) and (\ref{dispersion_helium})

\begin{equation}
\omega(q)=cq=\frac{\hbar q^2}{2m}~,
\end{equation}
and so
\begin{equation}
q^*=\frac{2mc}{\hbar}~.
\end{equation}

%Fig.~\ref{fig:FIG1} shows that an incident neutron can lose its
%entire energy by creating a phonon excitation inside the converter
%medium if it has the energy amount equal to the phonon excitations in
%the converter medium. If this energy is $E_0^*=\hbar^2 k_0^*/2m$,
%using Eqns.~(\ref{dispersion_helium}) and~(\ref{neutron_energy}),
%$k_0^*=2mc/ \hbar$.

The differential cross-section for neutron scattering is given by the
dynamic scattering function $S(q,\omega)$, which is the Fourier
transform of the Van Hove correlation function $G(r,t)$ in space and
time of the superfluid helium~\cite{Squires}:

\begin{equation}
\frac{d\sigma}{d\omega}=b^2 \frac{k_2}{k_1}S(q,\omega) d\Omega,
\end{equation}
where $b$ is the bound neutron scattering length for $^4$He,
$\hbar k_1$ is the momentum of the incident neutrons, and
$\hbar k_2=\hbar k_{\text{UCN}}$ is the momentum of the UCN. The
quantity $S(q,\omega)$ has been measured in great
detail~\cite{S_func1,gibbs1999collective,S_func3}. Performing the
change of variables,

\begin{equation}
d\Omega=2 \pi \sin \theta d \theta = 2 \pi \frac{q dq}{k_1 k_2}
\end{equation}
gives

\begin{equation}
 \frac{d\sigma}{d\omega}=2\pi b^2 \frac{k_2}{k_1}S(q,\omega)\frac{q
   dq}{k_1 k_2}=2\pi b^2 S(q,\omega)\frac{q dq}{k_1^2}~.
\end{equation}
This may effectively be integrated over the limits on $q$ which are

\begin{equation}
  k_1-k_2 < q < k_1+k_2~.
\end{equation}
Since
\begin{equation}
k_2=k_{\text{UCN}} \ll k_1~, \; \; q\sim k_1~,
\end{equation}
we may write $dq=2k_{\text{UCN}}$. This results in the cross-section
being related to $S(q,\omega)$ evaluated on the incident neutron's
dispersion curve:

\begin{equation}
\frac{d\sigma}{d\omega}=4\pi b^2 \frac{k_{\text{UCN}}}{k_1}S \left(
k_1, \omega=\frac{\alpha k_1^2}{2} \right),
\end{equation}
where $\alpha=\frac{\hbar}{m}=4.14$~meV/\AA$^2$, and $S(q,\omega)$
assumed to be constant over the narrow range $dq$. The approximation

\begin{equation}
\omega=\frac{\hbar (k_1^2-k_2^2)}{2m}=\frac{\alpha}{2} (k_1^2 -
k_2^2)\approx \frac{\alpha}{2}k_1^2
\end{equation}
has also been used.

The UCN production rate is given by
\begin{equation}
\label{UCN_production}
P(E_{\text{UCN}}) dE_{\text{UCN}} = N_{\text{He}} \int \frac{d\Phi
  (E_1)}{dE}\cdot \frac{d \sigma}{d \omega}(E_1 \rightarrow
E_{\text{UCN}}) dE_1 dE_{\text{UCN}}~,
\end{equation}
where $\frac{d\Phi (E_1)}{dE}$ is the differential incident neutron
flux, $N_{\text{He}}$ is the atomic density in the liquid helium, and
$\frac{d \sigma}{d \omega}(E_1 \rightarrow E_{\text{UCN}})$ is the
energy differential cross-section for the inelastic neutron scattering
or the probability of the incident neutrons with energy $E_1$ to
scatter from the helium nucleus and become UCN.  Then

\begin{equation}
\label{eqn:He_P_rate}
\begin{split}
\int _0 ^{E_c} P(E_{\text{UCN}})dE_{\text{UCN}} &= N_{\text{He}} 4 \pi b^2
\alpha^2 \left[ \int \frac{d\Phi(k_1)}{dE} S \left( k_1,
  \omega=\frac{\alpha k_1^2}{2} \right)dk_1 \right] \int_0^{k_c}
k_{\text{UCN}}^2dk_{\text{UCN}} \\ &=N_{\text{He}} 4 \pi b^2 \alpha^2 \left[
  \int \frac{d\Phi(k_1)}{dE} S \left( k_1, \omega=\frac{\alpha
    k_1^2}{2} \right) dk_1 \right] \frac{k_c^3}{3}\;
\text{UCN}/\text{cm}^3 \text{s},
\end{split}
\end{equation}
where $E_c$ and $k_c$ are the critical UCN energy and wave vector of
the walls of the storage chamber. This way of writing the UCN
production rate is more general, and it is useful to calculate the
single phonon and multiphonon contributions to the UCN production
rate. The one phonon production rate is found by evaluating
Eqn.~(\ref{eqn:He_P_rate}) over the one phonon peak ($q^*=0.7$/\AA).
Thus
\begin{equation}
P_{\text{UCN}}=9.44 \times 10^{-9}\frac{d\Phi (E_1^*)}{dE_1^*} \;
\text{UCN}/\text{cm}^3,
\end{equation}
 where $E_1^*$ is the energy of the incident neutrons at the one phonon peak.
%\begin{equation} P_{1p}=N4\pi b^2 S^* \alpha \beta
%\frac{k_c^3}{3k^*}\frac{d\Phi (E_1^*)}{dE} \; \text{UCN}/\text{cm}^3
%, \text{s} \end{equation}





% The differential cross-section $\frac{d \sigma}{d E_{UCN}}$ is given
% by \begin{equation} \label{energy_differential_cross_section}
% \frac{d\sigma}{d E_{UCN}}= 4 \pi b^2 \frac{k_{\text{UCN}}}{k_0}S(q,
% \omega) \end{equation} where $b$ is the bound neutron scattering
% length for $^4$He, $k_{\text{UCN}}$ is the UCN momentum and $S$ is
% the Fourier transform of the Van Hove correlation function $G(r,t)$
% in space and time of the superfluid helium~\cite{Squires}.
% Here \begin{equation}
% \omega=\frac{E_0-E_{UCN}}{\hbar} \end{equation} but since $E_0 \gg
% E_{UCN}$, then \begin{equation} \omega \simeq
% \frac{E_0}{\hbar} \end{equation}
% and, \begin{equation} \label{wave_vector_transfer} k_0 -
% k_{\text{UCN}} < q < k_0 + k_{\text{UCN}} \end{equation} where
% $k_{\text{UCN}} \ll k_0$ and therefore $q \simeq k_0$. Therefore
% $\omega=\frac{\alpha k_0^2}{2}$ where $%
% \alpha=\frac{\hbar}{m}=4.14$ meV \AA$^{-2}$. By substituting
% Eqn.~(\ref{energy_differential_cross_section}) to
% Eqn.~(\ref{UCN_production}), the general form of UCN production rate
% will be

% \begin{equation} \label{UCN_Production_rate_helium} P(E_c)= N_{\text{He}} 4
% \pi b^2 \alpha^2 \left[ \frac{d\Phi (k_0)}{dE} S(k_0,\omega) dk_0
% \right] \frac{k_c^3}{3} \end{equation} with the units of UCN
% cm$^{-3}$s$^{-1}$ \cite{Korobkina2002}. $E_c$ and $k_c$ are the
% critical UCN energy and wave vector of the walls of the storage
% chamber.  The UCN density produced by one-phonon excitation in
% superfluid helium was first calculated by Golub and Pendlebury
% \cite{Golub77}. They found a UCN density of $\rho_U\simeq 300$
% n~cm$^{-3}$ for a cylindrical bottle with radius 10 cm and diameter
% 20 cm. This was a substantial improvement over all existing UCN
% sources which yield $\rho\lesssim 1$ n~cm $^{-3}$.

%To achieve such UCN density, the density of $^3$He must be low enough
%so that the neutron absorption by $^3$He does not significantly
%affect the storage time in the vessel.

%In the first UCN production, the UCN output was a factor of 50 lower
%than expected \cite{Ageron1978}.  The UCN production rate (first
%calculated by Pendlebusyin 1982)can be calclated by evaluating
%Eqn.~(\ref{UCN_Production_rate_general}) over the one phonon peak
%which is at $k_0^*=0.7 \AA ^{-1}$.

% \begin{figure}[h!]  \begin{center}
% \includegraphics[width=0.5\textwidth]{FIG1.PNG} \caption{The
% dispersion curve for excitations in superfluid helium and the free
% energy of the neutron as a function of momentum transfer
% $q$~\cite{Brome2001}. The two curves cross at $q=0$ and at
% $q=q^*$, which corresponds to a neutron wavelength of 0.89~nm, oe an
% energy of
% 12~K.}  \vspace{-2.em} \label{fig:FIG1} \end{center} \end{figure}



% \item[$\bullet$] why helium? (small neutron absorption cross-section
% ) \item[$\bullet$] UCN production rate in superfluid He calculation
% (Golub77) \item[$\bullet$] Experiment shows the production rate
% matches theory (Ageron1978) and recent UCN rates
% (Zimmer2011) \item[$\bullet$] Experiment design (for example ILL
% design Baker 2003, Masuda2002, PSI source
% Anghel2009) \item[$\bullet$] Temperature of the helium source,
% (below 1K Masuda2012 )

% \item[$\bullet$] UCN storage time in superfluid helium (Golub79)
% \begin{description}

%%%%%%%%%%%%%%%%%%%%%%%%%%%%%%%%%%%%%%%%%%%%%%%%%%%%%%%%%%%%%%%%%
%%   FOURTH EDIT UP TO HERE
 %%%%%%%%%%%%%%%%%%%%%%%%%%%%%%%%%%%%%%%%%%%%%%%%%%

 \subsubsection{Multiphonon Scattering Contribution in UCN Production
   in Superfluid helium}
%So far, the UCN production was demonstrated by single phonon emission. Even though one phonon emission is the predominant process in UCN production for a monochromatic UCN beam, the
 For polychromatic neutron sources, UCN can also be produced by
 multiphonon processes in superfluid
 $^4$He~\cite{Korobkina2002,Schmidt2009}. Multiphonon production of
 UCN with various energy spectra of the neutron flux has been studied
 in Ref.~\cite{Korobkina2002}.  Fig.~\ref{fig:Korobkina2002} shows the
 energy spectra $\frac{d\phi}{dE}$ for three sources as a function of
 momentum $q$, compared to the dynamic scattering function
 $S(q,\omega=\hbar q^2/2m)$. The peak at $q=0.7$/\AA~in the dynamic
 scattering function corresponds to the one phonon excitation by
 superfluid helium. The values of $S$ above 1.2/\AA~are
 extrapolated. The value of $S$ above 2/\AA~are essentially zero.  The
 UCN production from one phonon and multiphonon processes have been
 calculated for three input neutron spectra: SNS ballistic guide,
 PULSTAR MC flux and HMI polarized flux.  The multiphonon contribution
 to the UCN production is calculated using Eqn.~(\ref{eqn:He_P_rate}),
 and calculating
 $\int \Phi(E_1)S(k_1,\omega=\frac{\alpha k_1^2}{2}) dk_1$.  The
 result showed that, for sources where He-II is exposed to the total
 thermal flux or at a dedicated spallation source, the multiphonon
 contribution can amount to slightly more than a factor of 2 increase
 in the UCN production.





\begin{figure}[h!]
\begin{center}
   \includegraphics[width=0.7\textwidth]{Korobkina2002.PNG}
   \caption[Cold neutron energy spectrum and $S$ function]{The energy
     spectrum of the incident cold neutron flux from three sources
     compared to the dynamic scattering function
     $S(q,\omega=\frac{\alpha k_1^2}{2})$~/meV as a function of
     $q$~/\AA.}
%     \vspace{-2.em}
    \label{fig:Korobkina2002}
    \end{center}
\end{figure} 



%They took the energy spectrum of the neutron flux from three sources
%and extrapolated it for values of $q > 1.2$ \AA $^{-1}$ and they
%compared the results to the scattering function.

%The proposed UCN source at the North Carolina State (NC) locates a
% UCN source in the thermal column of the campus 1-MW PULSTAR reactor
% after removing the graphite. The main point of this design was the
% independence of the UCN production process to the direction of the
% incident neutrons. The proposed UCN source at the Spallation Neutron
% Source (SNS) places a superthermal UCN source on a monochromatic
% 0.89~nm beam at a guide tube. Two types of guides as ``ordinary''
% supermirror guide and ``ballistic'' guide. The ballistic guide has
% more flux at the critical 0.89~nm wavelength but less flux at
% wavelengths shorter than 0.6~nm.  The results are summarized in
% Table~\ref{tab:multiphonon}.  \begin{table} \begin{center} \begin{tabular}{|c|c|c|c|c|c|}
% \hline & NC state & SNS ord & SNS ball & HMI a.u. & Maxwell
% \\ \hline Multi-ph & 490 & 1.0 & 0.94 & 4.7 & 1.7\\ \hline Single-ph
% & 375 & 1.8 & 2.4 & 5.5 & 1.5 \\ \hline Mph/1ph & 1.4 & 0.55 & 0.4 &
% 0.85 & 1.13 \\ \hline \end{tabular} \caption{Predicted production
% rates of UCN from single and multiphonon emission from three sources
% and comparison to Maxwellian
% spectrum} \label{tab:multiphonon} \end{center} \end{table} For the
% NC state proposed UCN source, where the helium is exposed to the
% total thermal flux, the inclusion of multiphonon contribution
% increases the UCN production rate by a little more than a factor of
% 2.

%%%%%%%%%%%%%%%%%%%%%%%%%%%%%%%%%%%%%%%%%%%%%%%%%%%%%%%%%%%%%%%%%%%%%%%%%%%%%%
%% I can get rid of this part
%%%%%%%%%%%%%%%%%%%%%%%%%%%%%%%%%%%%%%%%%%%%%%%%%%%%%%%%%%%%%%%%%%%%%%%%%%%%%%
UCN production by multiphonon excitation in superfluid helium under
pressure was studied in Ref.~\cite{Schmidt2009}.  The dynamic
scattering function $S(q,\omega)$ of the superfluid helium strongly
depends on pressure, leading to a pressure-dependent differential UCN
production rate. The expression for the multiphonon part of $S$
describing UCN production is derived from the inelastic neutron
scattering data.  Application of pressure to superfluid helium
increases the velocity of sound, such that the dispersion curves of
the $^4$He and of the free neutron cross at shorter neutron
wavelength.

Since for neutron beams from a liquid deuterium cold source, the
differential flux density $\frac{d\Phi}{dE}$ in the range
8-9~\AA~normally increases for decreasing wavelength of the cold
neutron flux, and also since pressure increases the density of He-II,
it was expected to observe an increase in the single phonon UCN
production rate, and different multiphonon contribution with pressure
increase.  It was observed that, both the single and the multiphonon
scattering functions change with pressure. The single phonon
excitation moves to a shorter wavelength~(see
Fig.~\ref{fig:Schmidt_S}) and the value for $S$ decreases. It leads to
a reduction in one-phonon UCN production.  The multiphonon excitations
increase with pressure, and the peak of the scattering function $S$
moves to shorter incident-neutron wavelengths, see
Fig.~\ref{fig:Schmidt_S}. However, the UCN production rate decreases
with pressure increase.  Only if the cold neutron flux at
8.3~\AA~exceeds by more than 2.5 times that at 8.9~\AA, an increase in
the UCN production rate may be expected. However, it has to be
considered that the application of pressure requires a window for UCN
extraction which causes severe UCN losses. Therefore, UCN production
in superfluid helium under pressure was concluded not to be
attractive~\cite{Schmidt2009}.




\begin{figure}[h!]
\begin{center}
   \includegraphics[width=0.7\textwidth]{Schmidt_S.PNG}
   \caption[Multiphonon scattering function of superfluid helium at
   differnet pressures]{~\cite{Schmidt2009} Multiphonon scattering
     function at SVP~(Saturated Vapour Pressure) and 20~bar. The
     extrapolation to short wavelength of Korobkina {\it {et
         al.}}~\cite{Korobkina2002} at SVP is linear in $k$, whereas
     the calculation of Schott {\it {et al.}}~\cite{Schott2003} is
     based on the static structure factor of the superfluid
     helium. The data point ($A$) is taken from
     Ref.~\cite{Fak1991}. The one-phonon peaks are indicated by
     vertical arrows: SVP~(dotted line) and 20~bar~(solid line).  }
%     \vspace{-2.em}
    \label{fig:Schmidt_S}
    \end{center}
\end{figure} 



\subsubsection{UCN upscattering and lifetime in superfluid
  helium~\label{sec:upscattering}}

%%%%%%%%%%%%%%%%%%%%%%%%%
%%% ADD THIS For a long UCN lifetime in superfluid helium, besides the
%%% low temperature of the converter, the $^3$He contamination must be
%%% low ($^3$He/$^4$He $\le 10^{-12}$), due to its large absorption
%%% cross-section, which requires $^4$He purification.
%%%%%%%%%%%%%%%%%%%%%%%%


Superfluid $^4$He has a zero neutron absorption cross-section, and if
the converter is kept at sufficiently low temperatures~(typically
$\lesssim$ 1~K), thermal upscattering of UCN is sufficiently
suppressed. This allows the produced UCN to survive in the converter
for times dominated by the wall losses of the vessel, typically
$>$100~s~\cite{Leung2016}.

The upscattering of neutrons is caused by the interactions between a
neutron at rest, and excitations in superfluid helium at different
temperatures. These excitations can be categorized in three groups:
one phonon absorption, two-phonon scattering, and roton-phonon
scattering. The total upscattering rate can be written as

\begin{equation}
\label{eqn:hE_{UCN}pscattering}
\frac{1}{\tau_{up}} =
\frac{1}{\tau_{1-ph}}+\frac{1}{\tau_{2-ph}}+\frac{1}{\tau_{rot-ph}},
\end{equation}
where 
%which are described by the following equation:
\begin{equation}
\label{eqn:1ph}
\frac{1}{\tau_{1-ph}}= A e^{-(12 K)/T}
\end{equation}
is the one phonon absorption contribution, 

\begin{equation}
\label{eqn:2ph}
\frac{1}{\tau_{2-ph}}= BT^7
\end{equation}
is the two-phonon scattering contribution~(one phonon absorbed and one
phonon emitted), and
\begin{equation}
\label{eqn:ph-rtn}
\frac{1}{\tau_{rot-ph}}= CT^{3/2}e^{-(8.6 K)/T}
\end{equation}
is the contribution from roton-phonon scattering with the absorption
of one roton followed by a phonon emission.

%The first term comes from one phonon absorption, the second term from
%two-phonon scattering where one phonon absorbed and one phonon
%emitted, and the third term from roton-phonon scattering which the
%absorption of one roton followed by a phonon emission.
The values of $A$, $B$ and $C$ are extracted from data for temperature
ranges of up to 2.4~K~\cite{Leung2016}. The comparison between the UCN
production and upscattering rate to the theoretical temperature
dependence of these processes showed that, the main contribution is
from two-phonon scattering $\frac{1}{\tau_{up}}=BT^7$ with
$B = 0.0076$~/(s K$^7$) at 1~K and $B = 0.0088$~/(s K$^7$) at
0.6~K\cite{Leung2016}.
%$B=(4-16)\times 10^{-3}$~/(s K$^7$)~\cite{Leung2016}.


%The most recent study of the upscattering rates in superfluid helium
%is done by Leung {\it{et al.}}~\cite{Leung2016}.  They calculated
%the upscattering rate due to each individual excitation in a
%temperature range of 0.5~K to 2.2~K as shown in
%Fig.~\ref{fig:Leung2016}. They calculated the $A$, $B$ and $C$
%coefficients for $T \lesssim 1$~K; however, their temperature
%dependencies are weak compared to the overall sizes of the terms.
%They varied the temperature of the converter from 1.2~K to 2.4~K and
%studied the UCN production and upscattering rates. They fit data to
%the theoretical temperature dependencies of these processes and they
%determine the values of $A$, $B$ and $C$. Their analysis showed that
%they only need to include two-phonon scattering meaning
%$\frac{1}{\tau_{up}}=BT^7$ with $B=(4-16)\times 10^{-3}$~/s K$^7$.



% \begin{figure}[h!]  \begin{center}
% \includegraphics[width=0.5\textwidth]{Leung2016.PNG} \caption{The
% sizes of the three main processes contribution to upscattering of
% UCN by excitations in superfluid % helium from
% Eqn.~(11-14)~\cite{Leung2016}. The thickness of the
% $\tau_{2-ph}^{-1}$ line covers the rnge of values for $T < 1$~K
% owing to the different values of B ($8.8 \times
% 10^{-3}$~s$^{-1}$K$^{-7}$ at 0.6~K and $7.6
% \times10^{-3}$~s$^{-1}$K$^{-7}$ at 1.0~K). The dotted lines are used
% to indicate that the upscattering rates are only approximate for $T
% \gtrsim 1$~K due to temperature dependencies in the $A$, $B$, and
% $C$
% coefficients.}  \vspace{-2.em} \label{fig:Leung2016} \end{center} \end{figure}






%\item[$\bullet$] UCN upscattering in superfluid helium
%(Kilvington1987, Leung2016, Maris1977)


%\subsubsection{UCN lifetime in superfluid $^4$He}

\subsection{UCN production by Solid Deuterium}
Solid deuterium~(sD$_2$) is a material with small capture
cross-section, small incoherent scattering cross-section~(to minimize
upscattering), and the presence of numerous phonon modes, which can
inelastically scatter neutrons down to UCN energies.  A converter
based on sD$_2$ should be operated at temperatures below 10~K in order
to avoid subsequent upscattering of UCN by phonons within solid
deuterium.

The different molecular species, ortho-D$_2$ and para-D$_2$, have
significantly different UCN-phonon annihilation
cross-sections~\cite{Liu2000, Morris2002}. The presence of even small
concentrations of para-D$_2$ can dominate the upscattering rate which
gives rise to reduced UCN lifetimes in the solid and orders of
magnitude reduction in the achievable UCN density.
%In a D$_2$ solid,
%the populations of ortho and para states are typically determined by
%the ortho/para population of the gas phase before the D$_2$ is frozen
%into solid.  After cooling down the D$_2$ to the solid phase
%($T \sim$~6~K), it normally takes months to reach the equilibrium of
%99.999\% o-D$_2$.
Since the elimination of para-D$_2$ is necessary to achieve UCN
lifetimes comparable to the nuclear absorption time in solid
deuterium, using a para-D$_2$ to ortho-D$_2$ converter is crucial.


Theoretically and experimentally, it has been shown that sD$_2$ at
sufficiently low temperatures~(around 5K) with high enough purity and
with high ortho concentration can be used to produce a high density
UCN~\cite{Atchison2005}.

%\subsubsection{UCN Production Cross-Section and UCN Production Rate in Solid Deuterium}

%The formula for the UCN production in solid deuterium is very similar
%to that of the superfluid helium shown in Eqn.~(\ref{UCN_production}),
%with replacing $^4$He atomic density $N_{\text{He}}$ with molecular
%density of solid deuterium $N_{D_2}$, and noting that in sD$_2$ the
%neutron scattering cross-section may be written as a sum of coherent
%and incoherent contributions~\cite{ucnbook,Frei2010,Frei2009}:

%\begin{equation}
%\label{eqn:dsigma}
%\frac{d\sigma}{d\omega}=\left[ \frac{k_2}{k_1} 
%b_{\text{coh}}^2 S_{\text{coh}} (q,\omega) + \frac{k_2}{k_1} b_{\text{inc}}^2 S_{\text{inc}}(q,\omega) \right]
% d\Omega.
% \end{equation}


%In Ref.~\cite{Frei2010} the UCN production cross-section $\sigma$
%was determined by two ways. One way is the determination of the
%quasi-particle~(phonons and rotational excitations of the D$_2$
%molecule) density of states $G_1(E)$~(incoherent approximation) from
%the measured neutron cross-section $\frac{d\sigma}{d\omega}$, and the
%other method is the direct integration of the dynamical neutron
%cross-section $\frac{d\sigma}{d\omega}$ ($\hbar=1$) in the kinematical
%region along the free-neutron dispersion parabola.

%\paragraph{UCN production cross-section: Incoherent approximation.}
%With the knowledge of the quasi-particle density of states $G_1(E)$,
%it is possible to calculate the dynamical neutron cross-section
%$\frac{d\sigma}{d\omega}$~(averaged over the scattering angle, thus
%$q$). Vice versa it is also possible to extract $G_1(E)$ from a
%measured dynamical neutron cross-section~\cite{Turchin}.  If
%$G_1(E)$ is known, it is possible to calculate one-phonon and
%multiphonon contributions to the neutron cross-section

%$\frac{d\sigma}{d\omega}$.



%The method for the determination of $G_1(E)$ from the measured neutron
%scattering data in solid deuterium is studied in
%Ref.~\cite{Frei2009}. In the determination of $G_1(E)$, contributions
%of higher order multiphonons to $\frac{d\sigma}{dE}$ are incorporated.

%In the case of UCN production, the energy transfer of the
%downscattered neutron $E=E_1-E_{\text{UCN}}$ is approximately equal to
%the initial neutron energy $E_1$
%($E_{\text{UCN}} \ll E_1, E_{\text{UCN}}$: UCN energy). The total
%cross-section for UCN production can be calculated by

%\begin{equation}
%\sigma_{\text{UCN}}({E_1})=\int_0 ^{E_{\text{UCN}}^{max}} \frac{d\sigma (E_1)}{dE} dE_{\text{UCN}}.
%\end{equation}

%The calculated cross-section shown in
%Fig.~\ref{fig:Frei2010_sigma_G1}, is in agreement with data on UCN
%production using a cold neutron beam~($E_1 \sim 1.4$~meV to 20~meV).
%Here the one-quasi-particle and two-quasi-particle excitations are
%included in the calculations.  The UCN production cross-section is
%mainly determined by one-quasi-particle excitation for energies below
%15~meV. The two-quasi-particle contribution is non-negligible in the
%region of 5-25~meV.

%The application of the incoherent approximation in the case of sD$_2$
%has certainly to be questioned since the sD$_2$ crystal scatters
%neutrons more coherently than incoherently.

%\begin{figure}[h!]
%\begin{center}
%   \includegraphics[width=0.5\textwidth]{Frei2010_sigma_G1.PNG} \caption{UCN
%    production cross-section of sD$_2$ with 98\% ortho
%    concentration. UCN energy range 0-150~neV inside the solid
%    D$_2$. Solid line: cross-section calculated in incoherent
%    approximation. Dashed line: one-quasi-particle
%    contribution. Dotted line: two-quasi-particle
%    contribution. $\square$: data from measurements at
%    PSI~\cite{Atchison2007}.  }
%     \vspace{-2.em}
%    \label{fig:Frei2010_sigma_G1}
%    \end{center}
%\end{figure} 




%\paragraph{UCN production cross-section: Direct determination.}
%The easiest way of determining the cross-section for UCN production is
%the use of the dynamical scattering function $S(q,\omega)$ in the
%($q,\omega$)-phase space along the free-neutron parabola, as shown
%schematically in Fig.~\ref{fig:sD2_S}.

%This method allows the incorporation of all the coherent and
%incoherent contributions to the UCN production cross-section. Possible
%coherent contributions, which cannot be treated exactly with the
%incoherent approximation, appear directly in the deduced
%cross-section. Therefore, this method is superior in principle to the
%result obtained by the incoherent approximation.

%\begin{figure}[h!]
%\begin{center}
%   \includegraphics[width=0.7\textwidth]{sD2_S.PNG} \caption{\cite{Frei2010}
%    $S(q,\omega)$ ($q=Q ,\omega=E$) (arb. units) of 95.2\% solid
%    o-D$_2$ at $T=4$~K. Data from IN4 measurements. Black parabola:
%    dispersion of the free neutron. }
%%     \vspace{-2.em}
%    \label{fig:sD2_S}
%    \end{center}
%\end{figure} 



%The UCN production cross-section can be determined by
%\begin{equation}
%\sigma_{\text{UCN}}(E_1)=\frac{\sigma_1}{k_1} S(k_1, E_1) \frac{2}{3} k_{\text{UCN}}^{max} \; E_{\text{UCN}}^{max} ,
%\end{equation}
%where $E_1$ is the energy of the incoming neutrons in the
%downscattering process, $\sigma_1$ is a constant, and
%$k_{\text{UCN}}^{max}$ and $E_{\text{UCN}}^{max} $ are the upper
%limits for the UCN momentum and energy.  In order to obtain absolute
%cross-sections, $S(q,\omega)$ has to be calibrated to absolute values.
%The result of this calibration and the determination of the UCN
%production cross-section as a function of the energy of the incoming
%neutrons, and a comparison with the measurements of this cross-section
%is shown in Fig.~\ref{fig:Frei2010fig2}. This plot also contains the
%data, which were obtained with higher incoming-neutron energy
%($E_1=67$~meV).


%\begin{figure}[h!]
%\begin{center}
%  \includegraphics[width=0.5\textwidth]{Frei2010_sigma.PNG} \caption{\cite{Frei2010}
%    UCN production cross-section solid o-D$_2$ of
%    95.2\%~\cite{Frei2010}. A UCN energy range of 0-150~neV inside the
%    solid D$_2$ is assumed. Sample: fast frozen solid deuterium
%    ($T=4$~K); data from IN4 measurements. Blue $\square$:
%    $E_0=17.2$~meV. Red filled $\bigcirc$: $E_0=67$~meV,
%    $\blacksquare$: direct UCN production data from measurements at
%    PSI~\cite{Atchison2007}.  }
%%     \vspace{-2.em}
 %   \label{fig:Frei2010fig2}
 %   \end{center}
%\end{figure} 

%The comparison of the calculated UCN production cross-section,
%extracted from the incoherent approximation and parabola method, shows
%(see Fig.~\ref{fig:Frei2010_sigma_G1} and Fig.~\ref{fig:Frei2010fig2})
%a discrepancy in the region of $E\sim6$~meV.  The cross-section
%determined by the parabola method shows a pronounced maximum in the
%region of $E\sim6$~meV as compared to the incoherent approximation
%result. This peak corresponds to the coherent phonon contribution to
%the UCN production cross-section. The double-peak structure in the UCN
%production cross-section by the incoherent approximation is not
%present in Fig.~\ref{fig:Frei2010fig2}, and cannot be reproduced by the
%measured data shown in Figs.~\ref{fig:Frei2010_sigma_G1}
%and \ref{fig:Frei2010fig2}.  This means, a new experiment at a more
%intense cold neutron beam with a better energy resolution would be
%desirable to study this effect further.

%In Fig.~\ref{fig:sD2_S}, the parabola of the free neutron crosses the
%acoustical phonon dispersion curve at $E \sim 6$~meV. At this point,
%the UCN production cross-section is predominantly determined by
%coherent scattering.  This can explain a deviation from the production
%cross-section in incoherent approximation. Nevertheless, the general
%agreement of the incoherent approximation with the PSI data is
%remarkable~(as shown in Fig.~\ref{fig:Frei2010_sigma_G1}).


%The result for the calculated UCN production rate in solid o-D$_2$,
%exposed to a Maxwellian shaped neutron flux for different effective
%neutron temperatures is shown in
%Fig.~\ref{fig:sD2_production_rate}. The main conclusion from these
%results was the new understanding of possible higher energetic loss
%channels~(one-quasi-particle and two-quasi-particle) in solid
%deuterium for the downscattering of cold neutrons in the conversion
%process to UCN. The best value for the effective neutron temperature
%is in the region of $T_n \sim 40$~K which is larger than what was
%previously expected~($T_n \sim 30$~K~\cite{Yu1986}).



%\begin{figure}[h!]
%\begin{center}
%  \includegraphics[width=0.5\textwidth]{Frei2010_P.PNG} \caption{\cite{Frei2010}
%    Calculated UCN production rate of sD$_2$ with 98\% ortho
%    concentration for different Maxwellian neutron spectra with
%    effective neutron temperature $T_n$. UCN energy range:0-150~neV
%    inside the sD$_2$. Neutron capture flux $10^{14}$~/cm$^2$s. Solid
%    line: total production rate~(one- and two- particle
%    excitations). Dashed line: one-particle production rate. Dotted
%    line: two-particle production rate.}
%%     \vspace{-2.em}
%    \label{fig:sD2_production_rate}
%    \end{center}
%\end{figure} 



\subsubsection{UCN upscattering and UCN lifetime in sD$_2$}




%\begin{figure}[h!]
%\begin{center}
%  \includegraphics[width=0.8\textwidth]{Morris2002.PNG} \caption{\cite{Morris2002}
%    Left- Data points are measured sD$_2$ lifetimes as a function of
%    temperature, with the para-fraction fixed at 2.5\%. Only the
%    statistical errors are shown. Solid lines show the predicted
%    temperature dependence. The dashed line is the predicted effect of
%    departure from the solid lifetime model due to the upscattering
%    from the D$_2$ gas in the guide. Right- sD$_2$ lifetimes as a
%    function of para-fraction for all of the data taken below 6~K. The
%    solid line is the model prediction of the para-fraction dependence
%    at an average temperature of 5.6~K.  }
%%     \vspace{-2.em}
%    \label{fig:Morris2002}
%    \end{center}
%\end{figure} 

The lifetime of the UCN in sD$_2$ is limited by factors such as
upscattering from phonons in the solid, upscattering from p-D$_2$
contamination, and absorption inside the vessel.  Reducing the time
UCN spend inside the sD$_2$ can reduce the average absorption
rate. This led to the proposal of a thin-film source where a thin
layer of solid D$_2$ coats the inside of a storage bottle that is
embedded in a cold neutron flux~\cite{Golub83}. The possibility of a
smaller source volume combined with the higher operating temperature
of the thin film source offers significant technical
simplification. In Ref.~\cite{anghel2018solid} losses in frost layers
on the surface of the deuterium crystal and the role of annealing have
been quantified.

The UCN lifetime in the solid deuterium as a function of the temperature
and para/ortho fractions has been measured~\cite{Morris2002}. The
total loss rate can be written as

\begin{equation}
\label{eqn:SD_lifetime}
\frac{1}{\tau_{SD}}=\frac{1}{\tau_{phonon}}+\frac{1}{\tau_{para}}+\frac{1}{\tau_{Dabs}}+ \frac{1}{\tau_{Habs}},
\end{equation}
where $\frac{1}{\tau_{phonon}}$ is the upscattering rate from phonons
in SD$_2$, $\frac{1}{\tau_{para}}$ is the upscattering rate from para
deuterium molecules in the solid, $\frac{1}{\tau_{Dabs}}$ is the
upscattering rate from the absorption on deuterium and
$\frac{1}{\tau_{Habs}}$ is the upscattering rate from the absorption
on the hydrogen impurities in the solid.



%The results for UCN lifetimes
%$\tau_{SD}$ in sD$_2$ as a function of the sD$_2$ temperature and
%para/ortho fractions are shown in Fig.~\ref{fig:Morris2002}. The
%difference between the solid and dashed line demonstrates the need to
%include the effect of deuterium vapour in the guide on the lifetime at
%higher temperatures. With this correction, the measured lifetimes
%agree well with theoretical predictions of the upscattering rate.


\subsection{Comparison between sD$_2$ and superfluid helium sources}

The main differences between sD$_2$ and superfluid helium sources are
the UCN lifetime and the UCN production rate. While UCN can stay in
superfluid helium until it $\beta$-decays, UCN in solid deuterium are
captured by the deuteron in 150~ms after they are produced.  Once a
superfluid helium source is cooled down to temperatures below 0.75~K,
the upscattering rate is suppressed to a level comparable to neutron
$\beta$-decay.  Solid deuterium has a production rate two orders of
magnitude greater than superfluid helium, due to presence of more
modes. Therefore, solid deuterium sources output higher UCN current
compared to superfluid helium sources. However, the limiting
production time in superfluid helium may be four orders of magnitude
longer than sD$_2$. Thus, even with a smaller UCN production rate,
superfluid $^4$He can in principle achieve a UCN density larger than
that of solid deuterium.  The superthermal enhancement in solid
deuterium is limited by the large nuclear absorption loss, and thus
further cooling below 5~K will not significantly enhance the UCN
yield.

Anghel {\it{et. al.,}}~\cite{anghel2018solid} showed that in addition
to the quality of the bulk sD$_2$, the quality of its surface is
essential for UCN yield. They showed that the surface is deteriorating
due to a build-up of D$_2$ frost-layers under pulsed operation which
leads to strong albedo reflections of UCN and subsequent loss.

%Both types of sources use quantum excitations in the converter medium
%to create the UCN; these are phonons in the case of sD$_2$ and
%phonons and rotons in the case of superfluid $^4$He. Since $^4$He
%does not capture neutrons and has a small upscattering probability
%for UCN, the superfluid $^4$He source can be operated at lower
%currents for longer times, allowing a large density of neutrons to
%accumulate. Storage times of hundreds of seconds are
%achievable. Solid deuterium sources, on the other hand, must pulse
%the beam, then quickly isolate any UCN produced from the sD$_2$,
%usually with a valve directly above the deuterium, because a UCN in
%sD$_2$ will only survive for tens of milliseconds.  The TRIUMF's
%technology is therefore complementary to spallation sD$_2$ projects.






%While the thin film source produces lower UCN densities than the
%helium-based source, the ability of this source to operate with
%shorter storage times without further loss of UCN density will allow
%either a large beam area or a smaller source volume or some
%combination.
\subsection{Other UCN Sources}
%~\cite{Salvat2013,Atchison2009,Liu_thesis}
Superthermal UCN sources may be compared by

\begin{equation}
\sigma_s / \sigma_a,
\end{equation}

where $\sigma_s$ is the elastic scattering cross-section, and
$\sigma_a$ is the absorption
cross-section~\cite{Salvat2013,Atchison2009,Liu_thesis}. At low
energies ($<$~1~eV) $\sigma_a \sim 1/v$ where $v$ is the speed of the
neutrons. This means, the absorption cross-section is much larger at
lower energies.  Table~\ref{tab:other_sources} shows a list of
possible superthermal UCN sources~\cite{Liu_thesis}. The values of
$\sigma_a$ are for thermal neutrons.


\begin{table}
\begin{center}
\begin{tabular}{|l|l|l|}
\hline
Isotope & $\sigma_a$(barns) & $\sigma_s / \sigma_a$  \\
\hline
$^2$D & 0.000519 & 1.47 $\times 10^4$ \\
\hline
$^4$He & 0 & $\infty$ \\
\hline
$^{15}$N & 0.000024 & 2.1 $\times 10^5$ \\
\hline
$^{16}$O & 0.00010 & 2.2 $\times 10^4$ \\
\hline
$^{208}$Pb & 0.00049 &  2.38 $\times 10^4$\\
\hline
\end{tabular}
\end{center}
\caption{Candidates for a superthermal source\label{tab:other_sources}}
\end{table}

Solid $\alpha - ^{15}$N$_2$ is a potential alternative to
deuterium~\cite{Salvat2013}. Its absoption cross-section is only 5\%
of that of D$_2$, and it has a negligible incoherent scattering
cross-section. Additionally, rotation of the N$_2$ molecules in the
lattice is inhibited due to the anisotropy of the N$_2$
inter-molecular potential. This leads to the dispersive modes for the
rotational degrees of freedom~(librons), which provide additional
channels for neutron downscattering, and eliminates the rotational
incoherent upscattering. Measurements~\cite{Salvat2013} show that, the
production cross-section peaks near 6~meV, and the optimal incident
cold neutron temperature is 40~K. It was found that, the variation in
the cross-section is no more than 18\% in the range from 5 to
25~K~(increasing slightly with increasing temperature). The measured
cross-section was found to be somewhat lower than that of D$_2$ and
O$_2$.
%However, it has a longer mean free path compared to deuterium.
A nitrogen-based source may benefit from operating at lower
temperatures, if the upscattering cross-section can be further reduced
at lower temperatures~($\sim$1~K)~\cite{Salvat2013}.


$^{208}$Pb and solid deuterium have similar nuclear absorption
cross-sections. The natural solid form of $^{208}$Pb would avoid the
difficulties of growing cryogenic solids such as deuterium and
oxygen. However, its heavy mass prevents the neutron momentum transfer
to the solid phonon field. The heavy mass reduces the phonon creation
cross-section by $1/M$. As a result, one would expect its UCN yield to
be two orders of magnitude less than solid deuterium.

As other options, the properties of the new candidate converter
materials including solid heavy methane~(CD$_4$) and solid
oxygen~(O$_2$) have been investigated in the temperature range 8~K to
room temperature by measuring the production of UCN from a cold
neutron beam and the cold neutron transmission through the converter
materials~\cite{Atchison2009}. The liquid O$_2$, D$_2$ and CD$_4$ have
similar neutron scattering cross-sections.

$^4$He and D$_2$ are still the best commonly pursued options, although
there is a chance that other materials could lead to a breakthrough.
% \item[$\bullet$] Investigationofsolid D$_2$, O$_2$ and CD$_4$ for
% UCN production (Atchison2009) \item[$\bullet$] Investigating solid
% $\alpha$-$^{15}$N$_2$ as a new source of ultra-cold neutrons
% (Salvat2013) \item[$\bullet$] solid $\alpha$-oxygen (Gutsmiedl2011,
% Liu thesis)



% \subsubsection{UCN Production Rate with Single Phonon scattering in
% Superfluid helium} Golub and Pendlebury 1975 and 1977

% \subsubsection{Multiphonon Scattering Contribution in UCN Production in Superfluid helium}
% Korobkina {\it{et al.}} and Schott {\it{et al.}}

% \subsubsection{Neutron Upscattering Rate in Superfluid helium}
% Leung {\it{et al.}} and Maris {\it{et al.}}


% \subsubsection{UCN Production Rate with SD$_2$ converter}



% \subsubsection{UCN Upscattering Rate in SD$_2$}
% Liu {\it{et al.}}



% \section{UCN Extraction}
% Cut if possible


\section{Current Status of UCN Sources Worldwide}
%\subsection {It needs to be modified}
New UCN sources using superthermal technology are under development at
various laboratories across the world. Neutrons are produced by two
methods: proton-induced spallation off a heavy nuclear target (e.g.,
tungsten), and fission where neutrons are produced by a nuclear
reactor. Table~\ref{tab:full_ucn_sources} shows a list of present and
future UCN sources worldwide.



\begin{table}[h!]
\begin{center}
\begin{tabular}{|l|l|l|l|}
\hline
Name & Source Type & Technology & Status \\
  \hline
  \hline
ILL  & Turbine & Reactor, CN beam & Running
%39 & few s 
\\
\hline
%J-PARC & Doppler SHifter & Spallation & Running
%\\
%\hline
ILL SUN-2 & LHe & Reactor, CN beam & Running 
%$\sim$15 peak (60~s,30~L) & 200(4~L, Fomblin grease, 80~neV)
\\
\hline
ILL SuperSUN & LHe & Reactor, CN beam & Future
%$\sim$150 peak (60~s, 30~L)& 800 %(12~L, 230~neV magnetic trap)
\\
\hline
%RCNP/KEK & LHe & Spallation & 26 & 81 (Ni) \\
%\hline
RCNP/TRIUMF/KEK & LHe & Spallation & Installing/Future
%600 polarized & 100 (NiP)
\\
\hline
PNPI Gatchina & LHe & Reactor & Future
%12000 & 10 (from He at 1.2~K)
\\
\hline
%ILL Turbine & sD$_2$ & Turbine & 39 & Few s \\
%\hline
LANL & sD$_2$ & Spallation & Running/Upgrading
%$\sim$25 polarized & 40 
\\
\hline
PSI & sD$_2$ & Spallation & Running
%Peak$\sim$23 & $\sim$90~s
\\
\hline
Mainz & sD$_2$ & Reactor & Running
%10 & Few s
\\
\hline
FRM II, Germany & sD$_2$ & Reactor & Future
%$\sim$5000 & Few s
\\
\hline
NCSU PULSTAR & sD$_2$ & Reactor & Installing
%$>$30 & Few s
  \\
  \hline
  SNS, Oakridge & LHe & Spallation & Future
  \\
  \hline
  J-PARC & Doppler Shifter & CN beam & Running
  \\
  \hline
\end{tabular}
\end{center}
\caption[UCN facilities worldwide]{Existing and future UCN sources
  worldwide. The existing or proposed sources at the following sites
  is listed: Institut Laue-Langevin~(ILL) in France, Reasearch Center
  for Nuclear Physics~(RCNP) in Japan, KEK and J-PARC in Japan, TRIUMF
  in Canada, Petersburg Nuclear Physics Institute~(PNPI) in Russia,
  Los Alamos National Lab~(LANL), PULSTAR and SNS in the US, Mainz and
  FRM II in Germany. }
\label{tab:full_ucn_sources}
\end{table}


Reactor sources place the moderators close to the reactor core~(FRM~II
and Gatchina~\cite{Serebrov_ascona}), or use existing CN beam
lines~(ILL~\cite{Piegsa2014}). At FRM~II, the sD$_2$ will be placed
around a solid hydrogen cold-moderator close to the fuel element. The
Gatchina superfluid $^4$He source will be placed inside their thermal
column, using immense pumping power to cool the converter to 1.1~K,
making rapid extraction necessary due to increased UCN upscattering at
this temperature.

The SuperSUN and SUN-2 experiments are the logical extensions of the
early superthermal source geometry at ILL.  A novel feature of the
SuperSUN experiment at ILL~\cite{Zimmer2015} is a magnetic multipole
reflector for a drastic enhancement of the UCN density with respect to
an existing prototype superfluid helium UCN source installed in a cold
neutron beam. A multipole magnet can lead to a large gain in the
saturated density of low-field-seeking UCNs because the presence of
the field reduces the number of neutrons hitting the material walls
and reduces the energy and wall collision rate of those that do. In
addition, it acts as a source-intrinsic UCN polarizer without need to
polarize the incident beam, and hence avoiding associated losses.
%This concept will lead to a drastic improvement of previous UCN
%storage time constants and hence provide a polarised UCN density
%beyond 1000 per cm$^3$.

The Los Alamos solid deuterium source~\cite{Ito_ascona} uses a
proton beam of 900~MeV and a W target to produce neutrons. The
neutrons get cooled down in a polyethylene cold moderator. The new
design includes a flapper valve to isolate the neutrons from the
sD$_2$ after the proton beam pulse.

The PSI UCN source~\cite{Ries_ascona} uses a 600~MeV proton beam
to hit a Pb/Zr target for neutron production. They use a 30~L volume
of sD$_2$ at 5~K as moderator and converter to produce UCN. This
volume is surrounded by D$_2$O thermal moderator. They also use a
flapper valve for UCN extraction between the proton beam pulses to
limit the losses. The UCN production has been running since 2012 with
an on-going EDM experiment, with a peak density of 23~UCN/cm$^3$.

The Mainz UCN source~\cite{Karch2014} is the only source that
operates at a low power university reactor, and is the newest
production source. The solid deuterium converter with a volume of $V =
160$~cm$^3$, which is exposed to a thermal neutron fluence of 4.5
$\times 10^{13}$~n/cm$^2$, delivers up to 240000 UCN~($v \leq$ 6~m/s)
per pulse outside the biological shield at the experimental area.  UCN
densities of $\approx$ 10/~cm$^3$ are obtained in stainless-steel
bottles of $V \approx$ 10~L. Their pulsed operation permits the
production of high densities for storage experiments.


At the SNS UCN source, the 8.9~$\AA$ cold neutrons are selected using
a monochromator, and are transported with neutron guides to two cells
made out of acrylic~(ultraviolet transmitting) and separated by a high
voltage electrode.  The neutrons entering the cell are
polarized. Within the cell, the cold neutrons become ultra-cold
neutrons via $^4$He single-phonon process~\cite{kolarkar2010}.


The UCN source at J-PARC is a doppler-shifter type of pulsed UCN
source~\cite{Imajo2015}. Very cold neutrons~(VCNs) with 136~m/s
velocity in a neutron beam supplied by a pulsed neutron source are
decelerated by reflection on a wide-band multilayer mirror, yielding
pulsed UCN. The mirror is fixed to the tip of a 2,000~rpm rotating arm
moving with 68~m/s velocity in the same direction as the VCN. The
repetition frequency of the pulsed UCN is 8.33~Hz and the time width
of the pulse at production is 4.4~ms. In order to increase the UCN
flux, a supermirror guide, wide-band monochromatic mirrors, focus
guides, and a UCN extraction guide have been newly installed or
improved. This source will be used to search for the nEDM.


The current UCN source at TRIUMF uses a W target to produce spallation
neutrons from a 500~MeV proton beam on site. The cold neutrons are
converted to UCN in superfluid helium.  The future UCN source is
projected to compete with the capabilities of the best planned future
UCN sources. If TRIUMF's estimated UCN density of 680 UCN cm$^{-3}$ is
achieved, it will be a new world record.

Other sources and nEDM experiments aim at similar goals of hundreds to
thousands of UCN cm$^{-3}$ in the measurement volume. However, to
date, superthermal sources have not produced considerably more UCN
than the ILL turbine source.


% Russ said: "This paragraph could be part of the conclusion.  The
% first paragraph should summarize what a superthermal source is and
% the main differences between He and sD2.  Then this paragraph.
 


%The neutrons are neutral particles, subjected to all fundamental
%forces which helps to investigate the physics beyond the standard
%model.



\section{Measurement of nEDM}

To measure the nEDM, an ensemble of polarized UCN are put in the
presence of aligned electric and magnetic fields. The Hamiltonian of
the interaction of the UCN with electric and magnetic fields is
described in Eqn.~(\ref{eqn:hamiltonian}). The Larmor precession
frequency of UCN is then measured in two oriantations of parallel and
anti-parallel electric and magnetic fields. For the parallel $\bf{E}$
and $\bf{B}$ fields, the Larmor precession frequency of UCN is written
as
\begin{equation}
\label{eqn:parallelEandB}
  h \nu_{\uparrow \uparrow} = 2 \mu_n \vert {\bf{B^{\uparrow \uparrow}}} \vert+ 2 d_n\vert \bf{ E^{\uparrow \uparrow}} \vert~,
\end{equation}
and for anti-parallel  $\bf{E}$ and $\bf{B}$ fields it is
\begin{equation}
\label{eqn:antiparallelEandB}
  h \nu_{\uparrow \downarrow} = 2 \mu_n \vert {\bf{B^{\uparrow \downarrow}}} \vert+ 2 d_n\vert \bf{ E^{\uparrow \downarrow}} \vert~.
\end{equation}
Here $\uparrow \uparrow$ indicates the parallel electric and magnetic
fields and $\uparrow \downarrow$ represent the anti-parallel
orientation of those fields.  A nonzero nEDM is then extracted from
any frequecy shift between these two measurements:
\begin{equation}
  \label{eqn:dn}
  d_n = \frac{h \left( \nu_{\uparrow \uparrow} - \nu_{\uparrow \downarrow} \right) - 2 \mu_n \left( \vert {\bf{B^{\uparrow \uparrow}}} \vert -\vert {\bf{B^{\uparrow \downarrow}}} \vert \right)}{2 \left(\vert \bf{ E^{\uparrow \uparrow}} \vert - \vert \bf{ E^{\uparrow \downarrow}} \vert \right)}~.
\end{equation}
The main reason to employ this method is because it is impossible to
completely eliminate the $\bf{B}$ field to extract the neutron
EDM. These measurements are either performed in two adjacent volumes
with
$\vert E^{\uparrow \uparrow} \vert = - \vert E^{\uparrow \downarrow}
\vert$, and
$\vert B^{\uparrow \uparrow}\vert - \vert B^{\uparrow \downarrow} \vert= 0
$, or measured in the same volume where the configuration of the fields
change in time. In the first case, it is essential to make sure that
the magnetic field inside both volumes are the same, and there is no
field gradient, and in the second method it is essential to make sure
that the magnetic field is stable in time.

\section{Ramsey's Method of Separated Oscillating
  Fields\label{sec:Ramsey}}

The Ramsey method of separated oscillating fields is the well-known
measurement technique to extract the nEDM. Ramsey obtained an
expression for the quantum mechanical transition probability of a
system between two states, when the system is subjected to separated
oscillating
fields~\cite{ramsey1950}. Fig.~\ref{fig:ramsey}~\cite{Schmidt2016}
left shows a cycle of measurement. An ensemble of polarized UCN with
the inital spin $\ket{ \uparrow}$ are exposed to a DC magnetic field
$B_0$.  A first RF pulse $B_1 \cos (\omega_{rf}t)$, prependicular to
the $B_0$ field, tips the spin of the neutrons to the transverse
plane. The neutrons precess freely with their Larmor precession
frequency $\omega_0$ for some time $T$, while accumulating a phase
$\phi = \gamma_n BT$. Then again, the second coherent oscillating
magnetic field pulse of $B_1 \cos (\omega_{rf}t)$ is applied to the
neutron ensemble. The essential idea is to compare the phase $\phi$
with $\omega_{rf}T$, and if they are identical then
$B= \omega_{rf} / \gamma_n$.

\begin{figure}[h]
  \centering
  \includegraphics[width=1.0\textwidth]{ramsey.png}
  \caption[Ramsey cycle]{\cite{Schmidt-Wellenburg:2016nfv} Ramsey
    method of separated oscillating fields. Left shows the scheme of a
    measurement procedure and right shows the data points. The blue
    points are the UCN counts with the spin up and the red points are
    the UCN with spin down~(data from the PSI-nEDM collaboration). The
    width at half height~$\Delta \nu$ of the central fringe is
    approximately $1/2T$, the four vertical lines indicate the working
    points~(see text).}
  \label{fig:ramsey}
\end{figure}

The probability to find the UCN with spin up is~(see full calculation
in Appendix~\ref{app:ramsey})
\begin{align}
  \label{eqn:spinup}
  P(T, \omega_{rf}) &= \bra{\uparrow} U(T, \omega_{rf}) \ket{\uparrow} \\ \nonumber
  &= 1 - \frac{4 \omega_1^2}{\Omega^2} \sin ^2 \frac{\Omega t_{\pi/2}}{2} \left[ \frac{\Delta}{\Omega} \sin  \frac{\Omega t_{\pi/2}}{2} \sin \frac{T\Delta}{2} - \cos  \frac{\Omega t_{\pi/2}}{2} \cos\frac{T\Delta}{2} \right]^2~,
\end{align}
where $U(T, \omega_{rf})$ is the time evolutions operator,
$\omega_1 = - \gamma_n B_1$, $\Delta = \omega_{rf} - \omega_0$,
$\omega_0 = \gamma B_0$, and $\Omega = \sqrt{\Delta^2 +
  \omega_1^2}$. When the spin-flipping pulses are optimized, we would
have $\gamma_n B_1 t_{\pi/2} = \pi / 2$. In this case, the central
fringe range ($\Delta \ll \omega_1$), and Eqn.~\ref{eqn:spinup}
simplifies to
\begin{equation}
  P(T, \omega_{rf}) = \frac{1}{2} \left( 1 - \cos(T\Delta) \right)~.
\end{equation}
In a real measurement with $N$ UCN inside a magnetic field region this
becomes
\begin{equation}
  \label{eqn:Nup}
  N^{\uparrow} = \frac{N}{2} \left\lbrace 1 - \alpha(T) \cos \left[ (\omega_{rf} - \omega_0 ) \cdot \left(T+\frac{T+4t_{\pi/2}}{\pi}\right)\right]\right\rbrace~,
\end{equation}
where $\alpha$ is the visibility of the central fringe with spin
either up or down
\begin{equation}
  \label{eqn:visibility}
  \alpha^{\uparrow /\downarrow} = \frac{N_{max}^{\uparrow /\downarrow} - N_{min}^{\uparrow /\downarrow}}{N_{max}^{\uparrow /\downarrow}+ N_{min}^{\uparrow /\downarrow}}~.
\end{equation}
The term $4t_{\pi/2}/\pi$ is necessary since the pulse lenght
$t_{\pi/2}$ is finite. The graph in Fig.~\ref{fig:ramsey} shows the
Ramsey interference pattern by scanning $\omega_{rf}$, while
everything else is kept the same~\cite{Schmidt-Wellenburg:2016nfv}. In
the actual nEDM measurements, only 4 points with the highest
sensitivity are measured. These points are refered to as the working
points. For each configuration of the electric and magnetic
fields~(parallel or anti-parallel), Eqn.~\ref{eqn:Nup} is fitted to
the data to extract the Larmor frequency $\omega_0$. Taking the
differences of those Larmor frequencies then give access to the nEDM
\begin{equation}
  \label{eqn:fitteddn}
  d_n = \frac{\hbar (\omega_0 ^{\uparrow \uparrow} - \omega_0 ^{\uparrow \downarrow})}{2(E^{\uparrow \uparrow} - E^{\uparrow \downarrow})} = \frac{\hbar \Delta \omega}{4E}~
\end{equation}
with the assumption that, the magnetic field is constant~(see
Eqn.~\ref{eqn:dn}).


\section{Statistical and Systematic Errors}
\subsection{Statistical Sensitivity}
The statistical sensitivity of the nEDM measurement is
\begin{equation}
  \label{eqn:dnsensitivity}
  \sigma(d_n) = \frac{\hbar}{2 \alpha T E \sqrt{N}}~,
\end{equation}
where visibility $\alpha$ is a factor related to the neutrons
polarization, $N$ is the number of detected UCN, $T$ is the free
precession time, and $E$ is the electric field. The visibility
$\alpha = e^{T/T_2}$ depends on the transverse spin relaxation time
$T_2$ respectively. The transverse spin relaxation time $T_2$ arises
from inhomogeneities in the magnetic field 
\begin{equation}
\frac{1}{T_2} = \frac{1}{T_2^{\prime}}+\frac{1}{T_1}~,
\end{equation}
where $T_1$ is the longitudinal relaxation time, and $T_2^{\prime}$ is
the transverse relaxation time only due to the field inhomogeneities.

\subsection{Systematic Errors}
The dominant systematic errors in the previous best experiment arose
due to magnetic field instability~(uncorrelated with the electric
field $E$), and magnetic field inhomogeneity which can give rise to a
false EDM for particle traps~\cite{pendlebury2004}.
% ~(seeAppendix~\ref{app:GPE}).
%The GPE arises due to a combination of magnetic field
%inhomogeneity and motion of the particles in the electric field during
%the measurement time, when the neutrons and co-magnetometer atoms are
%confined in the trap. The spins of the species in the trap acquire
%phases relative to one another resulting in a false EDM
%signal.
The false EDM could be understood by considering transverse fields
originating from the gradient of the uniform $B_0$ field in the axial
direction~($\partial{B_{0z}}/\partial{z}$), and the motion of UCN in
the electric field~($\bf{B_v} =(\bf{E} \times \bf{v})/c^2 $),
respectively. Radial fields like these rotate as the particle moves in
the EDM cell.  These fields rotate with the same frequency as
particles move in the EDM cell, thereby inducing a Bloch-Siegert shift
on the resonant frequency. The false EDM arises from a cross term
between the radial component of the applied $B_0$ field, and $B_v$ in
the Bloch-Siegert shift.

The false EDM could be corrected by the frequency ratio of the
neutrons to the co-magnetometer atoms~($^{199}$Hg) used to sense the
gradient. Graphing the measured neutron EDM as a function of this
ratio then allows to correct to the zero gradient and hence discover
the true neutron EDM. This is also supplemented by gradient
determination using surrounding Cs magnetometers.


Pendlebury {\it{et.~al.,}}~\cite{Pendlebury2015} measured the nEDM by
the Ramsey technique and using mercury as a comagnetometer to make
cycle-by-cycle corrections.  As they show, the measured nEDM has a
downward~(upward) slope as a function of the relative frequency shift
of neutrons and Hg when the applied $\bf{B_0}$ field is
upward~(downward),
\begin{equation}
  d_{\mathrm {meas}}= d_n \pm d_{\mathrm{false}}~.
\end{equation}
One would assume that the crossing point $d_x$ happens when there is
no magnetic field gradient. However, various effects could potentially
shift the crossing point.


Table~\ref{tab:nedmsystematics} shows a summary of the nEDM systematic
errors and their uncertainties for the Pendlebury
{\it{et.~al.,}}~\cite{Pendlebury2015} nEDM measurement. The
description of each effect is breifly listed below:

\begin{table}[h!]
  \begin{center}
    \begin{tabular}{|c||c|c|}
      \hline
      \bf{Effect} & \bf{Shift} & $\boldsymbol{\sigma}$ \\ \hline \hline
      $\nu_{\mathrm{Hg}}$ light shift (included in $d_x$) & (0.35) & 0.08 \\ \hline
      $\chi_{\nu}^2 = 1.2$ adjustment & 0 & 0.68 \\ \hline
      Quadrupole fields and Earth's rotation & 0.33 & 0.14 \\ \hline
      Dipole field & -0.71 & 0.07 \\ \hline
      Hg door PMD & 0.00 & 0.60 \\ \hline
      $\bf{v} \times \bf{E}$ translational & 0.000 & 0.001 \\\hline
      $\bf{v} \times \bf{E}$ rotational & 0.00 & 0.05 \\ \hline
      Second-order $\bf{v} \times \bf{E}$ & 0.000 & 0.000 \\ \hline
      Uncompensated $\bf{B}$ drift & 0.00 & 0.34 \\ \hline
      Hg atom EDM & -0.002 & 0.006 \\ \hline
      Electric forces & 0.00 & 0.04 \\ \hline
      Leakage currents & 0.00 & 0.01 \\ \hline
      AC fields & 0.000 & 0.001 \\ \hline
      Nonuniform Hg depolarization & 0.000 & 0.001 \\ \hline
      Total shift of $d_X$ & -0.38 & 0.99 \\ \hline
    \end{tabular}
  \end{center}
  \caption[Summary of systematic errors for the most recent nEDM
  measurement]{\cite{Pendlebury2015}Summary of systematic errors and
    their uncertainties, in units of $10^{-26}$~e$\cdot$cm. Correction
    for the mercury light shift is already incorporated run by run
    prior to the crossing-lines fit; other corrections are then
    applied to the crossing-point EDM value $d_x$ .}
  \label{tab:nedmsystematics}
\end{table}

\begin{description}
\item{ $\bullet$ \bf{$\boldsymbol{\nu_{\mathrm{Hg}}}$ light shift:}}
  The presence of mercury reading light can shift the frequency of the
  mercury atoms.
  
\item{$\bullet$\bf{$\boldsymbol{\chi_{\nu}^2 = 1.2}$ adjustment:}}
  Over the four years of data taking, the profile of the magnetic
  field could in principle change due to opening and closing the
  magnetic shields and their demagnetization. These random variations
  are taken into account by increasing the uncertainty of the global
  fit $\chi^2_{\nu}$.
  
\item{$\bullet$ \bf{Quadrupole fields and Earth's rotation:}} In the
  presence of a field with gradient in the $x$ and $y$ direction only,
  the neutron spins follow the total field direction adiabatically
  while the nonadiabatic Hg atoms stay insensitive to such
  changes. This affect the neutron to Hg frequency ratio, which in
  turn causes a false nEDM. The rotation of the Earth shifts all of
  the frequency ratio measurements of neutron and Hg to lower values
  when the $\bf{B_0}$ field is upwards, and to higher values by the
  same amount when the $\bf{B_0}$ field is downwards.
  
\item{$\bullet$\bf{Dipole field:}} The field of a permanent magnetic
  dipole close near the UCN storage results in a field gradient which
  can induce a false nEDM.

\item{$\bullet$\bf{Hg door PMD:}} This is the effect of the permanent
  magnetic dipole~(PMD) near the Hg door which induces a false nEDM.
  
\item{$\bullet$$\bf{v}\times\bf{E}$ \bf{effects}:} A false nEDM signel
  arises when UCN experiences a magnetic field gradient in the
  presence of the electric field as it.

\item{$\bullet$\bf{Uncompensated
      $\bf{B}$ drift:}} In principle there could be some residual
  effects due to the fluctuations in the magnetic field, such as
  hysteresis in the mu-metal shield. This could eventually induce a
  false EDM.

\item{$\bullet$\bf{Hg atom EDM:}} Since the mercury is used to
  compensate for shifts in the magnetic field, any EDM-like component
  contributing to the mercury frequency will affect the measurement of
  the neutron EDM.

\item{$\bullet$\bf{Electric forces:}} An electrostatic force can
  slightly move the electrodes of the high voltage EDM cell which can
  cause a systematic uncertainty.

\item{$\bullet$\bf{Leakage currents:}} The leakage current that flows
  through or along the surface of the insulator between the high
  voltage electordes with a component along the $\bf{B_0}$ field would
  produce a frequency shift that gives rise to a false nEDM.

\item{$\bullet$ \bf{AC fields:}} An oscillation in the high voltage
  can generate an oscillating displacement current in the nEDM cell
  which causes an oscillating $\bf{B}$ field. This can make a change
  in the frequency shift and therefore the nEDM.

\item{$\bullet$\bf{Nonuniform Hg depolarization:}} If the average
  depolarization time of the Hg atoms are different for two
  orientations of the $\bf{E}$ field, or if the Hg frequency has some
  dependencies to the depolarization time, a false EDM would be
  induced.

\end{description}




%This method can be calibrated for larger applied
%gradient fields. However, If the neutron resonant frequency is
%corrected by the co-magnetometer resonant frequency in the usual way,
%this results to false EDM for the neutrons induced by the
%co-magnetometer atoms. This type of false EDM could be reduced by
%using the buffer-gas effects to limit the mean free path $\lambda$ of
%the co-magnetometer atoms. If the mean free path of the atoms is
%limited, with interparticle collision times becoming small, a
%suppression factor may reduce the false EDM. If no buffer gas effect
%can be obtained for Xe, the simultaneous introduction of Hg and Xe
%atoms into the vessel can cancel out the GPE~(dual co-magnetometer).

%%%%%%%%%%%%%%%%%%%%%%%%%%%%%%%%%%%%%%%%%%%%%%%
% To cite this article: A P Serebrov 2017 J. Phys.: Conf. Ser. 798 012206
% for the nedm history of measurements
% saved it on my laptop
%%%%%%%%%%%%%%%%%%%%%%%%%%%%%%%%%%%%%%%%%%%%%%%%
\section{nEDM Status Worldwide}
In 1950, the first upper limit on neutron EDM was discussed by Purcell
and Ramsey to be $3 \times 10^{-18}$~\cite{PhysRev.78.807}. Since then
many groups around the world attempted to measure the nEDM and
increase its sensitivity~(see Fig.~\ref{fig:nEDMhistory}).

The most recent nEDM measurement at ILL found that to be
$d_n< 3.0 \times 10^{-26}$~e$\cdot$cm (90\%
CL)~\cite{pendlebury2015revised}. The new $^{199}$Hg EDM measurement
constrains the nEDM better than direct nEDM measurements,
$d_n < 1.6 \times 10^{-26}$~e$\cdot$cm, although subject to
uncertainty from Schiff screening~\cite{graner2016reduced}.

There are several ongoing experiments seeking to measure the
nEDM. Most groups are aiming initially for an improvement of the
uncertainty on $d_n$ to the $10^{-27}$~e$\cdot$cm level, ultimately
improving to the $10^{-28}$~e$\cdot$cm level over time.


\begin{figure}[h!]
  \centering
  \includegraphics[width=0.9\textwidth]{nedm_vs_year.png}
  \caption[History of nEDM measurement]{The timeline of the upper
    bound on the neutron EDM from previous and future experiments. The
    yellow squares are the previous bounds, and the blue dot is the
    sensitivity target of next generation experiments.  In the
    standard model, the expected size of nEDM is more than 5 orders of
    magnitude below the current experimental bound, as shown by the
    green shaded area in the plot. In extensions of the standard
    models, however, the expected size of nEDM covers the regions of
    the current and future experimental bound, as shown by the red
    shaded area.~\cite{yoon2018neutron} }
  \label{fig:nEDMhistory}
\end{figure}

The PSI nEDM experiment uses an improved version of the former Sussex-
RAL-ILL single-cell apparatus. Several innovations have been made at
PSI, including a new SD$_2$ spallation-driven UCN source. The
experiment employs several Cs magnetometers outside the EDM cell, and
a $^{199}$Hg comagnetometer. Active magnetic shielding and other
environmental controls have been improved. A new detector that can
simultaneously count both spin states of UCN has also been
implemented. The final sensitivity expected is
$\simeq 10^{-26}$~e$\cdot$cm~\cite{Schmidt2016}. Some of the chief
improvements made at PSI recently have been in the area of nearby
alkali atom~(Cs) magnetometry, Hg comagnetometry, and neutron
magnetometry. A recent achievement at PSI is the understanding of the
Cs magnetometer signals in terms of magnetic field gradients internal
to the magnetic shielding. This has led to a detailed understanding of
the false EDM of the Hg comagnetometer~\cite{Afach2015_2}. Another
recent achievement is in using the neutrons themselves to measure
gradients~\cite{Afach2015_3}. PSI also aims to improve their
magnetometry with $^3$He magnetometers inside the electrodes of the
double EDM measurement cells for their future n2EDM effort. They have
performed R\&D using Cs magnetometers to sense the free-induction
decay signal from $^3$He, which resulted in a new high-precision
magnetometer possessing excellent long-term
stability~\cite{Koch2015}. The precision goal for n2EDM is
$5 \times 10^{-28}$~e$\cdot$ cm~\cite{Bernhard_talk,baker2011search}.


The nEDM collaboration at SNS plans to measure
$\ d_n\approx 2 \times 10^{-28} $~e$\cdot$cm, two orders of magnitude
improvement from the current limit~\cite{peng2008neutron}.  They plan
to use a unique experimental technique. A Cold Neutron~(CN) beam from
the SNS will impinge upon a volume of superfluid $^4$He creating
UCN. The nEDM measurement will also be conducted in the superfluid. A
small amount of polarized $^3$He introduced into the superfluid $^4$He
will act as both a comagnetometer and spin analyzer for the UCN. The
$^3$He neutron capture rate is strongly spin dependent, and will beat
at the difference of the Larmor precession frequencies of the neutrons
and $^3$He. A non-zero EDM would change the beat frequency with
E-reversal. Scintillation light produced in the superfluid will be
used to detect the capture products. The target precision is
$10^{-28}$~e$\cdot$cm. The false EDM of the $^3$He comagnetometer may
be reduced by collisions in the surrounding
$^4$He~\cite{LamGol2005}. The group aims to commission the experiment
at SNS by 2020.


A new room-temperature nEDM experiment will be conducted using an
upgraded LANL UCN source~\cite{Steven_talk}. The aim of the project is
to increase the UCN density by a factor of five to ten, which could
then be used to carry out a $\sim 10^{-27}$~ e$\cdot$cm determination
of the nEDM.  The experiment aims for completion of a $10^{-27}$-level
result, to be completed in the years prior to the SNS nEDM experiment,
which shares a number of collaborators.  Two other room temperature
nEDM experiments are being pursued at the FRM2 reactor in
Munich~\cite{altarev2012} and at ILL~\cite{Serebrov2015}. Both
experiments feature double measurement cells and Cs magnetometers
internal to the innermost magnetic shield. The Munich effort features
an impressive new effort in active and passive magnetic
shielding~\cite{altarev2014magnetically,
  altarev2015large,altarev2015minimizing,altarev2012next}, and uses
$^{199}$Hg comagnetometry. The ILL/Gatchina experiment has produced
results at ILL~\cite{Serebrov2015}. This could be improved in further
runs at ILL in the EDM position, or in runs using the superfluid He
UCN source at ILL, where a statistical sensitivity of
$3.5 \times 10^{-27}$~e$\cdot$cm could be
obtained~\cite{Serebrov_talk}. The group will build a UCN source at
the WWR-M reactor in Gatchina in order to increase the UCN flux.




















 
% \begin{table}
% \begin{center}
% \begin{tabular}{|l|l|l|l|
% }
% \hline 
% Name & Technology  &  Storage time (s) & Density in cell \\ 
% \hline 
% ILL SUN-2 & - & 200 (4~L, Fomblin grease, 80~neV) & $\sim$15 peak (60~s,30~l) \\ 
% \hline 
% \textbf{SuperSUN} & - & 800 (12~L, 230~neV magnetic trap)& $\sim$150 peak (60s,30l) \\ 
% \hline 
% RCNP/KEK & spallation-based  & 81 (Ni)& 26 \\ 
% \hline 
% \textbf{TRIUMF/KEK} & spallation-based & 100 (NiP) & 600 polarized \\ 
% \hline 
% PNPI & reactor-based & 10 (from He at 1.2~K)& 12000 \\ 
% \hline 
% \end{tabular} 
% \label{tab:he}
% \end{center}
% \caption{Current status of the ongoing and future superfluid
% $^4$He-based UCN sources worldwide.}
% \end{table}

% \begin{table}
% \begin{center}
% \begin{tabular}{|c|c|c|c|}
% \hline 
% Name & Technology  &  Storage time (s)& Density in cell \\ 
% \hline 
% LANL & spallation-based & 40 &\\ 
% \hline 
% PSI & spallation-based & $\sim$90 &  \\ 
% \hline 
% Mainz & reactor-based & Few s & \\ 
% \hline 
% \textbf{FRM II} & reactor-based & Few s &  \\ 
% \hline 
% \textbf{PULSTAR} & -  & Few s & \\ 
% \hline 
% \end{tabular} 
% \end{center}
% \caption{blah}
% \label{tab:sd2}
%s \end{table}





% \item[$\bullet$] RCNP Osaka (Masuda2012)
% \item[$\bullet$] TRIUMF (Proposal)
% \item[$\bullet$] PSI source(Anghel2009)
% \item[$\bullet$] Los Alamos source (Saunders2013)
% \item[$\bullet$] ILL (Leung2016)


% \begin{table}
% \begin{center}
% \begin{tabular}{|l|l|l|}\hline
%  Location & Technology & Comments, program \\\hline\hline
% RCNP Osaka & spallation $^4$He & running/upgrading (-2015), nEDM \\
% TRIUMF & spallation $^4$He   & future (2016-), nEDM, other experiments\\\hline
% PSI    & spallation SD$_2$   & running, nEDM\\
% LANL   & spallation SD$_2$   & running, beta-decay, lifetime, nEDM\\
% ILL Grenoble (SuperSUN) & CN beam $^4$He (Zimmer) & running/upgrading, GRANIT, Gatchina-nEDM\\
% SNS ORNL & CN beam $^4$He     & future, cryogenic nEDM\\
% Munich & reactor SD$_2$      & future, nEDM, beta-decay, others\\
% Mainz  & reactor SD$_2$      & running, beta-decay, others\\ 
% NCSU (PULSTAR)  & reactor SD$_2$      & future, beta-decay, $nbar{n}$, others\\
% Gatchina & reactor $^4$He    & future, nEDM, others\\\hline
% \end{tabular}
% \end{center}
% \caption{Existing and future superthermal UCN sources worldwide. The
% existing or proposed sources at the following sites is listed:
% Paul-Scherrer Institut (PSI), Los Alamos National Lab (LANL),
% Institut Laue-Langevin (ILL), the Spallation Neutron Source (SNS) at
% Oak Ridge National Lab (ORNL), the Munich Forschungs-Neutronenquelle
% Heinz Maier-Leibnitz (FRM II), the Mainz TRIGA reactor (Training,
% Research, Isotope Production, General Atomics), the NCSU PULSTAR
% reactor, and the Gatchina WWR-M reactor. Other sources feature
% spallation- or reactor-based sources using solid deuterium (SD$_2$)
% or reactor-based sources using superfluid $^4$He.  The RCNP/TRIUMF
% effort is the only UCN source in the world to couple a $^4$He
% production volume to a proton-induced spallation
% target.\label{tab:ucnsources}}
% \end{table}

% \end{description}






\section{Summary}
Precision experiments involving the UCN provide an attractive avenue
to investigate physics beyond the standard model. Measurement of the
neutron EDM is an example of such experiments. For such studies high
densities of UCN are needed.

UCN are very slow neutrons with velocities $<8$~m/s that can be
trapped in matter, magnetic and gravitational fields.  Superthermal
UCN sources could produce high densities of UCN. Such sources should
have a very small neutron absorption cross-section and upscattering
rate, while having a high UCN production rate. So far, the best
candidates are superfluid helium and solid deuterium.
%The UCN production rate, the upscattering rate and the UCN lifetime
%of $^4$He and sD$_2$ are discussed in detail.

Both $^4$He and solid D$_2$ UCN sources use quantum excitations in the
converter medium to create the UCN; these are phonons in the case of
superfluid sD$_2$ and phonons and rotons in the case of $^4$He. Since
$^4$He does not capture neutrons, and has a small upscattering
probability for UCN, the superfluid $^4$He source can be operated at
lower currents for longer times, allowing a large density of neutrons
to accumulate. In the case of superfluid helium, the storage times of
hundreds of seconds are achievable. The production rate in sD$_2$ is
higher than in supefluid $^4$He, however, the neutron storage lifetime is
only tens of milliseconds.

The TRIUMF UCN project is the only spallation-driven superfluid-$^4$He
source proposed at this time in the world~\cite{Ruediger}. The
spallation-driven UCN sources at PSI~\cite{Ries_ascona} and
LANL~\cite{Ito_ascona} use the phonons in solid deuterium as an
alternative method of UCN production.
%The production rate in D$_2$ is higher than in superfluid $^4$He, but
%the neutron storage lifetime of the latter is much longer (hundreds
%of seconds compared to milliseconds) if the phonon density is
%suppressed by cooling to $<$ 1~K.
The TRIUMF's UCN source uses an optimum proton beam structure on the
minute scale to produce the highest density of UCN in the world, while
sD$_2$ spallation sources benefit from pulsing the beam, then isolate
any UCN produced as quickly as possible to achieve high UCN densities.
The detail of the current UCN facility at TRIUMF is presented in
Chapter~\ref{chap:UCNattriumf} and the result of the first UCN
production with the vertical UCN source is discussed in
Chapter~\ref{chap:UCNresult}.

The neutron EDM is measured using the Ramsey method of separated
oscillatory fields. In this technique, polarized UCN are trapped in a
material bottle in the presence of aligned electric and magnetic
fields. The Larmor precision frequency of UCN is measured when the $E$
and $B$ fields are parallel and once when they are anti-parallel. Any
frequency shift in the Larmor frequency of UCN is then an indicative
of a non-zero nEDM. The components of the future nEDM measurement at
TRIUMF are described in Chapter~\ref{chap:nedm}.
