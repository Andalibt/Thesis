\chapter{Introduction}
\renewcommand{\thepage}{\arabic{page}}% Arabic numerals for page
                                      % counter
\setcounter{page}{1}% Start page number with 1



This thesis is focused on two important factors of successfully
measuring the neutron Electric Dipole Moment~(nEDM) at TRIUMF. Those
include having a very stable magnetic field environment as well as a
very high density Ultra Cold Neutron~(UCN) source.

For the future nEDM measurement at triumf several types of shielding
will be used.  Magnetic shielding is vital to create a quiet magnetic
environment. An important component of the magnetic shielding system
is the large magnetic shields that are made of highly permeable
materials. A prototype of such shields exists at the University of
Winnipeg. the first half of the thesis is related on magnetic field
stability studies on the prototype shields at the University of
Winnipeg.

The The second half of the thesis is dedicated to the UCN production
and analysis at TRIUMF using the vertical UCN cryostat which was
previously used at RCNP ~(reference).
 



\section{History of Fundamental Symmetries }

Over the last few decades the interest in the invariance of the
discrete symmetries have been increased. Such studies revealed the
internal structure of the elementary particles and helped develop the
underlying theories.

There are three significant symmetries in physics as Charge
conjugation~($C$), Parity~($P$) and Time-reversal~($T$). $C$-symmetry simply
decribes physical laws under a charge-conjugation
transformation. Parity transformation, is simply the inversion of
spatial coordinates and Time-reversal transformation is changing the
direction of time.
Tests of Charge $C$,$P$ and $T$ symmetries established the structure
of the Standard Model~(SM)~\cite{pospelov2005electric}.  \\

In 1956, fall of discrete symmetries started with the famous
$\theta-\tau$ paradox in the K-mesons decay. Yang and Lee suggested
that the paradox is originated from a $P$ violation in the weak
interactions~\cite{lee1957parity}. Immediately after, an experimental
search was suggested by Ramsey for Parity violation in the $\beta$
decay of Co-60. Within a few months, $P$ violation was demonstrated by
three different experiments
~\cite{PhysRev.105.1413,PhysRev.105.1415,friedman1957nuclear}.  After
the observation of $P$ violation, Landau showed that Electric Dipole
Moments~(EDMs) are forbidden by $T$
symmetry~\cite{landau1957conservation} and then it was suggested that
$T$ symmetry should also be checked experimentally
\cite{PhysRev.106.517}.  \\
%In 1964 it was discovered that the $C$ and $P$ symmetries are broken in
%the $K$-meson decay~\cite{christenson1965regeneration}. 

One of the most fundamental symmetries in physics is the
$CPT$~(Charge-Parity-Time) symmetry. The simultaneous operation of of
$C$, $P$ and $T$ leaves the system unchanged. To date there is no
experimental evindence for $CPT$ symmetry breaking.  Because of the
$CPT$ invariance, breakdown of $CP$ symmetry should be accompanied by
violation of Time-reversal symmetry.  \\

A finite EDM provides a good source of $CP$ violation. EDMs caused by
$CP$ violation in the SM are negligible. But most extensions of SM
such as supersymmetry naturally produces EDMs that are comparable to
or larger than present experimental limits~\cite{romalis2001new}.  The
search for EDMs can be traced back to 1950 when Purcell and Ramsey
tested the possibility of finding EDMs for particles and
nuclei. Smith, Purcell and Ramsey started an experiment to search for
neutron EDM $d_n$ and they achieved the upper limit of $d_n < 5 \times
10^{-20}$~e $\cdot$ cm~\cite{smith1957experimental}.  Over the years
the upper limit on the neutron EDM has been improved by many orders of
magnitude. Measurement of particle EDMs provide some of the tightest
constraints on extensions to the SM to probe $CP$ violation. The most
recent upper limit on the neutron EDM is found to be $|d_n| < 3.0
\times 10^{-26} $~e$\cdot$ cm~\cite{pendlebury2015revised}.

\section{Neutron Electric Dipole Moment}

A permanent neutron electric dipole moment is an intrinsic property of
a neutron. This fundamental property is a measure for the separation
of positive and negative charges internal to the neutron. However no
nEDM has been measured so far.

The interaction of a nonrelativistic neutron with
the electromagnetic field can be descibed by the follwoing
hamiltonian:

\begin{equation}
  \label{hamiltonian}
 \begin{split}
 H&= -\boldsymbol{\mu_n} \cdot \textbf{B} - \bf{d}_n \cdot \textbf{E}
 \end{split}
 \end{equation}
where $\boldsymbol{\mu}_n$ is the magnetic moment of the neutron
interacting with the magnetc field \textbf{B} and $\bf{d}_n$ is
the electric dipole moment of the neutron interacting with the
electric field \textbf{E}.

The properties of the Hamiltonian under discrete symmetries is
summarized in table.~\ref{Hsymmetry}. Based on this, the first term is
$CP$-even and $T$-even and the second term is $cp$-odd and $T$-odd
where both terms are $CPT$-invariant. Therefore, a nonzero EDM may
exists if both Parity and Time-reversal symmetries are broken.


\begin{table}
  \label{Hsymmetry}
\begin{center}
\begin{tabular}{| l | l | l | l |} 
\hline
 & C & P & T \\ \hline
\textbf{B} & - &+ &- \\ \hline
\textbf{E} & -&- &+ \\ \hline
$\boldsymbol{\mu}$ &- &+ &- \\ \hline 
\textbf{d} & -&+ &- \\ \hline
\end{tabular}
\caption{Symmetry properties of different components of the EDM Hamiltonian}
\end{center}
\end{table}
  


\subsection{Baryon Asymmetry of the Universe}
The neutron EDM provides a highly sensitive diagnostic for $CP$
violation which is an important element for the observed
baryon asymmetry in the universe.  The dominance of matter over
antimatter in the universe can be characterized by~\cite{Cline}
\begin{equation}
\eta = \frac{n_b-\bar{n_b}}{n_{\gamma}} \simeq 6 \times 10^{10}
\end{equation}
where $n_b$ is the number of baryons, $\bar{n_b}$ is the number of
anti-baryons and $n_{\gamma}$ is the number of photons in the Cosmic
Microwave Backgorund.

It is possible to assume that maybe the universe is baryon symmetric
in a very large scale and it is split into regions that are made of
only baryons or anti-baryons. If that was the case, an excess of gamma
rays in between these separated regions was expected due to
annihilation. But, even in the least dense regions of the space,
there is hydrogen gas cloud.  \\

\subsubsection{Sakharov criteria}
There are three ingredients needed to
create baryon asymmetry known as Sakharov conditions~\cite{Sakharov:1967dj}:
\begin{center}
\begin{description}
\item[$\bullet$]Baryon number violation
\item[$\bullet$] $C$(Charge) and $CP$(Charge-Parity) violation
\item[$\bullet$] Departure from thermal equilibrium.
\end{description}
\end{center}

The first condition is obvious which means in a reaction, if the net
baryon number is zero, there would be no baryon asymmetry. In the
reactions that violate baryon number, if there is no $C$ and $CP$
violation, the net baryon number would be zero~\cite{theearlyuniverse}.
The third condition is essential for a net nonzero baryon asymmetry
since the equilibrium average of $B$ vanishes. Sakharov suggested that
baryogenesis took place immediately after the big bang, at a
temperature not far below the Planck scale of $10^{19}$ GeV, when the
universe was expanding so rapidly that many processes were out of
thermal equilibrium~\cite{cohen1993progress}.


%\subsection{Physics Beyond the Standard Model}

\subsection{The nEDM Measurement Technique}

\subsection{neutron Electric Dipole Moment Status Worldwide}

The most recent neutron electric dipole moment measurement at ILL
found that be $d_n< 3.0 \times 10^{-26}$ e$\cdot$cm (90\% CL)
\cite{pendlebury2015revised}. The new $^{199}$Hg EDM
measurement constrains the nEDM better than direct nEDM measurements,
$d_n < 1.6 \times 10^{-26}$ e$\cdot$cm although subject to uncertainty
from Schiff screening~\cite{graner2016reduced}.\\

There are several ongoing experiments seeking to measure the
nEDM. Most groups are aiming initially for an improvement of the
uncertainty on $d_n$ to the $10^{-27}$ e$\cdot$cm level, ultimately
improving to the $10^{-28}$ e$\cdot$cm level over time.

The PSI nEDM measurement aims for a measurement at the $5 \times
10^{-28}$ e$\cdot$cm level~\cite{baker2011search}.  (Add some
  detail about how they are going to measure it, what their technique
  is, reactor or spallation, solid deuterium or superfluid helium)

The nEDM collaboration at SNS plans to measure $\ d_n\approx 2 \times
10^{-28} $ e$\cdot$cm, two orders of magnitude improvement from the
current limit~\cite{peng2008neutron}.  (Add some detail about how
  they are going to measure it, what their technique is, reactor or
  spallation, solid deuterium or superfluid helium)

The room temperature nEDM measurement at Munich also aims for nEDM
measurement of $10^{-27}$ e$\cdot$cm level, and is a world leader in
active and passive magnetic
shielding~\cite{altarev2014magnetically,altarev2015large,altarev2015minimizing,altarev2012next}.
(Add some detail about how they are going to measure it, what their
technique is, reactor or spallation, solid deuterium or superfluid
helium)

(I am sure there are other places as well. They should be included here).

The nEDM experiment at TRIUMF is aiming for a determination of nEDM at
the $10^{-27}$ e$\cdot$cm level. A key
factor for success is a unique high density ultracold neutron (UCN)
source.
The TRIUMF's approach is unique in a
sense that it is the only place that is comibining a spallation method
with superfluid helium for UCN production.
Another novel feature is a dual comagnetometer ($^{129}$Xe and
$^{199}$Hg) to characterize systematic errors.


\section{Ultracold Neutrons}
(This section needs to be expanded)
Ultracold neutrons (UCN) are neutrons with kinetic energy $\lesssim
300$ neV corresponding to velocity~$\lesssim 8$ m/s or temperatures
$\lesssim 3$ mK. Because of their low energy, UCN can be reflected
from many materials under arbitrary angles of incident and therefore,
it makes UCN storable in a material bottle. UCN are subjected to all
four fundamental forces in the following
ways~\cite{beatrice,knecht,golub1994neutron,golub1991ultra}:

\subsection{Ultracold Neutrons Interaction with Fundamental Forces}

\subsubsection{The Gravitational Interaction:}
The interaction of UCN with the earth's gravitation field is described by
\begin{equation}
V_g=mgh
\end{equation}
where
\begin{equation}
mg=102\; \text{neV/m}
\end{equation}
which is comparable to the UCN kinetic energy. This means a UCN of
energy 200~neV can rise by at most 2~m.

 \subsubsection{The Weak Interaction:}
UCN decay via
\label{neutrondecay}
\begin{equation}
n\longrightarrow p+e^{-}+\bar{\nu_{e}}.
\end{equation}
with a lifetime of $880.3\pm 1.1$ S~\cite{PDG}. UCN can be bottled for
times comparable to this time.

\subsubsection{The Electromagnetic Interaction:} Although a neutron is
electrically neutral, it possesses a magnetic dipole moment which
interacts with a magnetic field \textbf{B} by the interaction
\begin{equation}
V_m=-\boldsymbol{\mu}_n \cdot \textbf{B}
\end{equation}
where
\begin{equation}
\vert \boldsymbol{\mu}_n \vert =60 \; \text{neV/T}.
\end{equation}
\indent UCN of anti-parallel spin to the magnetic field ( high field
seeker) have negative $V_m$, accelerate toward higher fields and are
attracted and UCNs with parallel spin (low field seeker) have positive
$V_m$ and are repelled. If the UCN spin adiabatically traces the
magnetic field, it will be fully polarized which can be achieved by
passing UCN through a strong~$\sim 6$~T magnetic field.

 \subsubsection{The strong Interaction:} The strong interaction governs the
 UCN interaction with material walls. It can be described by the Fermi
 potential $V_F$ which arises from the coherent elastic scattering
 from nuclei. The highest known value ($V_F=335$~neV) is measured for
 $^{58}$Ni and it sets the upper limit of the UCN kinetic energy,
 since UCN are normally defined by the property of total reflection
 from materials.

 \subsection{Superthermal Sources of Ultracold Neutrons}
\label{sec:ucn_with_heII}
 


%%%%%%%%%%%%%%%%%%%%%%%%%%%%%%%%%%%%%%%%%
%%%%%%%%%%%%%%%%%%%%%%%%%%%%%%%%%%%%%%%%%
%\begin{description}
%\item{A few Paragraphs about what the whole thesis is about. It is more like
%  the introduction to the introduction!}
    
%\item{Physics Interest in nEDM and why do we bother measuring it. I
%  guess it means I have to include some theory here but it can also go
%  to the next chapter. Depends on how much information I want to
%  include here. I am more thinking of having some background theory
%  related to my EDM report.}

%\item{nEDM status worldwide, where all the facilities around the world
%  are and the current upper limit, a short comparison of these
%  facilities. This is a way to motivate that the TRIUMF's attempt is
%  unique in combining spallation and superfluid helium. I guess this
%  point should also go into the conlcusion.}


%\item{A short history of EDM maybe? I am interested in this!}

  
%\item{An intro to the importance of magnetic stability for the nEDM
%  measurement. It might be hard to motivate this without providing
%  much informattion (maybe?) but there is a whole chapter dedicated to
%  it so it might be OK. }
  
%\item{A short intro to UCN and what they are, their properties. There
%  is a chapter dedicated to UCN.}

%\item{It is also good to say why UCNs are interesting (kind of related
%  the the previous bullet) and what experiments are done with UCN}

%\item{what else?}
  
%\end{description}

