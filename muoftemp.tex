\chapter{Temperature Dependence of Magnetic Permeability\label{chap:muofT}}
This chapter is based on a study of the sensitivity of fields
generated within magnetically shielded volumes to changes in magnetic
permeability presented in Ref.~\cite{Andalib:2016ahj}.

The previous best nEDM measurement~\cite{Baker2006, Pendlebury2015} showed that
the dominant systematic uncertainty is related to the magnetic field
homogeneity and instability. As a result, much of the research and
development efforts for these experiments are focused on careful
design and testing of various magnetic shield geometries with
precision magnetometers~\cite{altarev2014magnetically,
  brys2005magnetic,afach2014dynamic,patton2014all}. As an example,
mechanical and temperature changes of the passive magnetic
shielding~\cite{voigt2013,thiel2007demagnetization}, and the
degaussing
procedure~\cite{thiel2007demagnetization,altarev2015minimizing,sun2016dynamic}
(also known as demagnetization, equilibration, or idealization),
affect the stability of the magnetic field within magnetically
shielded rooms. Using a shield-coupled coil, since the $B_0$ coil is
magnetically coupled to the innermost magnetic shield, any change in
the magnetic properties of the innermost shield will result in a change in
the internal magnetic field. This chapter focuses on this
effect and characterize one possible source of instability: changes of
the magnetic permeability $\mu$ of the material with temperature when
the magnetic shield is used as a return yoke for the internal coil
system. This shield-coupled coil design was used in the previous best
nEDM experiment.



%In the previous best experiment to measure
%nEDM~\cite{Baker2006,Pendlebury2015} which discovered
%$d_n<3.0\times 10^{-26}$~e$\cdot$cm~(90\% C.L), effects related to
%magnetic field homogeneity and instability were found to dominate the
%systematic error. To reduce the external magnetic fields several types
%of shielding such as active and passive shielding are essential for
%such precision measurements. In addition, internal coils that are
%usually coupled to the passive shields are used to generate the $B_0$
%field. As a result, a detailed understanding of passive and active
%magnetic shielding, magnetic field generation within shielded volumes,
%and precision magnetometry is expected to be crucial to achieve the
%systematic error goals for the next generation of experiments. Much of
%the research and development efforts for these experiments are focused
%on careful design and testing of various magnetic shield geometries
%with precision magnetometers~\cite{altarev2014magnetically,
%  brys2005magnetic,afach2014dynamic,patton2014all}.

%In nEDM experiments, the spin precession frequency $\omega$ of neutrons
%placed in static magnetic $B_0$ and electric $E$ fields is
%measured~(see Chapter~\ref{chap:nedm}). The measured frequencies for
%parallel $\omega_+$ and antiparallel $\omega_-$ relative orientations of the
%fields is sensitive to the neutron electric dipole moment $d_n$
%\begin{equation}
%\hbar\omega_\pm=2\mu_nB_0\pm 2d_nE
%\end{equation}
%where $\mu_n$ is the magnetic moment of the neutron.
%In such experiments, typically $B_0=1~\mu$T is used to provide the
%quantization axis for the UCN. The $B_0$ magnetic
%field generation system typically includes a coil placed within a
%passively magnetically shielded volume.


%Mechanical and temperature changes of the passive magnetic
%shielding~\cite{voigt2013,thiel2007demagnetization}, and the degaussing
%procedure~\cite{thiel2007demagnetization,altarev2015minimizing,sun2016dynamic}
%(also known as demagnetization, equilibration, or idealization),
%affect the stability of the magnetic field within magnetically
%shielded rooms. Active stabilization of the background magnetic field
%surrounding magnetically shielded rooms can also improve the internal
%stability~\cite{voigt2013,afach2014dynamic,Franke_thesis}. The
%current supplied to the $B_0$ coil is generated by an ultra-stable
%current source~\cite{brys2005magnetic}. The coil must also be
%stabilized mechanically relative to the magnetic shielding.

%One additional effect, which is the subject of this chapter, relates to
%the fact that the $B_0$ coil in most nEDM experiments is magnetically
%coupled to the innermost magnetic shield. If the magnetic properties
%of the innermost magnetic shield change as a function of time, it then
%results in a source of instability of $B_0$.  In the present work, we
%estimate this effect and characterize one possible source of
%instability: changes of the magnetic permeability $\mu$ of the
%material with temperature.



The chapter proceeds in the following fashion:
\begin{itemize}
\item The dependence of the internal field on magnetic permeability of
  the innermost shielding layer for a typical nEDM experiment geometry
  is estimated using a combination of analytical and finite element
  analysis techniques.  This sets a scale for the stability
  requirement.
\item New measurements of the temperature dependence of the magnetic
  permeability are presented.  The measurements were done in two ways
  in order to study a variety of systematic effects that were
  encountered.
\item Finally, the results of the calculations and measurements are
  combined to provide a range of temperature sensitivities that takes
  into account sample-to-sample and measurement-to-measurement
  variations.
\end{itemize}

\input{calculations.tex}


\section{Measurements of $\mu(T)$\label{sec:tdep}}

\subsection{Previous Measurements and their Relationship to nEDM
  Experiments\label{sec:previousmeasurement}}

Previous measurements of the temperature dependence of the magnetic
properties of high permeability alloys have been summarized in
Refs.~\cite{couderchon1982some,bozorth1993ferromagnetism,pfeifer1980soft}. These
measurements are normally conducted using a sample of the material to
create a toroidal core, where a thin layer of the material is used in
order to avoid eddy-current and skin-depth
effects~\cite{pfeifer1980soft,kruppvdm}. A value of $\mu$ is
determined by dividing the amplitude of the sensed $B_m$-field by the
amplitude of the driving AC $H_m$-field (similar to the method
described in Section~\ref{sec:transformer}). Normally the frequency
of the $H_m$-field is 50 or 60~Hz.  The value of $\mu$ is then quoted
either at or near its maximum attainable value by adjusting the
amplitude of $H_m$.  Depending on the details of the $B_m-H_m$ curve
for the material in question, this normally means that $\mu$ is quoted
for the amplitude of $H_m$ being at or near the coercivity of the
material~\cite{couderchon1982some,kruppvdm}, resulting in large values
up to $\mu_r=4\times 10^5$.

It is well known that $\mu$ measured in this fashion for toroidal,
thin metal wound cores depends on the annealing process used for the
core.  There is a particularly strong dependence on the take-out or
tempering temperature after the high-temperature portion of the
annealing process has been
completed~\cite{pfeifer1980soft,kruppvdm,couderchon1982some}. Such
studies normally suggest a take-out temperature of 490-500$^\circ$C.
This ensures that the large $\mu_r=4\times 10^{5}$ is furthermore
maximal at room temperature. Slight variations around room
temperature, and assuming the take-out temperature is not controlled
to better than a degree, imply a scale of possible temperature
variation of $\mu$ of approximately
$\left|\frac{1}{\mu}\frac{d\mu}{dT}\right|\simeq 0.3$-1\%/K at room
temperature~\cite{couderchon1982some,kruppvdm}.

A challenge in applying these results to temperature stability of nEDM
experiments is that, when used as DC magnetic shielding, the
high-permeability alloys are usually operated for significantly
different parameters ($B_m$, $H_m$, and frequencies).

For example, when used in a shielding configuration, the effective
permeability is often measured to be typically $\mu_r=20,000$ rather
than $4\times 10^5$. This arises in part because $H_m$ is well below
the DC coercivity.  As noted in Section~\ref{sec:calculation}, a more
appropriate $H_m$ for the innermost magnetic shield of an nEDM
experiment is $<0.007$~A/m, whereas the coercivity is
$H_c=0.4$~A/m~\cite{kruppvdm}.  The frequency dependence of the
measurements could also be an issue.  Typically, nEDM experiments are
concerned with slow drifts at $<0.01$~Hz timescales whereas the
previously reported $\mu(T)$ measurements are performed in an AC mode
at 50-60~Hz. The goal of the experiments discussed in this chapter
was to develop techniques to characterize the material properties of
our own magnetic shields post-annealing, in regimes more relevant to
nEDM experiments.

%-40,000$ for
%3.5~$\mu$T applied external axial and transverse fields at frequencies
%$<0.1$~Hz, where the uncertainty in $\mu_r$ comes a model uncertainty
%resulting from incomplete inclusion of the geometry.



The magnetic shielding factors of each of our prototype four
cylindrical shells, and of various combinations of them, were measured
and found to be consistent with $\mu_r\sim 20,000$.  In our studies of
the material properties of these magnetic shields, two different
approaches to measure $\mu(T)$ were pursued.  Both approaches involved
experiments done using witness cylinders made of the same material and
annealed at the same time as the prototype magnetic shields.  We
therefore expect they have the same magnetic properties as the larger
prototype shields, and they have the advantage of being smaller and
easier to perform measurements with.

The two techniques employed to determine $\mu(T)$ were the following:
\begin{enumerate}
\item measuring the low-frequency AC axial magnetic shielding factor
  of the witness cylinder as a function of temperature, and
\item measuring the temperature dependence of the slope of a minor $B-H$
  loop, using the witness cylinder as a transformer core, similar to
  previous measurements of the temperature dependence of $\mu$, but
  for parameters closer to those encountered in nEDM experiments.
\end{enumerate}
The details and results of each technique are presented below.

% Axial shielding factor measurements Section moved to axial.tex


\subsection{Axial Shielding Factor Measurements\label{sec:axial}}

In these measurements, a witness cylinder was used as a magnetic
shield.  The shield was subjected to a low-frequency AC magnetic field
of $\sim 1$~Hz.  The amplitude of the shielded magnetic field $B_s$
was measured at the center of the witness cylinder using a fluxgate
magnetometer.  Changes in $B_s$ with temperature signify a dependence
of the permeability $\mu$ on temperature.  The relative slope of
$\mu(T)$ can then be calculated using
\begin{equation}
\frac{1}{\mu}\frac{d\mu}{dT}=-\frac{\frac{1}{B_s}\frac{dB_s}{dT}}{\frac{\mu}{B_s}\frac{dB_s}{d\mu}}.
\label{eqn:axial}
\end{equation}
The numerator was taken from the measurements described above. The
denominator was taken from finite-element simulations of the shielding
factor for this geometry as a function of $\mu$.

This measurement technique was sufficiently robust to extract the
temperature dependence of the shielding factor with some degree of
certainty. Possible drifts and temperature dependence of the fluxgate
magnetometer offset were mitigated by using an AC magnetic field.  Any
temperature coefficients in the rest of the instrumentation were
controlled by performing the same measurements with a copper
cylindrical shell with the similar size and shape as the mu-metal
witness cylinders in place of the mu-metal witness cylinder.

This technique is quite different than the usual transformer core
measurements conducted by other groups.  As shall be described, it
offers an advantage that considerably smaller $B_m$ and $H_m$ fields
can be accessed.  Measuring the temperature dependence of the
shielding factor is also considerably easier than measuring the
temperature dependence of the reaction factor, since the sensitivity
to changes in $\mu(T)$ is considerably larger in magnitude for the
shielding factor case where $\frac{\mu}{B_s}\frac{dB_s}{d\mu}\sim -1$
compared to the reaction factor case where
$\frac{\mu}{B_0}\frac{dB_0}{d\mu}\sim 0.01$.


\subsubsection{Experimental Apparatus for Axial Shielding Factor
  Measurements}

The witness cylinder was placed within a homogeneous AC magnetic
field. The field was created within the magnetically shielded volume
of the prototype magnetic shielding system (described previously in
Section~\ref{sec:previousmeasurement} and chapter~\ref{chap:nedm}) in
order to provide a controlled magnetic environment.  A short solenoid
inside the shielding system was used to produce the magnetic
field. The solenoid has 14 turns with 2.6~cm spacing between the
wires.  The solenoid was designed so that the field produced by the
solenoid plus innermost shield approximates that of an infinite
solenoid.  The magnetic field generated by the solenoid was typically
1~$\mu$T in amplitude.  The solenoid current was varied sinusoidally
at typically 1~Hz.

The witness cylinder was placed into this magnetic field generation
system as shown schematically in Fig.~\ref{fig:geometry}. The
cylinder was held in place by a wooden stand.

A Bartington fluxgate magnetometer Mag-03IEL70~\cite{bartman} (low
noise) measured the axial magnetic field at the center of the witness
cylinder.  The fluxgate was a ``flying lead'' model, meaning that each
axis was available on the end of a short electrical lead, separable
from the other axes.  One flying lead was placed in the center of the
witness cylinder, the axis of the fluxgate being aligned with that of
the witness cylinder.  The fluxgate was held in place rigidly by a
plastic mounting fixture, which was itself rigidly mounted to the
witness cylinder.

To increase the resolution of the measured signal from the fluxgate, a
Bartington Signal Conditioning Unit (SCU) was used with a low-pass
filter set to typically 10-100~Hz and a gain set to typically $>50$.
The signal from the SCU was demodulated by an SR830 lock-in
amplifier~\cite{lockin} providing the in-phase and out-of-phase
components of the signal.  The sinusoidal output of the lock-in
amplifier reference output itself was normally used to drive the
solenoid generating the magnetic field.  The time constant on the
lock-in was typically set to 3 seconds with 12~dB/oct. rolloff.

\begin{figure}
  \begin{center}
    \includegraphics[width=0.7\textwidth]{to_jeff_new4.pdf}
    \caption[Drawing of the axial shielding factor measurement setup
    ]{Axial shielding factor measurement setup. The witness cylinder
      with an inner diameter of 5.2~cm and a length of 15.2~cm is
      placed inside a solenoid (shown in red) with a diameter of
      30.8~cm and a length of 35.5~cm, containing 14 turns.  The
      thickness of the witness cylinder is $1/16''=0.16$~cm.  The loop
      coil (shown in blue) is mechanically coupled to the witness
      cylinder and has a diameter of 9.7~cm.}
    \label{fig:geometry}
  \end{center}
\end{figure}

As shall be described in this section, a concern in the
measurement was changes in the field measured by the fluxgate that
could arise due to motion of the system components, or other
temperature dependences. This could generate a false slope with
temperature that might incorrectly be interpreted as a change in the
magnetic properties of the witness cylinder.

To address possible motion of the witness cylinder with respect to the
field generation system, another driving coil (the loop coil, also
shown in blue in Fig.~\ref{fig:geometry}) was wound on a plastic
holder mounted rigidly to the witness cylinder.  The coil was one loop
of copper wire with a diameter of 9.7~cm.  Plastic set screws in the
holder fixed the loop coil to be coaxial with the witness cylinder.

Systematic differences in the results from the two coils (the
solenoidal coil, and the loop coil) were used to search for motion
artifacts.  As well, some differences could arise due to the different
magnetic field produced by each coil, and so such measurements could
reveal a dependence on the profile of the applied magnetic field.
This is described further in this section.

The temperature of the witness cylinder was measured by attaching four
thermocouples at different points along the outside of the cylinder.
This allowed us to observe the temperature gradient along the witness
cylinder.  To reduce any potential magnetic contamination, T-type
thermocouples were used, which have copper and constantan conductors.
(K-type thermocouples are magnetic.)  Thermocouple readings were
recorded by a National Instruments NI-9211 temperature input module.
The magnetic field (signified by the lock-in amplifier readout) and
the temperature were recorded at a rate of 0.2~Hz. Temperature
variations in the experiment were driven by ambient temperature
changes in the room, although forced air and other techniques were
also tested. These are described further in this section.


\subsubsection{Data and Interpretation\label{sec:axialsyst}}

An example of the typical data acquired is shown in
Fig.~\ref{fig:B_vs_Temp}.  For these data, the field applied by the
solenoid coil was 1~$\mu$T in amplitude, at a frequency of 1~Hz.
Fig.~\ref{fig:B_vs_Temp}(a) shows the temperature of the witness
cylinder over a 70-hr measurement.  The temperature changes of 1.4~K
are caused by daily temperature variations in the laboratory. The
shielded magnetic field amplitude $B_s$ within the witness cylinder is
anti-correlated with the temperature trend as shown in
Fig.~\ref{fig:B_vs_Temp}(b).  Here, $B_s$ is the sum in quadrature of
the amplitudes of the in-phase and out-of-phase components (most of
the signal is in phase).  The magnetic field is plotted as a function
of temperature in Fig.~\ref{fig:B_vs_Temp}(c).  The slope in
Fig.~\ref{fig:B_vs_Temp}(c) was calculated using a linear fit to the
data.  The relative slope at 23$^\circ$C was found to be
$\frac{1}{B_s}\frac{dB_s}{dT}=-0.75\%$/K.

\begin{figure}
  \begin{center}
    \includegraphics[width=\textwidth]{fig3.png}
    \caption[Ambient temperature and shielded magnetic field amplitude
    measurement]{Ambient temperature and shielded magnetic field
      amplitude, measured over a 70 hour period. (a) temperature of
      the witness cylinder as a function of time.  (b) magnetic field
      amplitude measured by fluxgate at center of witness cylinder
      vs.~time.  (c) magnetic field vs.~temperature with linear fit to
      data giving $\frac{1}{B_s}\frac{dB_s}{dT}=-0.75\%$/K (evaluated
      at 23$^\circ$C).  In panels (d), (e), and (f), the same
      quantities are shown for a 20-hour run with a copper cylinder in
      place of the witness cylinder with the linear fit giving
      $\frac{1}{B_s}\frac{dB_s}{dT}=-0.03\%$/K.}
    \label{fig:B_vs_Temp}
  \end{center}
\end{figure} 

Figs.~\ref{fig:B_vs_Temp}(d), (e), and (f) show the same measurement
with essentially the same settings, when the mu-metal witness cylinder
was replaced by a copper cylinder.  A similar relative vertical scale
was used in Figs.~\ref{fig:B_vs_Temp}(e) and (f) as
Figs.~\ref{fig:B_vs_Temp}(b) and (c).  This helps to emphasize the
considerably smaller relative slope derived from panel (f) compared to
panel (c).  A variety of measurements of this sort were carried out
multiple times for different parameters such as coil current.  Running
the coil at the same current tests for effects due to heating of the
coil, whereas running the coil at a current which equalizes the
fluxgate signal to its value when the mu-metal witness cylinder is
present tests for possible effects related to the fluxgate.  For all
measurements the temperature dependence of the demodulated magnetic
signal was $<0.1$\%/K, giving confidence that unknown systematic
effects contribute below this level.

Some deviations from the linear variation of $B_s$ with $T$ can be
seen in the data, particularly in Figs.~\ref{fig:B_vs_Temp}(a), (b),
and (c).  For example, when the temperature changes rapidly, the
magnetic field takes some time to respond, resulting in a slope in
$B_s-T$ space that is temporarily different than when the temperature
is slowly varying.  This is typical of the data that we acquired, that
the data would generally follow a straight line if the temperature
followed a slow and smooth dependence with time, but the data would
not be linear if the temperature varied rapidly or non-monotonically
with time.  We also tried other methods of temperature control, such
as forced air, liquid flowing through tubing, and thermo-electric
coolers.  The diurnal cycle driven by the building's air conditioning
system gave the most stable method of control and the most
reproducible results for temperature slopes.

As mentioned earlier, data were acquired for both the solenoid coil
and the loop coil. A summary of the data is provided in
Table~\ref{tab:axial}.  Repeated measurements of temperature slopes
using the loop coil fell in the range
0.4\%/K~$<\vert\frac{1}{B_s}\frac{dB_s}{dT}\vert<$~1.5\%/K.  Similar
measurements for the solenoidal coil yielded
0.3\%/K~$<\vert\frac{1}{B_s}\frac{dB_s}{dT}\vert<$~0.8\%/K.

\begin{table}
\begin{center}
\begin{tabular}{cccc}\hline
Trial & $\frac{1}{B_s}\frac{dB_s}{dT}$ & Coil \\
\#    & (\%/K) & type \\\hline
 1 & -0.32 & solenoid \\
 2 & -0.30 & solenoid \\
 3 & -0.33 & solenoid \\
 4 & -1.53 & loop \\
 5 & -0.42 & loop \\
 6 & -1.30 & loop \\
 7 & -0.74 & solenoid \\
 8 & -1.05 & loop \\
 9 & -0.73 & solenoid \\
10 & -1.23 & loop \\
11 & -0.75 & solenoid \\
12 & -1.12 & loop \\\hline
\end{tabular}
\caption[Summary of the AC axial shielding factor
measurements]{Summary of data acquired for the AC axial shielding
  factor measurements, in chronological order.  Data with an applied
  field of $\sim 1-6~\mu T$ and a measurement frequency of 1~Hz are
  included.  Data which used daily fluctuations of the temperature
  from 21-24$^\circ$C over a 10-80 hour period are included.  Other
  data acquired for systematic studies are not included in the
  table.\label{tab:axial}}
\end{center}
\end{table}

In general, the slopes measured with the loop coil were larger than
for the solenoidal coil.  This is particularly evident for
measurements 6-12, which were acquired daily over the course of a few
weeks alternating between excitation coils but all used the same
witness cylinder and otherwise without disturbing the measurement
apparatus.  A partial explanation of this difference is offered by the
field profile generated by each coil, and its interaction with the
witness cylinder.  This is addressed further in this section.

The other difference between the loop coil and the solenoidal coil was
that the loop coil was rigidly mounted to the witness cylinder,
reducing the possibility of artifacts from relative motion.  Given
that this did not reduce the range of the measured temperature slopes
we conclude that relative motion was well controlled in both cases.

Several other possible systematic effects were considered, all of
which were found to give uncertainties on the measured slopes
$<0.1\%$/K.  These included: thermal expansion of components including
the witness cylinder itself, temperature variations of the magnetic
shielding system within which the experiments were conducted,
degaussing of the witness cylinder, and temperature slopes of various
components e.g. the fluxgate magnetometer and the lock-in amplifier.
Some of these studies are described below.



\paragraph{Methods of Temperature Variation}
In addition to ambient temperature changes, I tried other methods of
forced temperature change.  In one design, Tygon tubing was wrapped
around the witness cylinder in a spiral pattern to flow water whose
temperature could be controlled.  Mechanical stability issues clearly
dominated the systematic uncertainty in that measurement. When water
was flowing, the flexibility of the tubing caused a movement in the
witness cylinder. The motion was itself temperature-dependent since
warmer water caused the tubing to become more supple.  To address this
effect, the tubes were replaced by copper tubing.  But in this case,
the challenge was to create enough contact between the tubes and the
witness cylinder which was not successful. In another design, a TEC
was replaced with the tubing. The main issue with this design was that
it did not provide enough cooling for the witness cylinder and also it
was creating only local temperature changes on the witness
cylinder. In addition, despite of using heat sinks, the heat created
by the TEC itself made it very inefficient. I also tried using
forced air to heat the witness cylinder. This worked rather well, but
the heating had to be done slowly in order to avoid temperature
gradients across the apparatus, including the witness cylinder. In
the end, using the ambient temperature changes in the room gave the
most reproducible results.  These followed a relatively stable diurnal
cycle with the function of the building's air conditioning system.

Although for most of the measurements the general trend of $B(T)$
graphs was consistent, the shape and positions of the nonlinear parts
of $B-T$ graphs were changing.  The changes in the $B$ vs.~temperature
slope always correlate with sharper changes in the temperature with
time.  The effect is most pronounced when a temperature that is
decreasing with time suddenly changes to increasing, or vice-versa.
However, I have incorporated the uncertainty from this effect into our
stated range of values.




\paragraph{Mechanical Stability}
Other potential motion artifacts due to thermal expansion of
components was also considered.  The thermal expansion coefficient of
mu-metal is $\sim$10~ppm/K~\cite{kruppvdm}.  However if the witness
cylinder expands uniformly in both thickness and radius, the shielding
factor is to first order unchanged.  In general, even unnatural
asymmetric and twisting motions of the fluxgate sensor and witness
cylinder tended to generate temperature slopes in the magnetic field
at the level $<30$~ppm/K. The general homogeneity of the magnetic
field at the fluxgate sensor position and of the applied magnetic
field within which the witness cylinder was placed aided in minimizing
motion artifacts.


As another mechanical stability study, the movement of the Bartington
fluxgate flying lead due to thermal expansion was estimated. If the
fluxgate flying lead move about 1 mm normal to its axis of symmetry
which is parallel to the axis of the witness cylinder, the magnetic
field will change about 30~ppm/K over 20~K temperature changes.



%\paragraph{Vibration}
%Since the experiment site was located at the heart of downtown in Winnipeg it is
%also possible that vibrations of the building affected the experiment
%setup and its machanical stability.

\paragraph{Temperature Dependence of Reaction Factor}
As the witness cylinder was put through its diurnal heating and
cooling cycles, so too was the magnetic shield within which the
apparatus was placed.  Since this magnetic shield is used as a flux
return, especially for the solenoidal coil, a concern could be that
the measurement confounds temperature dependence of the flux return
with the temperature dependence of the shielding factor of the
solenoid.  We want to clarify that this cannot be the case: any change
in $\mu$ of the flux return will have an exceedingly small effect on
the field produced by the solenoid.  This is perhaps best demonstrated
by Fig. \ref{fig:Magnetic_Field}, where the reaction factor in a
similar cylindrical geometry is graphed as a function of $\mu$. Based
on our measurements, this limits systematic errors from such an effect
to be $<200$~ppm/K.


\paragraph{Degaussing}
The magnetization of the witness cylinder changes the magnetic
permeability of the material and so the shielding factor changes.  Our
studies of degaussing the witness cylinder were consistent with
studies that we will report in
Section~\ref{sec:transformersystematics}.  Essentially, if the shields
were degaussed, or if they were left for long periods of time in the
small AC field generated by the solenoid, the results for temperature
dependences were similar.  Improper degaussing procedures were found
to induce long-term drifts in the measurement, uncorrelated with
temperature.  We do not include such data when quoting our
measurements of temperature slopes.  We do think that part of the
range of slopes that we measured is due to the magnetic properties of
the material, and that it is possible that some of this range is yet
due to insufficient degaussing on our part.  This is something we plan
to improve in planned future experiments on DC field stability.




\paragraph{Temperature Slopes of Various Components of the Apparatus}

The temperature coefficients of various components that could affect
the measurement were also considered.

The Mag-03IEL70 Bartington magnetic field sensor has a scaling
temperature coefficient of 15~ppm/K~\cite{bartman}.  There is also
temperature coefficient for the offset of these sensors, but this is
irrelevant for this measurement because of the AC fields and
demodulation technique used.

The SRS830 lock-in amplifier has 50~ppm/K amplitude
stability~\cite{lockin}.
% We did this for the transformer technique.
% Yes, but I think it is valid for this measurement as well, so let's put
% it here.
To further test this, the lock-in amplifier was connected to the coil
through a 1~$\Omega$ resistor with small temperature coefficient.  The
voltage across the resistor was measured with the lock-in amplifier
itself.  Any change would then be interpreted as a change in the
current supplied to the coil by the lock-in amplifier. The measured
temperature dependence was always $<0.1$\%/K.


%\paragraph{Stability Test with Copper Cylinder}
%The stability of the system was also tested by replacing the mu-metal
% witness cylinder with a copper cylinder of very similar dimensions.
% All other components of the system were the same.  The apparatus was
% then run through its usual experimental cycle over several days.
% For all such measurements the temperature dependence of the
% demodulated magnetic was $<0.1$\%/K.  This kind of measurement is
% unable to address all possible systematic uncertainties. For
% example, if moving the mu-metal witness cylinder due to some thermal
% expansion would change the field at the site of the fluxgate, moving
% the copper cylinder will not make the same change.  Nonetheless it
% was an encouraging result that the system does not measure a strong
% temperature dependence of the magnetic field when no mu-metal
% witness cylinder is present.  The magnitude of the magnetic field
% measured at the fluxgate sensor was larger during these measurements
% because of the lack of magnetic shielding from the copper.  So, in
% some tests the magnetic field generated by the reference channel of
% the lock-in amplifier was reduced to search for any problems arising
% from smaller fluxgate signals. No difference was seen within the
% upper bound stated above.


\paragraph{Different Witness Cylinders}
The manufacturer of our prototype magnetic shields provided us with
three witness cylinders.  All three were used in these measurements.
Different cylinders possessed systematically different temperature
slopes, although always within the ranges quoted above.  These changes
are believed to arise from the manufacturing and annealing process.
It is known that the take-out temperature in the annealing process has
a strong effect on the temperature slopes measured at
50~Hz~\cite{kruppvdm}.




\paragraph{Stability Test with Copper Cylinder}
%%% null meast
As mentioned earlier in reference to Fig.~\ref{fig:B_vs_Temp}(d), (e),
and (f), the stability of the system was also tested by replacing the
mu-metal witness cylinder with a copper cylinder and in all cases
temperature slopes $<0.1$\%/K were measured, giving confidence that
other unknown systematic effects contribute below this level.

Based on the systematic effects that we studied, we conclude that they
do not explain the ranges of values measured for
$\frac{1}{B_s}\frac{dB_s}{dT}$.  We suspect that the range measured is
either some yet uncharacterized systematic effect, or a complicated
property of the material. We use this range to set a limit on the
slope of $\mu(T)$


\subsubsection{Geometry correction and determination of $\mu(T)$\label{sec:axialsims}}

To relate the data on $B_s(T)$ to $\mu(T)$, the shielding factor of
the witness cylinder as a function of $\mu$ must be known. Finite
element simulations in FEMM and OPERA were performed to determine this
factor.  The simulations are also useful for determining the effective
values of $B_m$ and $H_m$ in the material, which will be useful to
compare to the case for typical nEDM experiments when the innermost
shield is used as a flux return.

For closed objects, such as spherical
shells~\cite{bidinosti2014passive, urankar1996design}, the shielding
factor approaches infinity as $\mu \rightarrow \infty$, and
$\vert\frac{\mu}{B_s}\frac{dB_s}{d\mu}\vert\rightarrow 1$.  Because
the witness cylinders are open ended, the shielding factor
asymptotically approaches a constant rather than infinity in the
high-$\mu$ limit, and as a result
$\vert\frac{\mu}{B_s}\frac{dB_s}{d\mu}\vert<1$ here.  From the
simulations the ratio $\frac{\mu}{B_s}\frac{dB_s}{d\mu}$ was
calculated.  A linear model of the material was used where
$\bold{B_m}=\mu\bold{H_m}$ with $\mu$ constant.


The simulations differed slightly in their results, dependent on
whether OPERA or FEMM was used, and whether the solenoidal coil or
loop coil were used.  Based on the simulations, the result is
$\vert\frac{\mu}{B_s}\frac{dB_s}{d\mu}\vert=0.42-0.50$ for the
solenoidal coil, with the lower value being given by FEMM and the
upper value being given by a 3D OPERA simulation, for identical
geometries.  This is somewhat lower than the value suggested by
Ref.~\cite{paperno1999charts} with fits to simulations performed
in OPERA, which we estimate to be 0.6.  We adopt our value since it is
difficult to determine precisely from
Ref.~\cite{paperno1999charts}.  For the loop coil, we determine
$\vert\frac{\mu}{B_s}\frac{dB_s}{d\mu}\vert=0.56-0.65$, the range
being given again by a difference between FEMM and OPERA.

Combining the measurement and the simulations, the temperature
dependence of the effective $\mu$ (at $\mu_r=20,000$ which is
consistent with our measurements) can be calculated by
equation~(\ref{eqn:axial}).  The results of the simulations and
measurements is presented in Table~\ref{tab:axialsummary}. Combining
the loop coil and solenoidal coil results, we find
0.6\%/K~$<\frac{1}{\mu}\frac{d\mu}{dT}<2.7\%$/K to represent the full
range for the possible temperature slopes of $\mu$ that were observed
in these measurements.

\begin{table}
\begin{center}
\begin{tabular}{|c|c|c|c|}
\hline 
  & $\vert \frac{\mu}{B_s}\frac{dB_s}{d\mu}\vert$ & $\vert \frac{1}{B_s} \frac{dB_s}{dT}\vert$~(\%/K) & $\frac{1}{\mu}\frac{d\mu}{dT}$~(\%/K) \\ 
 & (simulated) & (measured) & (extracted) \\
\hline 
Solenoidal Coil & 0.42-0.50 & 0.3-0.8 & 0.6-1.9 \\ 
\hline 
Loop Coil & 0.56-0.65 & 0.4-1.5 & 0.6-2.7 \\ 
\hline 
\end{tabular} 
\caption[Summary of finite element simulations and shielding factor
measurements]{Summary of OPERA and FEMM simulations and shielding
  factor measurements, resulting in extracted temperature slopes of
  $\mu$.}
\label{tab:axialsummary}
\end{center}

\end{table}


As stated earlier, the simulations also provided a way to determine
the typical value of $B_m$ and $H_m$ internal to the material of the
witness cylinder.  According to the simulations, the $B_m$ amplitude
was typically 100~$\mu$T and the $H_m$ amplitude was typically
0.004~A/m.  These are comparable to the values normally encountered in
nEDM experiments, recalling from Section~\ref{sec:calculation} that
$H_m<0.007$~A/m for the innermost magnetic shield of an nEDM
experiment.  A caveat is that these measurements were typically
conducted using AC fields at 1~Hz, as opposed to the DC fields
normally used in nEDM experiments.


% Transformer core measurements Section moved to transformer.tex

\input{transformer.tex}

\section{Relationship to nEDM experiments\label{sec:relationship}}
%\subsection{Summary of $\mu(T)$}

Neutron EDM experiments are typically designed with the DC coil
magnetically coupled to the innermost magnetic shield. As discussed in
Section~\ref{sec:calculation}, if the magnetic permeability of the
shield changes, this results in a change in the field in the
measurement region by an amount
$\frac{\mu}{B_0}\frac{dB_0}{d\mu}=0.01$.

The temperature dependence of $\mu$ has been constrained by two
different techniques using open-ended mu-metal witness cylinders
annealed at the same time as our prototype magnetic shields.  We
summarize the overall result as
0.0\%/K~$<\frac{1}{\mu}\frac{d\mu}{dT}<$~2.7\%/K, where the range is
driven in part by material properties of the different mu-metal
cylinders, and in part by day-to-day fluctuations in the temperature
slopes.



We note the following caveats in relating this measurement to nEDM
experiments:
\begin{itemize}
\item Although the measurement techniques rely on considerably larger
  frequencies and different $H_m$-fields than those relevant to
  typical nEDM experiments, we think it reasonable to assume the
  temperature dependence of the effective permeability should be of
  similar scale.  For frequency, both techniques typically used a 1~Hz
  AC field, whereas for nEDM experiments the field is DC and stable at
  the 0.01~Hz level.  Furthermore, in one measurement technique the
  amplitude of $H_m$ was $\sim 0.004$~A/m and in the other was $\sim
  0.1$~A/m.  For nEDM experiments $H_m<0.007$~A/m and is DC.
\item Both measurement techniques extract an effective $\mu$ that
  describes the slope of minor loops in $B_m-H_m$ space.  A more
  detailed treatment would include a more comprehensive account of
  hysteresis in the material, which is beyond the scope of this work.
\end{itemize}

Our measurement of
0.0\%/K~$<\frac{1}{\mu}\frac{d\mu}{dT}<2.7$\%/K and the generic EDM
experiment sensitivity of $\frac{\mu}{B_0}\frac{dB_0}{d\mu}=0.01$
give a temperature dependence of the magnetic field in a typical
nEDM experiment of $\frac{dB_0}{dT}=0-270$~pT/K.  To achieve a goal of
$\sim 1$~pT stability in the internal field for nEDM experiments, the
temperature of the innermost magnetic shield in the nEDM experiment
should then be controlled to the $<0.004$~K level if the worst-case
dependence is to be taken into account.  This represents a potentially
challenging design constraint for future nEDM experiments.

As noted by others~\cite{khriplovich1997cp}, the use of
self-shielded coils to reduce the coupling of the $B_0$ coil to the
innermost magnetic shield is an attractive option for EDM experiments.
The principle of this technique is to have a second coil structure
between the inner coil and the shield, such that the net magnetic
field generated by the two coils is uniform internally but greatly
reduced externally.  For a perfect self-shielded coil, the field at
the position of the magnetic shield would be zero, resulting in
perfect decoupling, which is to say a reaction factor that is
identically unity.  For ideal geometries, such as spherical
coils~\cite{brown1945fluxball, wheeler1958spherical,purcell1989} or infinitely long
sine-phi coils~\cite{bethBNL,bethUSpatent,bidinosti2005}, the
functional form of the inner and outer current distributions are the
same, albeit with appropriately scaled magnitudes and opposite sign.
More sophisticated analytical and numerical methods have been used
extensively in NMR and MRI to design self-shielded
gradient~\cite{turner1986,hidalgo},
shim~\cite{brideson,forbes}, and transmit
coils~\cite{bidinosti2005,kuzmin}, and should be of value in the
context of nEDM experiments, as well.


\section{Conclusion}

In the axial shielding factor measurement, we found
0.6\%/K~$<\frac{1}{\mu}\frac{d\mu}{dT}<2.7\%$/K, with the measurement
being conducted with a typical $H_m$-amplitude of 0.004~A/m and at a
frequency of 1~Hz.  In the transformer core case, we found
0.0\%/K~$<\frac{1}{\mu}\frac{d\mu}{dT}<2.2\%$/K, with the measurement
being conducted with a typical $H_m$-amplitude of 0.1~A/m and at a
frequency of 1~Hz.

The primary caveat to these measurements is that both measurements
(transformer core and axial shielding factor) do not truly measure
$\mu$.  Rather they measure observables related to the slope of minor
hysteresis loops in $B_m-H_m$ space.  They would be more appropriately
described by a hysteresis model like that of
Jiles~\cite{jiles1994frequency}, but to extract the temperature
dependence of all the parameters of the model is beyond the scope of
this work.  Instead we acknowledge this fact and relate the
temperature dependence of the effective $\mu$ measured by each
experiment.

We think it is interesting and useful information that the two
experiments measure the same scale and sign of the temperature
dependence of their respective effective $\mu$'s.  This is a principal
contribution of this work.

In future work, we plan to measure $B_0(T)$ directly for nEDM-like
geometries using precision atomic magnetometers.  We anticipat based
on the present work that self-shielded coil geometries will achieve
the best time and temperature stability. The R\&D towards
self-shielded coils so far has resulted in three summer student
projects and an honours thesis~\cite{Rosie_thesis}.

%\textbf{add sth about Rosie's stuff here, waiting for Russ's
%  response}.







%\section{Shield Coupled Coils}
%\subsection{Spherical Geometry}
%\subsection{Cylindrical Geometry}


%\section{Method one: Measurement with (How did I used to call it???) }

%\section{Method two: Measurements with a transformer}




%\begin{description}
%\item{This chapter I think should be very similar to the paper. It
%  should be just the extended version of that with more detail (not a
%  lot more though! It means I should not include every single data
%  that I took!)}
%  \item{Motivation: Why are we interested in the temperature
%    dependence of mu. The idea of coupling the field of an internal
%    coil to the shield and that is because the shield act as a return
%    yoke. Then it makes sense to have a stable shield so that the
%    internal field does not change. And then explain that one of the
%    factors are actually the changes in the temperature that affects
%    the magnetic permeability.}
%  \item{The main part of this has to be about the measurements,
%    explain the technique, show several picture of the setup. Look at
%    the long version of the paper on github to remember what we did
%    and why and then include the data that made sense.}
%  \item{Another relevant part is to make the connection between these
%    measurements and the actual changes in the $B$ field and for that
%    I have to include the simulations. An explanation of the OPERA
%    simulations and FEMM. Explain what each one is and what I did.}
%\end{description}


