\chapter{UCN Facility at TRIUMF\label{chap:UCNattriumf}}


The vertical UCN cryostat at TRIUMF is the same UCN cryostat developed
and tested at
KEK-RCNP~\cite{masuda2002spallation,masuda2012spallation}. This source
is referred to as the vertical UCN source because the UCN exit the
source vertically.  In October 2016, the cryostat was shipped to
TRIUMF, and in 2017 it was installed at a dedicated spallation neutron
source for further UCN experiments. The main purpose of the initial
experiments were for better understanding of the vertical UCN source
which would guide the design of the next generation UCN source for
higher statistics. The cyclotron at TRIUMF provides up to 40~$\mu$A of
proton beam that can be diverted onto a tungsten spallation
target. The vertical UCN source is placed above the target and is
surrounded by graphite blocks serving as neutron reflectors.


The vertical source was modified to fulfill the safety requirements at
TRIUMF. The modifications included installing pressure relief valves
on the cryostat and the UCN guides, and additional radiation
shielding. The extra shielding requires longer UCN guides compared to
RCNP. The current location of the vertical source is at the Meson Hall
experimental area. A map of TRIUMF displaying the location of Meson
Hall is shown in Fig.~\ref{fig:sitemap}.

\begin{figure}[h!]
  \centering
  \includegraphics[width=1.0\textwidth]{sitemap.png}
  \caption[A map of TRIUMF]{A map of TRIUMF. The UCN facility is
    located at the Meson Hall area shown in Blue.}
  \label{fig:sitemap}
\end{figure}

The unique feature of the UCN source at TRIUMF is the combination of
spallation neutrons and superfluid helium for UCN production. The
important elements of the UCN facility at TRIUMF are described below.


\section{Proton Beam-line for UCN Facility~(BL1U)\label{sec:bl1u}}
TRIUMF produces negatively charged hydrogen ions H$^-$ from an ion
source. The ions are then accelerated in the 520~MeV cyclotron in an
outward spiral trajectory. A thin graphite stripper foil removes the
electrons from the hydrogen ion resulting in protons which pass
through. The proton, because it is positively charged, is deflected in
the outward direction due to the magnetic field, and is directed to a
proton beam-line.
%The cyclotron has three independent extraction
%probes with various sizes of foils to provide protons to up to three
%beam-lines~(BL) simultaneously~(see
Figure~\ref{fig:cyclotron} shows the beam line~(BL) delivering protons
exiting the cyclotron to the UCN facility.

\begin{figure}[h!]
  \centering
  \includegraphics[width=0.8\textwidth]{cyclotron.png}
  \caption{TRIUMF cyclotron and the three beam-lines.}
  \label{fig:cyclotron}
\end{figure}


The 120~$\mu$A beam~(BL1A) enters the Meson Hall, normally delivering
protons at 480~MeV to two target systems: T1 and T2 for the $\mu$SR
experimental channels. Beam-line 1B~(BL1B) separates off BL1 at the
edge of the cyclotron vault, and provides international users with the
Proton Irradiation Facility (PIF), which mimics space radiation for
testing computer chips. The new BL1U~\cite{beamline}~(shown in green
in fig.~\ref{fig:cyclotron}) provides beam to the UCN source. BL2A
provides 480~MeV proton beams for the targets that produce exotic ion
beams for a host of experiments in ISAC facilities.


The macrostructure of BL1A is in pulses with approximately 1~ms
periods of beam followed by a 50-100~$\mu$s periods of no beam. The
structure is shown schematically in
fig.~\ref{fig:bl1u}~\cite{Nick_thesis}. A kicker magnet delivers 1/3
of the beam from BL1A onto the septum magnet and onward to BL1U and
transport it to a conventional dipole~(bender) magnet~(see the bottom
panel of fig.~\ref{fig:bl1u} and fig.~\ref{fig:magnets}).

\begin{figure}[h!]
  \centering
  \includegraphics[width=0.9\textwidth]{bl1u.png}
  \caption[UCN beam structure]{UCN beam structure. The top graph shows
    the 120~$\mu$A BL1A in 1~ms period of beam followed by a
    50-100~$\mu$s of no beam. The middle graph shows the same
    beam-line when the kicker magnet is on. The bottom graph shows the
    1/3 of the beam that goes to the UCN area.}
  \label{fig:bl1u}
\end{figure}



\begin{figure}[h!]
  \centering
  \includegraphics[width=0.9\textwidth]{magnets.png}
  \caption[The kicker, septum and dipole magnets]{The kicker, septum
    and dipole~(bender) magnets define the front two sections of
    BL1U.}
  \label{fig:magnets}
\end{figure}
The vertical UCN cryostat is above the tungsten target. The target is
designed for a maximum of 40~$\mu$A beam on target. As a result, only
one third of the beam can go to the UCN experimental area, and the
rest is shared with other users~(see fig.~\ref{fig:bl1u}).



After the bender magnet, the beam passes through a cored shielding
block, and reaches the two quadrupole magnets, providing the final
focus of the beam onto a 12~cm thick tungsten spallation target.  The
target is located inside a hermetically-sealed target crypt, which
also envelops the beam-line exit window that defines the end of BL1U.
Upstream of the beam-line window, there is a collimator to reduce the
halo from the proton beam, as well as to help reduce the flux of
neutrons and photons streaming back into the beam-line from the target
region. The collimator also increases the impedance for the passage of
gas arising from any target or window failure, to allow time for the
cyclotron fast valves to close. This last part of the beam-line also
contains a variety of beam position and current monitors. The
spallation target and UCN source, located downstream of the beam
line-exit window, are enclosed in a large shielding pyramid shown in
fig.~\ref{fig:pyramid}.
\begin{figure}[h!]
  \centering
  \includegraphics[width=0.8\textwidth]{pyramid.png}
  \caption[Two quadrupole magnets and the shielding pyramid]{Two
    quadrupole magnets which focus the proton beam onto a 12~cm thick
    tungsten spallation target, located inside a hermetically-sealed
    target crypt. Also shown is the UCN shielding pyramid, which
    encases both the spallation target and the UCN source, and is
    designed to meet the dose rate requirements specified by the
    TRIUMF Safety Group.}
  \label{fig:pyramid}
\end{figure}


\section{Tungsten Spallation Target\label{sec:target}}
The spallation target is located at the downstream end of BL1U. The
UCN spallation target is composed of a series of rectangular blocks,
adding up to one stopping length of tungsten~(11~cm for 480~MeV
protons). This geometry is very similar to~(and motivated by) a
neutron spallation target design used previously at KEK~(KENS
facility)~\cite{kawai2001fabrication}: five blocks of tungsten with
tantalum plating constitute the target with 78~mm height and 57~mm
width, three with 20~mm length in the beam direction, two with 30~mm
length~(see fig.~\ref{fig:target}). The target has a support and
cooling system, designed to allow for remote-handling and ease of
servicing. The target-cooling and remote-handling systems are designed
for an instantaneous proton current of 40~$\mu$A~(10~$\mu$A
time-averaged).
%\begin{figure}[h!]
%  \centering
%  \includegraphics[width=0.8\textwidth]{target.png}
%  \caption[TUCAN's spallation target]{(a) Tungsten Target Blocks from
%    the spallation target at KEK. The target blocks are plated with
%    tantalum. (b) Present design for the tungsten spallation target at
%    the TRIUMF UCN facility. The target blocks have a cross-section of
%    $5.7 \times 7.8$~cm$^2$ , and thicknesses of 2.0, 2.0, 3.0, and
%    5.0~cm, respectively.}
%  \label{fig:target}
%\end{figure}

\begin{figure}[h!]
  \centering
  \begin{subfigure}{.5\textwidth}
    \centering
    \includegraphics[width=0.7\textwidth]{target_kek.png}
    \caption{}
    \label{fig:target_kek}
  \end{subfigure}%
  \begin{subfigure}{.5\textwidth}
    \centering
    \includegraphics[width=0.7\textwidth]{target_photo.png}
    \caption{}
    \label{fig:target_photo}
  \end{subfigure}
  \caption[TUCAN's spallation target]{(a) Tungsten Target Blocks from
    the spallation target at KEK. The target blocks are plated with
    tantalum. (b) Present design of the tungsten spallation target at
    the TRIUMF UCN facility. The target blocks have a cross-section of
    $5.7 \times 7.8$~cm$^2$ , and thicknesses of 2.0, 2.0, 2.0, 3.0,
    and 3.0~cm, respectively.}
  \label{fig:target}
\end{figure}

The target is water cooled. A water flow of approximately
0.8~L/s cools the target. Horizontal channels around the blocks create
a uniform flow. To reduce the beam absorption in the water the last
two blocks are thicker. Cooling water corrode tungsten.  Therefore, a
coating of tantalum with a thickness of $< 0.1$~mm prevents corrosion
by the water-cooling system. The estimated lifetime of the target is
longer than 10 years. An extraction system allows to exchange the
target when necessary.


\section{Vertical UCN Source at TRIUMF\label{sec:vertical_source}}
Neutrons are produced via proton-induced spallation off tungsten
target. Spallation is a nuclear reaction where high energy particles
interact with the atomic nucleus. This creates many high-energy
neutrons and less gamma radiation than fission~(per neutron). Since
the temperature of the superfluid helium is crucial, the heat load
into it should be kept as small as possible. Two effects dominate the
heat load: (1) heating by prompt $\gamma$ radiation and neutrons; and
(2) residual heating by radioactive decays of materials activated by
neutron capture.  Any heat deposited in the converter vessel and the
connected UCN guide will contribute to the heat load on the He-II, so
they should be thin-walled and made out of a material with low
density, low atomic number, and low neutron-absorption cross
section. Additionally, the vessel and UCN guide should be leak-tight
for superfluid helium, have a high optical potential for UCN, or allow
coating with a suitable material. The cold- and thermal-moderator
vessels have less direct impact on the heat load into the UCN
converter, but secondary particles from (n, $\gamma$) reactions and
radioactive decays can still contribute.




The target is surrounded by several blocks of lead and graphite. The
fast neutrons are reflected and moderated down and enter the warm
D$_2$O moderator at room temperature~(300~K) and become thermal
neutrons with an energy of 0.025~eV, or a speed of 2.2~km/s. To slow
the fast neutrons down via elastic scattering we need to choose a
moderator with a mass close to the mass of the free neutron. H$_2$O
will not work since hydrogen can absorb the neutron and become
deuterium. Therefore D$_2$O is a great choice as a neutron moderator.
Heavy water ice at 10~K is used as a cold moderator. After passing
through the warm D$_2$O, thermal neutrons enter the the cold moderator
and become cold neutrons. These neutrons have the speed of several
hundred~m/s. UCN are produced when the CN enter the isotopically pure
superfluid helium at 0.84 to 0.92~K. The CN induce phonon transitions
inside the superfluid helium in order to become UCN, as discussed in
Section~\ref{sec:ucn_with_heII}.




\begin{figure}[h!]
  \centering
  \includegraphics[width=0.9\textwidth]{vertical_source.png}
  \caption[Schematic diagram of the vertical UCN source at
  TRIUMF]{Schematic diagram of the vertical UCN source at
    TRIUMF. Spallation neutrons are moderated in warm D$_2$O vessel
    and become cold neutrons in Iced D$_2$O. The cold neutrons then
    enter the superfluid helium bottle where they become UCN by phonon
    excitations in the superfluid. The isotopically pure superfluid
    helium is cooled down to below 1~K via a $^3$He pot. The $^3$He
    pot is cooled down to 0.7~K via the 1~K pot and further
    pumping~(see text for more details). }
  \label{fig:source}
\end{figure}




\begin{figure}[h!]
  \centering
  \includegraphics[width=1.1\textwidth]{Source_all.png}
  \caption{The UCN source and the guide geometry at TRIUMF}
  \label{fig:Source_all}
\end{figure}


A schematic diagram of the vertical source is shown in
fig.~\ref{fig:source} and a 3D drawing of the vertical source and the
guide geometry is shown in fig.~\ref{fig:Source_all}. The
UCN-production volume is filled with superfluid helium and is cooled
by a Cu heat exchanger with the $^3$He pot~(see
section~\ref{sec:He3pot}). The combined height of the UCN-production
volume and vertical UCN guide is 1.25~m, with the lower 0.62~m filled
with superfluid helium. Right above the liquid surface, a short,
narrow section of the vertical guide blocks superfluid film flow and
radiant heat to reduce the heat load. Above the cryostat, the UCN
guide turns horizontally in a vacuum jacket to transition from
cryogenic to room temperature. It ends with a burst disk for emergency
pressure relief and a gate valve~(GV) with a protective ring improving
UCN transmission in the open state~(indicated as UCN valve on
fig.~\ref{fig:source} and fig.~\ref{fig:Source_all}).

Downstream of the GV, the UCN guide follows the 45$^\circ$ kinks to
avoid radiation leaking through a direct line of sight to the
experimental area. Finally, it penetrates through 3~m of additional
shielding and drops to accelerate the UCN by gravity to penetrate a
0.1~mm-thick aluminum foil and to enter the main detector with high
efficiency. The foil separates the helium vapour-filled UCN guide from
the detector vacuum to reduce contamination of the source. The main
detector is a $^6$Li detector described in
Section~\ref{sec:Li6detector}. A secondary $^3$He proportional counter
with its own aluminum window is mounted to a 5~mm pinhole in the guide
and serves as a monitor detector.




The neutron moderators, the helium circulation system, and the process
of cryostat operations are explained below.


\subsection{Neutron D$_2$O Moderators}
%Deuterium is an isotope of hydrogen which has one proton and one
%neutron in the nucleus, and it has a lower probability to absorb
%neutrons. As a result, heavy water is used as a neutron moderator.
The warm D$_2$O moderator to create thermal neutrons from spallation
neutrons is at room temperature. The cold moderator for the production
of cold neutrons is at much lower temperatures~($\sim$~10~K).

\subsubsection{D$_2$O Solidification}
The D$_2$O ice vessel has a capacity of 100~L. About 14~L of liquid
D$_2$O is injected to the vessel initially. This is followed by adding
11~L of D$_2$O to the vessel 8 times.  After filling up the vessel,
Gifford McMahon refrigerators solidify the heavy water and further
cool it down to 20~K. The process of icing the heavy water takes about
6 days and cooling it down to 20~K takes another 7 days.


\subsection{Helium Circulation and Superfluid Helium Condensation}
Helium circulation and superfluid helium condensation normally starts
once the temperature of D$_2$O has reached 10~K. The stages leading to
superfluid helium condensation are presented below. The full operation
and design details are available in Ref.~\cite{matsumiya_thesis}.

\subsubsection{Liquid Helium Reservoir}
The first step is to fill the 4~K liquid helium reservoir. The full
capacity of the reservoir is 50~L. In the 2017 experimental run, the
TUCAN collaboration developed a Labview program to automatically fill
the reservoir from a nearby 500~L dewar. This dewar is referred to as
the ``stationary dewar''. In the program, it is possible to set the
desired minimum and maximum levels of the helium in the 4~K reservoir,
and set it to fill automatically. The helium autofill system is shown
in fig.~\ref{fig:ucnarea}. The helium level is measured by a level
meter. The DAQ system and sensor positions are described in
section.~\ref{sec:DAQ} and the gas flow diagram is available in
fig.~\ref{fig:gasflow} of appendix~\ref{app:gasflow}. In addition, two
flow meters were used to observe the gas flow and evaporation of the
superfluid in the 4~K reservoir~(FM4 and FM5 in fig.~\ref{fig:gasflow}
of Appendix~\ref{app:gasflow}). The stationary dewar was filled from
350~L transfer dewars which were in turn filled from the Meson Hall
liquifier.

\begin{figure}[h!]
  \centering
  \includegraphics[width=0.9\textwidth]{ucnarea.png}
  \caption[UCN experimental area during the mini shutdown in October
  2017]{A photograph of the UCN experimental area during the mini
    shutdown in October 2017. Some experimental components are shown
    and are labelled. The yellow concrete blocks are blocking the
    radiation during the target irradiation times. The vertical UCN
    cryostat could be seen because of the removal of some radiation
    shielding. }
  \label{fig:ucnarea}
\end{figure}

Figure~\ref{fig:4kfilling} shows 5 filling cycles of the 4~K reservoir
on April 22, 2017 during the first cooling test. Liquid helium
transfer starts once the liquid level in the 4~K reservoir reaches
20\%. Once the transfer starts, the liquid level starts to decrease
with a sharper slope. This is because it takes some time to cool the
transfer line from the stationary dewar to the reservoir. The warmer
helium vapour causes a heat load to the 4~K reservoir. Any helium gas
evaporated from the reservoir returns through a recovery line to the
liquifier. Liquid helium transfer stops once the 4~K reservoir is 60\%
filled.

\begin{figure}[h!]
  \centering
  \includegraphics[width=1.0\textwidth]{april_4kfilling.png}
  \caption{The 4~K reservoir filling during the cool down test in
    April 2017.}
  \label{fig:4kfilling}
\end{figure}
The efficiency of each transfer from the stationary dewar to the 4~K
reservoir was about 40\% to 60\% on average.

\subsubsection{Liquid Helium Pot at 1 Kelvin}
The 4.2~K liquid helium in the helium reservoir is transferred to a pot
called ``1~K pot''. The flow rate of the transferred liquid helium is
controlled by a needle valve~(JT valve) where the Joule-Thomson
expansion happens and the liquid helium cools down. The Joule-Thomson
expansion occurs when a gas or liquid passes through a valve which has
different temperatures and pressures on both sides, while there is no
heat exchange to the environment.  The 1~K pot is always pumped by a
pumping system. Pumping on the pot provides a pressure drop across the
valve which then gives rise to expansion, and therefore, cooling of
liquid helium down to about 1.4~K via Joule-Thomson~(JT) expansion as
discussed below. The pressure in the pot determines the saturated
vapour pressure above the liquid, and hence, the temperature of the
liquid.  Calculations of refrigerator performance were studied
recently in the context of the vertical UCN source
\cite{Florian_thesis} and in the future horizontal UCN source
\cite{sweitz}. The level of helium in the 1~K pot is measured by a
liquid level meter. The maximum level of the 1.4~K liquid helium is
about 15~cm. At this level, the volume of the 1.4~K liquid helium is
about 1.3~L.

   

\subsubsection{Liquid $^3$He Pot\label{sec:He3pot}}

Once the 1~K pot is ready, the $^3$He gas circulates in a loop from
room temperature to the $^3$He pot. $^3$He is a very valuable gas, so
the entire $^3$He gas system is kept below atmospheric pressure. In
case of a leak, the system will be contaminated, but we will not lose
$^3$He. The $^3$He pot works based on the Joule-Thomson effect as
well, as described in the section above.  To start, the needle valve
in the $^3$He reservoir is opened. A vacuum pump compresses the $^3$He
gas. The $^3$He gas is then purified by a room temperature and a cold
purifier, and is precooled by the 4~K reservoir, and then condensed in
a Cu tube in the 1~K pot. The liquid $^3$He then undergoes
Joule-Thomson~(JT) expansion through NV into the $^3$He pot which is
decompressed by vacuum pumps. The JT expansion is isenthalpic.

%During JT expansion,
%a part of the liquid $^3$He is evaporated. Since the expansion is
%adiabatic, total enthalpy is conserved and the liquid fraction
%remaining after JT expansion can be calculated.

The $^3$He pot is connected to the isotopically pure $^4$He volume by
a copper heat exchanger which conducts the heat in the
isotopically pure $^4$He to the $^3$He pot.
% There are several counter-flow heat exchangers to recover enthalpy
% of the evaporated helium gas.


%Once the 1~K pot is ready, the $^3$He circulation starts to condense
%helium into the ``$^3$He pot''. To start, the needle valve in the $^3$He
%reservoir is opened. A vacuum pump compresses the $^3$He gas. The
%$^3$He gas is then purified by a room temperature and a cold purifier
%and enters the 4~K reservoir to get precooled. The further cooling down
%to 1~K, and condensation happen via the 1~K pot. The liquid $^3$He is
%then transfered to the $^3$He pot, and is further cooled down to 0.7~K
%via pumping. The evaporated $^3$He is pumped out and goes through an
%oil filter and goes back to the beginning point of the circulation.

\subsubsection{Isopure Helium}
After filling the $^3$He pot with 0.7~K liquid $^3$He, the
condensation of the isotopically pure~(isopure) superfluid helium
starts. The isopure helium has much less $^3$He than $^4$He~(less than
$10^{-10}$).  Even though the natural abundance of $^3$He is
$1.37 \times 10^{-6}$ in the atmosphere, this value is still large
because of the large neutron absorption cross section of $^3$He. The
existence of $^3$He causes the UCN storage lifetime to decrease~(see
section~\ref{sec:basic_idea}).

The isopure helium is stored in the isopure helium tank shown in
fig.~\ref{fig:source}. Before entering the cryostat, the isopure
helium goes through a purifier. The purifier is composed of low
temperature charcoals cooled by LN$_2$.  The isopure He is precooled
in the 4~K reservoir and goes into the heat exchange pot attached to
the bottom of the $^3$He cryostat. The bottom of the $^3$He cryostat
and the top of the heat exchange pot is connected via the copper heat
exchanger. The isopure He in the heat exchange pot is cooled by the
0.7~K liquid $^3$He via the copper heat exchanger and becomes
He-II. The condensed He-II fills the He-II bottle with a volume of
8.5~L and is cooled to $\sim$~0.83~K.




\section{Data Acquisition System\label{sec:DAQ}}
The TUCAN UCN DAQ system accumulates data from different devices and
integrates them into a MIDAS file.

For the 2017 data acquisition, almost all the sensors such as
temperature sensors, flow meters, pressure gauges and etc. were
connected to a Programmable Logic Controller~(PLC). The PLC receives
information from the connected sensors or input devices, processes the
data, and triggers outputs based on pre-programmed parameters.
Depending on the inputs and outputs, a PLC can monitor and record
data, automatically start and stop processes, generate alarms based on
the applied limits, and more.

\begin{figure}[h!]
  \centering
  \includegraphics[width=0.8\textwidth, angle = 90]{PLC.JPG}
  \caption[TUCAN's PLC in the meson hall]{A photograph of the PLC in
    the meson hall. The grey terminal blocks are used to connect the
    signal from the devices to the computing modules. The first two
    top rows include the computing modules. Each sensor is connected
    to a specific terminal on a specific module. The bottom row is
    where the power supplies and the fuses are positioned. }
  \label{fig:PLC}
\end{figure}

A picture of the PLC is shown in fig.~\ref{fig:PLC}. The PLC modules
are placed in the top two rows. The bottom row includes the power
supplies and the fuses for the sensors. The middle section of the PLC
box has the terminal blocks for all the sensors. The terminal blocks
for each sensor are labelled. The number of terminal blocks for each
sensor depends on its wiring diagram. The green terminal blocks are
for the ground connection and the grey terminal blocks are for all
other readings.  The cables from the sensors enter the PLC via the two
holes on the top of the PLC box. They are then routed on the right
side of the panel and connected to the designated terminal blocks. The
input connection from the sensors to the terminal blocks enter from
the bottom row and the output connection to the PLC modules leave the
terminal blocks from their top row. We draw the wiring diagram for
each sensor using Altium software.

Figure~\ref{fig:altium} shows the Altium drawing for the PG3H sensor
in the $^4$He system which is named UCN:He4:PG3H. The first part
indicates that it is a sensor for the UCN source, the middle part
indicates which sub-system this system belongs to~(e.g., $^3$He or
$^4$He or UCN guides UGD) and the last part is the name of the sensor
which in this case is pressure gauge 3 high (PG3H). The left wires on
the left side of the drawing show the connection to the modules. Here
we used the red~(RD), black~(BK) and shield~(SH) of a Belden 9462
cable. An orange wire~(OR) then goes to the bottom of the PLC and
connects to a fuse and a violet wire~(VI) connects to the top row~(D0)
module one~(M1) and terminals T7 and T9. The other
values~(UCN:HE4:PG3H and RADCPRESS) are related to the PLC
programming. On the left side of the graph we see wires from the
sensor itself to the terminal block. As the figure shows, the sensor is
a Omega PXM219-1.6bar pressure gauge. To see the location of the
sensor please look at appendix~\ref{app:gasflow}.

\begin{figure}[h!]
  \centering
  \includegraphics[width=1.0\textwidth, angle = 0]{altium_2.png}
  \caption{Altium drawing of the pressure gauge PG3H~(UCN:HE4:PG3H) }
  \label{fig:altium}
\end{figure}



The communication between the PLC and the screen is handled by EPICS.
The EPICS screen defines the user interface for the controls. It
provides readouts of variables, indications of device status, and
various user input controls for turning devices on/off, resetting
devices, etc. The screen shows the approximate physical layout of the
apparatus being controlled, with each device and its controls placed
in its actual location. The colours of the devices are used to
indicate their current status~\cite{Sean_manual}. Figure~\ref{fig:epics}
shows the thermal EPICS screen for the TUCAN vertical UCN source
during the November 2017 experimental run. The gas flow screen~(not
shown, very similar to the thermal screen) is intended to contain all
the information about pressures, flows, levels, and controls for pumps and
valves.

\begin{figure}[h!]
  \centering
  \includegraphics[width=1.0\textwidth]{epics.png}
  \caption[TUCAN's EPICS thermal screen]{EPICS thermal screen. The
    approximate location of each temperature sensor is shown.The
    thermal screen is intended to contain all the information about
    temperatures, and controls for compressors and heaters. }
  \label{fig:epics}
\end{figure}

MIDAS is a modern data acquisition system developed at PSI and TRIUMF
written in C/C++ which runs on all operating systems. MIDAS logs data
in two different ways: History logging where some data is saved
periodically~(every 1-10~s) and can be plotted from history page and
file logging where all data is saved to MIDAS file to be analyzed
later. The TUCAN MIDAS DAQ has a web interface shown in
fig.~\ref{fig:midas}. The green colour indicates that the equipment
front-end is running. Each run can be started by pressing the button at
the top section.

\begin{figure}[h!]
  \centering
  \includegraphics[width=0.9\textwidth]{midas.png}
  \caption{TUCAN MIDAS web interface }
  \label{fig:midas}
\end{figure}

The main MIDAS front-ends that are being read out by the UCN DAQ:

\begin{itemize}
\item Source EPICS front-end (and beam line EPICS) front-end, which
  copies a subset of the EPICS Process Variables~(PVs) to the MIDAS
  history system.  The program that does this is called
  ``feSourceEpics''. On the MIDAS status page the equipment
  ``SourceEpics'' is green and producing new events at a rate of
  $\sim 0.1$~Hz. It is possible to add new history plot or add new
  variables to the existing history plot.
\item A Keithley picoammeter which is reading the current from a
  thermal neutron counter. The picoammeter is readout through a
  MIDAS-GPIB chain; the front-end to control it is called ``scpico''.
\item The V1720 for reading out the UCN $^6$Li detector.
\item A front-end program for controlling UCN sequence. It starts
  triggering after the end of target irradiation.
\item V785 peak sensing ADC for reading out the UCN $^3$He detector.
  This front-end is for digitizing the signal from the $^3$He UCN
  detector.  The program for reading out the digitizer is called
  ``fev785''.  These front-end program controls the readout from a CAEN
  V785 VME module. The trigger~(gate) signal for the V785 is generated
  by a bunch of NIM electronics that process the signal from the
  $^3$He detector.
\end{itemize}

For the 2017 experimental run, each experiment had a unique MIDAS run
number. The MIDAS files were then converted to ROOT files for data
analysis. ROOT is a scientific software package developed by
scientists at CERN~\cite{brun1997root}. It provides all the
functionalities needed to deal with data processing, statistical
analysis, visualization and storage. It is mainly written in C++, but
integrated with other languages such as Python and R. Each ROOT file
represents a particular Midas run and has 8 trees. A tree in ROOT is
like a table. Each row of the table is a branch in the
tree. Figure~\ref{fig:sfig1} shows the trees in a ROOT file. The XXX
indicates the number of the MIDAS run. The UCN Hit trees include the
data time-stamps, ChargeL, ChargeS, PSD~(see
Section~\ref{sec:detectors}), Baseline and etc. The run transition
trees include the UCN valve open time, UCN valve close time, Cycle
start time and etc.~(see Chapter~\ref{chap:UCNresult}). The
LND~(thermal neutron) detector tree has the thermal neutron detector
readings and its time-stamps. The Source Epics tree has the readings
for all the UCN source sensors integrated into the EPICS system, and
the Beam Line Epics tree has all the readings related to the beam
lines such as target temperatures and etc. The header tree includes
the experiment number~(e.g., TCN170XX which represents a TUCAN
experiment in 2017 and XX represents the experiment number), the
person who was on the shift, and the comments they entered on the
MIDAS web interface screen during the shift.

\begin{figure}
\begin{subfigure}{.45\textwidth}
  \centering
  \includegraphics[width=.7\textwidth]{data_struct.pdf}
  \caption{}
  \label{fig:sfig1}
\end{subfigure}%
\begin{subfigure}{.45\textwidth}
  \centering
  \includegraphics[width=.8\textwidth]{data_struct2.png}
  \caption{}
  \label{fig:sfig2}
\end{subfigure}
\caption[TUCAN's data structure]{(a)~A ROOT file and the trees that
  represent tables. (b)~A snapshot of the Source Epics tree with some
  of the branches. Each branch is a row of the Source Epics table. }
\label{fig:data_struct}
\end{figure}

We developed a ROOT program that extracts the UCN cycle information
for each MIDAS run and creates a .root file.  The output file has one
tree as ``cycle information'' table. It consists of branches with the
cycle average, maximum, and minimum values for the main sensors that
contribute to the analysis.  The analysis is based on the UCN cycles,
and it is essential to know, for example, how the temperature of the
superfluid helium changes for each cycle. Therefore, the average,
maximum and minimum values of the superfluid helium temperature for
each cycle for different temperature sensors are calculated and
inserted in the cycle information tree. In addition, the UCN
background, and the total UCN counts for each cycle are also added to
the tree. More details about these variables, and the result of the
analysis is presented in Chapter~\ref{chap:UCNresult}.


%\begin{figure}[h!]
%  \centering
%  \includegraphics[width=0.4\textwidth]{data_struct.pdf}
%  \caption{Left: A ROOT file and the trees }
%  \label{fig:data_struct}
%\end{figure}




\section{UCN Detectors\label{sec:detectors}}
For the TUCAN experimental runs in 2017, a $^6$Li and a $^3$He
detector were used. The main reason two detectors were used was to
check the consistency of the result, and the performance of each
detector. A brief description of each detector is available below.

\subsection{$^6$Li Detector\label{sec:Li6detector}}
The main detector used during the UCN measurements~(see
chapter~\ref{chap:UCNresult}) is a $^6\mathrm{Li}$ glass based
scintillator detector designed and built at the University of Winnipeg
for the TUCAN nEDM experiment at
TRIUMF~\cite{jamieson2017characterization, Lori}. Since
$^6\mathrm{Li}$ has a high neutron capture cross-section~(order of
$10^5$ b) at UCN energies, the scintillator glass is doped with
it. The charged particles in the reaction
\begin{equation}
^6Li + n \rightarrow \alpha (2.05~\mathrm{MeV}) + t (2.73~\mathrm{MeV})
\end{equation}
are detected. The mean range of the $\alpha$ is 5.3~$\mu$m, and the
mean range of the triton is 34.7~$\mu$m. To reduce the effect of
$\alpha$ or triton escaping the glass, a layer of 60~$\mu$m thick
depleted $^6\mathrm{Li}$ glass~(GS30), on top of a layer of 120~$\mu$m
thick dopped $^6\mathrm{Li}$~(GS20) were optically bonded. This design
allows the UCN to pass through the depleted layer before being
captured in the $^6$Li-rich layer, thus the resultant particles
deposit all their energy within the scintillating
glass. Table~\ref{tab:scintillator} shows the content and density of
those $^6\mathrm{Li}$ scintillators.

\begin{table}[h!]
  \centering
  \begin{tabular}{|c|c|c|}
    \hline
    Scintillator & GS20~($^6\mathrm{Li}$ Enriched) & GS30~( $^6\mathrm{Li}$ depleted) \\
    \hline
    Total Li content (\%) & 6.6 & 6.6 \\
    \hline
    $^6\mathrm{Li}$ fraction (\%) & 95 & 0.01 \\
    \hline
    $^6\mathrm{Li}$ density~(cm$^{-3}$) & $1.716 \times 10^{22}$ & $1.806 \times 10^{18}$ \\
    \hline
  \end{tabular}
  \caption{Properties of the glass
    scintillators\label{tab:scintillator}}
\end{table}


Making the scintillating Li glass as thin as possible reduces the
sensitivity to $\gamma$-ray-induced scintillation backgrounds, and
thermal neutron captures. In order to handle high UCN rates of up to
1~MHz, the $^6\mathrm{Li}$ detector face is segmented into 9
tiles~(see fig.~\ref{fig:Li6detector}). The scintillation light of
each tile is guided through ultra-violet transmitting acrylic
light-guide to its corresponding photo-multiplier tube~(PMT) outside
the vacuum region of the detector.

\begin{figure}[h!]
  \centering
  \includegraphics[width=0.5\textwidth]{Li6detector.png}
  \caption[3D drawing of the $^6$Li detector and its enclosure]{3D
    drawing of the $^6$Li detector and its enclosure. The enclosure is
    made of Al, and the rim of the adapter flange which UCN can hit is
    coated with 1~$\mu$m Ni by thermal evaporation. }
  \label{fig:Li6detector}
\end{figure}

The data acquisition for the detector is based on the CAEN V1720
digitizer which has a Pulse-Shape Discrimination~(PSD) firmware. Every
4~ns the digitizer samples the waveform on a 2~V scale into an ADC
value between 0 and 4096. Each channel of the digitizer sends a
trigger, whenever the number of counts in the ADC goes below a certain
baseline~(pedestal) value. The PSD calculates the sum of the signal
below the baseline for two time windows: $t_s = 40$~ns~(short gate)
and $t_L = 200$~ns~(long gate). The short gate is chosen in a way to
contain all of the charge for the $\gamma$-ray interactions in the
light-guide. The ADC sum for during the long gate below the baseline
is called $Q_L$~(read charge long) and for during the short gate below
the baseline is called $Q_S$~(read charge short). Charge long has the
total charge deposited for the neutron capture events. The PSD value is
defined as
\begin{equation}
  \label{eq:psd}
  \mathrm{PSD} = \frac{\left( Q_L - Q_S\right)}{Q_L}~,
\end{equation}  
which is the amount of charge in the tail of an event.

The absolute efficiency of this detector is $89.7^{+1.3}_{-1.9}$~\%
with a background contamination~(largely due to $\gamma$-ray
interactions in the lightguides, also see section~\ref{sec:DQC}) of
$0.3 \pm 0.1$~\%~\cite{jamieson2017characterization}. The detector is
stable at the 0.06~\% level or better, and that the variation in the
efficiency between the detector tiles is less than 5~\%.
\subsection{$^3$He Detector}

The $^3$He detector used for the data acquisition is a Dunia-10 type
which was used at RCNP. $^3$He provides an effective neutron
detector material for neutron detection by absorbing neutrons via the
following reaction

\begin{equation}
  \label{eqn:he3}
n + ^3\mathrm{He} \rightarrow p + t + 674~\mathrm{keV}.
\end{equation}
Before the start of the experiment, the $^3$He detector was tested
with an AmBe source, and it showed consistent result with what was
observed at RCNP. The detector was surrounded with paraffin blocks to
moderate the neutrons~(see fig.~\ref{fig:he3detector}).

\begin{figure}[h!]
  \centering
  \includegraphics[width=0.7\textwidth, angle = 270]{he3detector.png}
  \caption{$^3$He detector and paraffin blocks for neutron
    moderation.}
  \label{fig:he3detector}
\end{figure}
More detail about the $^3$He detector can be found in
Ref.~\cite{matsumiya_thesis}.
% The result of the comparison between the $^3$He and $^6$Li detectors
% are available in Sec.~\ref{sec:detector_comparison}.

%\begin{description}
%\item{An intro to whatever goes into this chapter}

%\item{Start by showing a nice drawing and then talk about each
 % componet of the facility:}
  
%\item{about proton beam that we get, the magnets and basically how the
%  beam reaches the target and how it looks like (Where can I get this
%  information? Is it written somewhere?)}
  
%\item{A short introduction to say the stages of UCN production and why
%  we need the vertical cryostat (Link to the next stage)}
  
%\item{It also has to be mention that it is the same vertical sourcse
%  as was used at the RCNP and some modifications were made to meet the
%  requirements at triumf. (Where can I find what modifications were
%  made?) Agian this has to be just as a link to the next chapter(maybe?)}
  
%\item{The target and shielding (with pictures?), only a few
%  paragraphs}
  
%\item{Moderation: D2O system (I can use Ryohei's thesis I guess)}
  
%\item{conversion. There is a whole chapter dedicated to the UCN
%  cryogenics. I have to go through details (not too much) of how the
%  cryostat works. I can borrow some infromation from Ryohei's
%  thesis. I am not sure how much of it is related to the next
%  chapter.}
 

%\item{Data acquisition system, epics and plc, I guess there are useful
%  informaion in Sean Vanbergen's report that I can use for this
%  section}
  
%\item{what else?}

%\end{description}
