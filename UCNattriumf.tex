\chapter{UCN Facility at TRIUMF\label{chap:UCNattriumf}}


The vertical UCN cryostat at TRIUMF is the same UCN cryostat developed
and tested at KEK-RCNP.  This source is referred to as the vertical
UCN source because the UCN exit the source vertically.  In October
2016, the cryostat was shipped to TRIUMF, and in 2017 it was installed
at a dedicated spallation neutron source for further UCN
experiments. The main purpose of such experiments were for better
understanding of the vertical UCN source, and the design of the next
generation UCN source for higher statistics. The 520~MeV cyclotron at
TRIUMF provides up to 40~$\mu$A of proton beam that can be diverted
onto a tungsten spallation target. The vertical UCN source is placed
above the target and is surrounded by graphite blocks serving as
neutron reflectors.


The vertical source was modified to fulfill the Canadian safety
requirements at TRIUMF. Those include installing pressure relief
valves on the cryostat and the UCN guides, and additional radiation
shielding. The extra shielding requires much longer UCN guides
compared to RCNP. The current location of the vertical source is at
the Meson Hall experimental area. A map of TRIUMF is shown in
Fig.~\ref{fig:sitemap}.

\begin{figure}[h!]
  \centering
  \includegraphics[width=1.0\textwidth]{sitemap.png}
  \caption{A map of TRIUMF. The UCN facility is located at the Meson
    Hall area shown in Blue.}
  \label{fig:sitemap}
\end{figure}

The unique feature of the UCN source at TRIUMF is the combination of
spallation neutrons and superfluid helium for UCN production. The
important elements of the UCN facility at TRIUMF are discribed below.


\section{Proton Beam-line for UCN Facility~(BL1U)}
TRIUMF produces negatively charged hydrogen ions $H^-$ from an ion
source. These ions are then accelerated in the 520~Mev cyclotron in an
outward spiral trajectory. A thin graphite stripper foil removes the
electrons from the hydrogen ion while protons can pass through. The
proton, because it is a positively charged particle, is deflected in
the outward direction due to the magnetic field, and is directed to a
proton beam-line. The cyclotron has three independent extraction
probes with various sizes of foils to provide protons to up to three
beam-lines~(BL) simultaneously~(see Fig.~\ref{fig:cyclotron}).

\begin{figure}[h!]
  \centering
  \includegraphics[width=0.8\textwidth]{cyclotron.png}
  \caption{TRIUMF cyclotron and the three beam-lines.}
  \label{fig:cyclotron}
\end{figure}


The 120~$\mu$A beam~(BL1A) enters the Meson Hall, routinely delivers
protons at 480~MeV to two target systems: T1 and T2 for the $\mu$SR
experimental channels. Beam-line 1B~(BL1B) separates off BL1 at the
edge of the cyclotron vault, and provides international users with the
Proton Irradiation Facility (PIF), which mimics space radiation for
testing computer chips. The new BL1U provides beam to the UCN
source. BL2A provides 480~MeV proton beams for the targets that
produce exotic ion beams for a host of experiments in ISAC facilities.


The microstructure of BL1A is in pulses with approximately 1~ms
periods of beam followed by a 50-100~$\mu$s periods of no beam.  This
is shown in Fig.~(\ref{fig:bl1u})~\cite{Nick_thesis}. A kicker magnet
and the septum magnet kick away 1/3 of the beam from BL1A to BL1U and
transport it to a conventional dipole~(bender) magnet~(see
Fig.~\ref{fig:magnets}).

\begin{figure}[h!]
  \centering
  \includegraphics[width=0.9\textwidth]{magnets.png}
  \caption{The kicker, septum and dipole~(bender) magnets define the
    front two sections of BL1U.}
  \label{fig:magnets}
\end{figure}
The vertical UCN cryostat is sitting above the tungsten target, and is
designed for a maximum of 40~$\mu$A beam on target. As a result, only
one third of the beam can go to the UCN experimental area, and the rest
is shared with other users.

\begin{figure}[h!]
  \centering
  \includegraphics[width=0.9\textwidth]{bl1u.png}
  \caption{UCN beam structure. The top graph shows the 120~$\mu$A BL1A
    in 1~ms period of beam followed by a 50-100~$\mu$s of no
    beam. The middle graph shows the same beam-line when the kicker
    magnet is on. The bottom graph shows the 1/3 of the beam that goes
    to the UCN area.}
  \label{fig:bl1u}
\end{figure}

After the bender magnet, the beam then passes through a cored
shielding block, and reaches the two quadrupole magnets, providing the
final focus of the beam onto a 12~cm thick tungsten spallation target.
The target is located inside a hermetically-sealed target crypt, which
also envelops the beam-line exit window that defines the end of BL1U.
Upstream of the beam-line window, there is a collimator to reduce the
halo from the proton beam, as well as to help reduce the amount of
neutrons and photons streaming back into the beam-line from the target
region~(the collimator also increases the impedance for the passage of
gas arising from any target or window failure, to allow time for the
cyclotron fast valves to close). This last part of the beam-line also
contains a variety of beam position and current monitors. The
spallation target and UCN source, located downstream of the beam
line-exit window, are enclosed in a large shielding
pyramid~Fig.~\ref{fig:pyramid}.
\begin{figure}[h!]
  \centering
  \includegraphics[width=0.8\textwidth]{pyramid.png}
  \caption{Two quadrupole magnets which focus the proton beam onto a
    12~cm thick tungsten spallation target, located inside a
    hermetically-sealed target crypt. Also shown is the UCN shielding
    pyramid, which encases both the spallation target and the UCN
    source, and is designed to meet the dose rate requirements
    specified by the TRIUMF Safety Group.}
  \label{fig:pyramid}
\end{figure}


\section{Tungsten Spallation Target\label{sec:target}}
The spallation target is located at the downstream end of BL1U. The
UCN spallation target comprises a series of rectangular blocks, adding
up to roughly one stopping length~(11~cm) of tungsten, with a
cross-section of $\sim6 \times 8$~cm$^2$. This geometry is very
similar to~(and motivated by) the neutron spallation target design
used at KEK~(KENS facility)~\cite{kawai2001fabrication}: five blocks
of tungsten constitute the target with 78~mm height and 57~mm width,
three with 20~mm length in beam direction and two with 30~mm
lenght~(see Fig.~\ref{fig:target}).
The target requires a support
and cooling system, and is designed to allow for remote-handling and
ease of servicing. The target-cooling and remote-handling systems are
designed for an instantaneous proton current of 40~$\mu$A~(10~$\mu$A
time-averaged).
\begin{figure}[h!]
  \centering
  \includegraphics[width=0.8\textwidth]{target.png}
  \caption{(a) Tungsten Target Blocks from the spallation target at
    KEK. The target blocks are plated with tantalum. (b) Present
    design for the tungsten spallation target at the TRIUMF UCN
    facility. The target blocks have a cross-section of
    $5.7 \times 7.8$~cm$^2$ , and thicknesses of 2.0, 2.0, 3.0, and
    5.0~cm, respectively.}
  \label{fig:target}
\end{figure}
The target is water cooled. A water flow of approximately
0.8~L/s cools the target. Horizontal channels around the blocks create
a uniform flow. To reduce the beam absorption in the water the last
two blocks are thicker. Cooling water corrode tungsten.  Therefore, a
coating of tantalum with a thickness of $< 0.1$~mm prevents corrosion
by the water-cooling system. The estimated lifetime of the target is
longer than 10 years. An extraction system allows to exchange the
target when necessary.


\section{Vertical UCN Source at TRIUMF\label{sec:vertical_source}}
At TRIUMF, neutrons are produced via the spallation process by hitting
a tungsten target with a proton beam. Spallation is refered to a
nuclear reaction where high energy particles interact with atomic
nucleus. This process creates many high energy neutrons and background
radiation.  Since the temperature of the superfluid helium is crucial,
the heat load into it should be kept as small as possible. Two effects
dominate the heat load: (1) heating by prompt $\gamma$ radiation and
neutrons; and (2) residual heating by radioactive decays of materials
activated by neutron capture.  Any heat deposited in the converter
vessel and the connected UCN guide will directly contribute to the
heat load, so they should be thin-walled and made out of a material
with low density, low atomic number, and low neutron-absorption cross
section. Additionally, the vessel and UCN guide should be leak-tight
for superfluid helium, have a high optical potential for UCN, or allow
coating with a suitable material. The cold- and thermal-moderator
vessels have less direct impact on the heat load into the UCN con-
verter, but secondary particles from (n, $\gamma$) reactions and
radioactive decays can still contribute.




The target is surrounded by several blocks of lead and
graphite. The fast neutrons are reflected and moderated down and enter
the warm D$_2$O moderator at room temperature~(300~K) and become
thermal neutrons with an energy of 0.025~eV, and the speed of 2.2~km/s.
Iced heavy water at 10~K is used as a cold moderator. After passing
through the warm D$_2$O, thermal neutrons enter the the cold moderator
and become cold neutrons. These neutrons have the speed of several
hundreds of meter per second.  UCN are produced when the slow neutrons
enter the isotopically pure superfluid helium at 0.84 to 0.92~K as a
result of phonon transitions inside the superfluid helium as discussed
in Section~\ref{sec:ucn_with_heII}.


\begin{figure}[h!]
  \centering
  \includegraphics[width=1.1\textwidth]{Source_all.png}
  \caption{The UCN source and the guide geometry at TRIUMF }
  \label{fig:Source_all}
\end{figure}




\begin{figure}[h!]
  \centering
  \includegraphics[width=0.9\textwidth]{vertical_source.png}
  \caption{Schematic diagram of the vertial UCN source at
    TRIUMF. Spallation neutrons are moderated in warm D$_2$O vessel
    and become cold neutrons in Iced D$_2$O. The cold neutrons then
    enter the superfluid helium bottle where they become UCN by phonon
    excitations in the superfluid. The isotopically pure superfluid
    helium is cooled down to below 1~K via a $^3$He pot. The $^3$He
    pot is cooled down to 0.7~K via the 1~K pot and further
    pumping~(see text for more details). }
  \label{fig:source}
\end{figure}



The schematic of the vertical source is shown in Fig.~\ref{fig:source}
and a 3D drawing of the vertical source and the guide geometry is
shown in Fig.~\ref{fig:Source_all}. The neutron moderators and the
helium circulation system are explained below.


\subsection{Neutron D$_2$O Moderators}
Deuterium is an isotope of hydrogen which has one proton and one
neutron in the nucleus, and it has a lower probability to absorb
neutrons. As a result, heavy water is used as a neutron moderator. The
warm D$_2$O moderator to create thermal neutrons from spallation
neutrons is at room temperature. However, the cold moderator for the
production of cold neutrons is at much lower
temperatures~($\sim$~10~K).

\subsubsection{D$_2$O Solidification}
The Iced D$_2$O vessel has a capacity of 100~L. About 14~L of liquid
D$_2$O is injected to the vessel initially. This is followed by adding 11~L of
D$_2$O to the vessel 8 times.  After filling up the vessel,
Gifford McMahon refrigerators solidify the heavy water and further
cool it down to 20~K. The process of icing the heavy water takes about
6 days and cooling it down to 20~k takes another 7 days.


\subsection{Helium Circulation and Superfluid Helium Condensation}
The helium circulation and the superfluid helium condensation could
get started once the temperature of D$_2$O is as low as 10~K. The
stages towards superfluid helium condensation is presented below. The
full operation and design details are available in
Ref.~\cite{matsumiya_thesis}.

\subsubsection{Liquid Helium Reservoir}
The first step of the helium circulation is to fill up the helium
reservoir with the commercially available 4.2~K helium. The full
capacity of the helium reservoir is 50~L. In the 2017 experimental
run, the TUCAN collaboration used a labview program to automatically
fill up the reservoir using a 500~L dewar of 4.2~K helium. This dewar
was refered to as the ``stationary dewar''. This way, it is possible
to set the desired minimum and maximum levels of the helium in the 4~K
reservoir, and set it to automatically get filled once it hits the
minimum value, and stop filling once it reaches the maximum value. The
helium levels were measured by level meters. In addition, two flow
meteres were used to observe the gas flow and evaporation of the
superfluid in the 4~K reservoir~(FM4 and FM5 in
Fig.~\ref{fig:gasflow}. The DAQ system and sensor positions are
described in Section.~\ref{sec:DAQ} and the gas flow diagram is
available in Fig.~\ref{fig:gasflow}. The stationary dewar was filled
up with 350~L dewars~( we reffered to them as ``transfer dewar'') of
4~K helium from the meson hall liquifier. The helium autofill system
is shown in Fig.~\ref{fig:ucnarea}.

\begin{figure}[h!]
  \centering
  \includegraphics[width=0.9\textwidth]{ucnarea.png}
  \caption{A photograph of the UCN experimental area during the mini
    shutdown in October 2017. Some experimental components are shown
    and are labeled. The yellow concrete blocks are blocking the
    radiation during the target irradiation times. The vertical UCN
    cryostat could be seen because of the removal of some radiation
    shielding. }
  \label{fig:ucnarea}
\end{figure}

Fig.~\ref{fig:4kfilling} shows the 5 filling cycles of the 4~K
reservoir on April 22, 2017 during the first cool down test. The
liquid helium transfer starts once the liquid level in the 4~K
reservoir reaches 20\%. Once the transfer starts, the liquid level
starts to decrease with a sharper slope. This is because of the
introduced heat load to the reservoir. It takes some time to cool down
the transfer line from the stationary dewar to the reservoir. The warm
liquid helium causes a boil off in the 4~K reservoir. The boiled off
helium gas goes through the recovery line to the liquifier. The liquid
helium transfer stops once the 4~K reservoir is 60\% filled.

\begin{figure}[h!]
  \centering
  \includegraphics[width=1.0\textwidth]{april_4kfilling.png}
  \caption{The 4~K reservoir filling during the cool down test in April 2017.}
  \label{fig:4kfilling}
\end{figure}
The efficiency of each transfer from the stationary dewar to the 4~K
reservoir was about 40\% to 60\% on average.

\subsubsection{Liquid Helium Pot at 1 Kelvin}
The 4.2~K liquid helium in the helium reservoir is transfered to a
pot called ``1~K pot''. The flow rate of the transferred liquid
helium is controlled by a needle valve. The 1~K pot is always pumped
by a pumping system to cool the 4.2~K helium down to about 1.4~K. The
level of helium in the 1~K pot is measured by a liquid level
meter. The maximum level of the 1.4~K liquid helium is about 15~cm. At
this level, the volume of the 1.4~K liquid helium is about 1.3~L.


\subsubsection{Liquid $^3$He Pot}

Isotopically pure $^4$He is cooled by the latent heat of decompressed,
liquid $^3$He inside a vessel called ``$^3$He pot''.


Once the 1~K pot is ready, the $^3$He gas circulates in a loop from
room temperature to the $^3$He pot. $^3$He is a very valuable gas, so
the entire $^3$He gas system is kept below atmospheric pressure. In
case of a leak, the system will be contaminated, but we will not lose
$^3$He. To start, the needle valve in the $^3$He reservoir is
opened. A vacuum pump compresses the $^3$He gas. The $^3$He gas is
then purified by a room temperature and a cold purifier and is
precooled by the $^4$K reservoir, and then condensed in the 1~K
pot. The liquid $^3$He then undergoes Joule-Thomson expansion into the
$^3$He pot which is decompressed by vacuum pumps. Joule-Thomson~(JT)
expansion occurs at a valve. During JT expansion, a part of the liquid
$^3$He is evaporated. Since the expansion is adiabatic, total enthalpy
is conserved and the liquid fraction remaining after JT expansion can
be calculated.

The $^3$He pot is connected to the isotopically pure $^4$He volume by
a copper heat exchanger which conducts the heat produced in the
isotopically pure $^4$He to the $^3$He pot. There are several
counter-flow heat exchangers to recover enthalpy of the evaporated
helium gas.


%Once the 1~K pot is ready, the $^3$He circulation starts to condense
%helium into the ``$^3$He pot''. To start, the needle valve in the $^3$He
%reservoir is opened. A vacuum pump compresses the $^3$He gas. The
%$^3$He gas is then purified by a room temperature and a cold purifier
%and enters the 4~K reservoir to get precooled. The further cooling down
%to 1~K, and condensation happen via the 1~K pot. The liquid $^3$He is
%then transfered to the $^3$He pot, and is further cooled down to 0.7~K
%via pumping. The evaporated $^3$He is pumped out and goes through an
%oil filter and goes back to the beginning point of the circulation.

\subsubsection{Isopure Helium}
After filling the $^3$He pot with 0.7~K liquid $^3$He, the
condensation of the isotopically pure~(isopure) superfluid helium
starts. The isopure helium has much less $^3$He than $^4$He~(less than
$10^{-10}$).  Even though the natural abundance of $^3$He is
$1.37 \times 10^{-6}$ in the atmosphere, this value is still large
because of the large neutron absorption cross section of $^3$He. The
existence of $^3$He causes the UCN storage lifetime to decrease~(see
Section~\ref{sec:basic_idea}).

The isopure helium is stored in the isopure helium tank shown in
Fig.~\ref{fig:source}. Before entering the cryostat, the isopure
helium goes through a purifier. The purifier is composed of low
temperature charcoals cooled by LN$_2$.  The isopure He is precooled
in the 4~K reservoir and goes into the heat exchange pot attached to
the bottom of the $^3$He cryostat. The bottom of the $^3$He cryostat
and the top of the heat exchange pot is connected via the copper heat
exchanger. The isopure He in the heat exchange pot is cooled by the
0.7~K liquid $^3$He via the copper heat exchanger and becomes
He-II. The condensed He-II fills the He-II bottle with a volume of
8.5~L and gets cooled down to $\sim$~0.83~K.




\section{Data Acquisition System\label{sec:DAQ}}
The TUCAN UCN DAQ system accumulates data from different devices and
integrates them into a MIDAS file.

For the 2017 data acquisition, almost all the sensors such as
temperature sensors, flow meteres, pressure gaugas and etc., were
connected to a Programmable Logic Controller~(PLC). The PLC receives
information from the connected sensors or input devices, processes the
data, and triggers outputs based on pre-programmed parameters.
Depending on the inputs and outputs, a PLC can monitor and record
data, automatically start and stop processes, generate alarms based on
the applied limits, and more.

\begin{figure}[h!]
  \centering
  \includegraphics[width=0.8\textwidth, angle = 90]{PLC.JPG}
  \caption{A photograph of the PLC in the meson hall. The grey
    terminal blocks are used to connect the signal from the devices to
    the computing moduels. The first two top rows include the
    computing modules. Each sensor is connected to a specific terminal
    on a specific module. The bottom row is where the power supplies
    and the fuses are positioned. }
  \label{fig:PLC}
\end{figure}

A picture of the PLC is shown in Fig.~\ref{fig:PLC}. The PLC modules
are placed in the top two rows. The bottom row includes the power
supplies and the fuses for the sensors. The middle section of the PLC
box has the terminal blocks for all the sensors. The terminal blocks
for each sensor are labeled. The number of terminal blocks for each
sensor depends on its wiring diagram. There green terminal blocks are
for the ground connection and the grey termianl blocks are for all
other readings.  The cables from the sensors enter the PLC via the two
holes on the top of the PLC box. These cables are then routed on the
right side of the pannel and connected to the designated terminal
blocks. The input connection from the sensors to the terminal blocks
enter from the bottom row and the output connection to the PLC modules
leave the terminal blocks from their top row. We draw the wiring
diagram for each sensor using Altium software.

Fig.~\ref{fig:altium} shows the altium drawing for the PG3H sensor in
the $^4$He system which is named UCN:He4:PG3H. The first part
indicates that it is a sensor for the UCN source, the middle part
indicates which sub-system this system blongs to e.g., $^3$He or
$^4$He or UCN guides~(UGD) and the last part is the name of the sensor
which in this case is pressure gauge 3 high (PG3H). The left wires on
the left side of the drawing show the connection to the modules. Here
we used the red~(RD), black~(BK) and shield~(SH) of a Belden 9462
cable. An orange wire~(OR) then goes to the bottom of the PLC and
connects to a fuse and a violet wire~(VI) connects to the top row~(D0)
module one~(M1) and terminals T7 and T9. The other
values~(UCN:HE4:PG3H and RADCPRESS) are related to the PLC
programming. On the left side of the graph we see wires from the
sensor itself to the terminal block. As the Fig. shows, the sensor is
a Omega PXM219-1.6bar pressure gauge. To see the location of the
sensor please look at appendix~\ref{app:gasflow}.

\begin{figure}[h!]
  \centering
  \includegraphics[width=1.0\textwidth, angle = 0]{altium_2.png}
  \caption{Altium drawing of the pressure gauge PG3H~(UCN:HE4:PG3H) }
  \label{fig:altium}
\end{figure}



The communication between the PLC and the screen is handled by EPICS.
The EPICS screen defines the user interface for the controls. It
provides readouts of variables, indications of device status, and
various user input controls for turning devices on/off, resetting
devices, etc. The screen shows the approximate physical layout of the
apparatus being controlled, with each device and its controls placed
in its actual location. The colours of the devices are used to
indicate their current status~\cite{Sean_manual}. Fig.~\ref{fig:epics}
shows the thermal EPICS screen for the TUCAN vertical UCN source
during the November 2017 experimental run. The gas flow screen~(not
shown, very similar to the thermal screen) is intended to contain all
the information about pressures, flows, levels, and controls for pumps and
valves.

\begin{figure}[h!]
  \centering
  \includegraphics[width=1.0\textwidth]{epics.png}
  \caption{EPICS thermal screen. The approximate location of each
    temperature sensor is shown.The thermal screen is intended to
    contain all the information about temperatures, and controls for
    compressors and heaters. }
  \label{fig:epics}
\end{figure}

MIDAS is a modern data acquisition system developed at PSI and TRIUMF
written in C/C++ which runs on all operating systems. MIDAS logs data
in two different ways: History logging where some data is saved
periodically~(every 1-10~s) and can be plotted from history page and
file logging where all data is saved to MIDAS file to be analyzed
later. The TUCAN MIDAS DAQ has a web interface shown in
Fig.~\ref{fig:midas}. The green color indicates that the equipment
frontend is running. Each run can be started by pressing the button at
the top section.

\begin{figure}[h!]
  \centering
  \includegraphics[width=0.9\textwidth]{midas.png}
  \caption{TUCAN MIDAS web interface }
  \label{fig:midas}
\end{figure}

The main MIDAS frontends that are being read out by the UCN DAQ:

\begin{itemize}
\item Source EPICS frontend (and beamline EPICS) frontend, which
  copies a subset of the EPICS Process Variables~(PVs) to the MIDAS
  history system.  The program that does this is called
  ``feSourceEpics''. On the MIDAS status page the equipment
  ``SourceEpics'' is green and producing new events at a rate of
  $\sim 0.1$~Hz. It is possible to add new history plot or add new
  variables to the existing history plot.
\item A Keithley picoammeter which is reading the current from a
  thermal neutron counter. The picoammeter is readout through a
  MIDAS-GPIB chain; the frontend to control it is called ``scpico''.
\item The V1720 for reading out the UCN $^6$Li detector.
\item A frontend program for controlling UCN sequence. It starts
  triggering after the end of target irradiation.
\item V785 peak sensing ADC for reading out the UCN $^3$He detector.
  This frontend is for digitizing the signal from the $^3$He UCN
  detector.  The program for reading out the digizer is called
  ``fev785''.  These frontend program controls the readout from a CAEN
  V785 VME module. The trigger~(gate) signal for the V785 is generated
  by a bunch of NIM electronics that process the signal from the
  $^3$He detector.
\end{itemize}

For the 2017 experimental run, each expriment had a unique MIDAS run
number. The MIDAS files were then converted to ROOT files for data
analysis. ROOT is a scientific software package developed by
scientists at CERN~\cite{brun1997root}. It provides all the
functionalities needed to deal with big data processing, statistical
analysis, visualisation and storage. It is mainly written in C++, but
integrated with other languages such as Python and R. Our data was
saved as .root format. Each data file represents a particular midas
run and has 8 trees. A tree in ROOT is like a table. Each row of the
table is a branch in the tree. Fig.~\ref{fig:sfig1} shows the trees in
a ROOT file. The XXX indicates the number of the MIDAS run. The UCN
Hit trees include the data timestamps, ChargeL, ChargeS, PSD~(see
Section~\ref{sec:detectors}), Baseline and etc. The run transition
trees include the UCN valve open time, UCN valve close time, Cycle
start time and etc.~(see Chapter~\ref{chap:UCNresult}). The LND
Detector tree has the thermal neutron detector readings and its
timestamps. The Source Epics tree has the readings for all the UCN
source sensors integrated into the EPICS system, and the Beam Line
Epics tree has all the readings related to the beamlines such as
target temperatures and etc. The header tree includes the experiment
number~(e.g., TCN170XX which represents a TUCAN experiment in 2017 and
XX represents the experiment number), the person who was on the shift,
and the comments they entered on the MIDAS web interface screen during
the shift.

\begin{figure}
\begin{subfigure}{.45\textwidth}
  \centering
  \includegraphics[width=.7\textwidth]{data_struct.pdf}
  \caption{}
  \label{fig:sfig1}
\end{subfigure}%
\begin{subfigure}{.45\textwidth}
  \centering
  \includegraphics[width=.8\textwidth]{data_struct2.png}
  \caption{}
  \label{fig:sfig2}
\end{subfigure}
\caption{(a)~A ROOT file and the trees that represent tables. (b)~A
  snapshot of the Source Epics tree with some of the branches. Each
  branch is a row of the Source Epics table. }
\label{fig:data_struct}
\end{figure}

We developed a ROOT program that extracts the UCN cycle information
for each MIDAS run and creates a .root file.  The output file has one
tree as ``cycle information'' table. It consists of branches with the
cycle average, maximum, and minimum values for the main sensors that
contribute to the analysis.  The analysis is based on the UCN cycles,
and it is essential to know, for example, how the temperature of the
superfluid helium changes for each cycle. Therefore, the average,
maximum and mimimum values of the superfluid helium temperature for
each cycle for different temperature sensors are calculated and
inserted in the cycle information tree. In addition, the UCN
background, and the total UCN counts for each cycle are also added to
the tree. More details about these variables, and the result of the
analysis is presented in Chapter~\ref{chap:UCNresult}.


%\begin{figure}[h!]
%  \centering
%  \includegraphics[width=0.4\textwidth]{data_struct.pdf}
%  \caption{Left: A ROOT file and the trees }
%  \label{fig:data_struct}
%\end{figure}




\section{UCN Detectors\label{sec:detectors}}
For the TUCAN experimental runs in 2017, a $^6$Li and a $^3$He
detector were used. The main reason for this was to check the
consistency of the result, and the performance of each detector. The
brief description of each detector is available below.

\subsection{$^6$Li Detector\label{sec:Li6detector}}
The main detector used during the UCN measurements~(see
Chapter~\ref{chap:UCNresult}) is a $^6\mathrm{Li}$ glass based
scintillator detector designed and built at the University of Winnipeg
for the TUCAN nEDM experiment at
TRIUMF~\cite{jamieson2017characterization}. Since $^6\mathrm{Li}$ has
a high neutron capture cross-section~(order of $10^5$ bn) at UCN
energies, the scintillator glass is doped with it. The charged
particles in the reaction
\begin{equation}
^6Li + n \rightarrow \alpha (2.05~\mathrm{MeV}) + t (2.73~\mathrm{MeV})
\end{equation}
are detected. To reduce the effect of $\alpha$ or triton escaping the
glass, a layer of 60~$\mu$m thick depleted $^6\mathrm{Li}$
glass~(GS30), on top of a layer of 120~$\mu$m thick dopped
$^6\mathrm{Li}$~(GS20) were optically bonded. This design allows the
resultant particles to deposit all of their energy within the
scintillating glass. Table~\ref{tab:scintillator} shows the content
and density of those $^6\mathrm{Li}$ scintillators.

\begin{table}[h!]
  \centering
  \label{tab:scintillator}
  \begin{tabular}{|c|c|c|}
    \hline
    Scintillator & GS20~($^6\mathrm{Li}$ Enriched) & GS30~( $^6\mathrm{Li}$ depleted) \\
    \hline
    Total Li content (\%) & 6.6 & 6.6 \\
    \hline
    $^6\mathrm{Li}$ fraction (\%) & 95 & 0.01 \\
    \hline
    $^6\mathrm{Li}$ desity~(cm$^{-3}$) & $1.716 \times 10^{22}$ & $1.806 \times 10^{18}$ \\
    \hline
  \end{tabular}
  \caption{Properties of the glass scintillators}
\end{table}


Making the scintillating Li glass as thin as possible reduces the
sensitivity to $\gamma$-ray scintillating backgrounds, and thermal
neutron captures. The mean range of the $\alpha$ is 5.3~$\mu$m, and the
mean range of the triton is 34.7~$\mu$m. This means, if the
thickness of the scintillator is less than 50~$\mu$m, the resultant
particles escape before stopping which gives rise to an efficiency
loss.  In order to handle high UCN rates of up to \~1~MHz, the
$^6\mathrm{Li}$ detector face is segmented into 9 tiles~(see
Fig.~\ref{fig:Li6detector}). The scintillation light is then guided
through ultra-violet transmitting acrylic light-guide to its
corresponding Photomultiplier Tube~(PMT) outside the vaccum region of
the detector.

\begin{figure}[h!]
  \centering
  \includegraphics[width=0.5\textwidth]{Li6detector.png}
  \caption{3D drawing of the $^6$Li detector and its enclosure. The
    enclosure is made of Al, and the rim of the adapter flange which
    UCN can hit is ccoated with 1~$\mu$m Ni by thermal evaporation. }
  \label{fig:Li6detector}
\end{figure}

The data acquisition with this detector includes a CAEN V1720
digitizer which has a Pulse-Shape Discrimination~(PSD) firmware that
triggers on pulses below a certain threshold for each channel. Every
4~ns the digitizer samples the waveform which is then digitized to a
voltage on a 2~V scale into an ADC value between 0 and 4096. Each
channel of the digitizer sends a trigger, whenever the number of
counts in the ADC goes below a certain baseline~(pedestal) value. The
PSD calculates the sum of the signal below the baseline for two time
windows: $t_s = 40$~ns~(short gate) and $t_L = 200$~ns~(long
gate). The short gate is chosen in a way to contain all of the charge
for the $\gamma$-ray interactions in the light-guide. The ADC sum for
during the long gate below the baseline is calle $Q_L$~(read charge
long) and for during the short gate below the baseline is called
$Q_S$~(read charge short). Charge long has the total charge deposit
for the neutron capture events. The PSD value is defined as
\begin{equation}
  \label{eq:psd}
  \mathrm{PSD} = \frac{\left( Q_L - Q_S\right)}{Q_L}~,
\end{equation}  
which is the amount of charge in the tail of an event.

Jamieson {\it{et al.}}, showed that the absolute efficiency of this
detector is $89.7^{+1.3}_{-1.9}$~\% with a background contamination of
$0.3 \pm 0.1$~\%~\cite{jamieson2017characterization}. The detector is
stable at the 0.06~\% level or better, and that the variation in the
efficiency between the detector tiles is less than 5~\%.
\subsection{$^3$He Detector}

The $^3$He detector used for the data acquisition is a Dunia-10 type
which was shipped from RCNP.  $^3$He provides an effective neutron
detector material for neutron detection by absorbing neutrons via the
following reaction

\begin{equation}
  \label{eqn:he3}
n + ^3\mathrm{He} \rightarrow p + t + 674~\mathrm{keV}.
\end{equation}

Before the start of the experiment, the $^3$He detector was tested
with an AmBe source, and it showed consistent result with what was
observed at RCNP. The detector was surrounded with paraffin blocks to
moderate the neutrons~(see Fig.~\ref{fig:he3detector}).

\begin{figure}[h!]
  \centering
  \includegraphics[width=0.7\textwidth, angle = 270]{he3detector.png}
  \caption{$^3$He detector and paraffin blocks for neutron
    moderation.}
  \label{fig:he3detector}
\end{figure}

More detail about the $^3$He detector could be found in
Ref.~\cite{matsumiya_thesis}.
% The result of the comparison between the $^3$He and $^6$Li detectors
% are available in Sec.~\ref{sec:detector_comparison}.

%\begin{description}
%\item{An intro to whatever goes into this chapter}

%\item{Start by showing a nice drawing and then talk about each
 % componet of the facility:}
  
%\item{about proton beam that we get, the magnets and basically how the
%  beam reaches the target and how it looks like (Where can I get this
%  information? Is it written somewhere?)}
  
%\item{A short introduction to say the stages of UCN production and why
%  we need the vertical cryostat (Link to the next stage)}
  
%\item{It also has to be mention that it is the same vertical sourcse
%  as was used at the RCNP and some modifications were made to meet the
%  requirements at triumf. (Where can I find what modifications were
%  made?) Agian this has to be just as a link to the next chapter(maybe?)}
  
%\item{The target and shielding (with pictures?), only a few
%  paragraphs}
  
%\item{Moderation: D2O system (I can use Ryohei's thesis I guess)}
  
%\item{conversion. There is a whole chapter dedicated to the UCN
%  cryogenics. I have to go through details (not too much) of how the
%  cryostat works. I can borrow some infromation from Ryohei's
%  thesis. I am not sure how much of it is related to the next
%  chapter.}
 

%\item{Data acquisition system, epics and plc, I guess there are useful
%  informaion in Sean Vanbergen's report that I can use for this
%  section}
  
%\item{what else?}

%\end{description}
